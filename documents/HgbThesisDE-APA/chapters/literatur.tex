\chapter[Umgang mit Literatur]{Umgang mit Literatur und anderen Quellen}
\label{cha:Literatur}

Der APA-Zitierstil%
\footnote{\url{https://apastyle.apa.org/style-grammar-guidelines/references/}}
benötigt unterschiedliche Zitationsmakros, abhängig von der Verwendung \bzw dem Auftreten der Quelle im Text.

\section{Narrative Verweise}

Bei narrativen Verweisen (engl. "narrative citations") wird die Quelle wie ein Subjekt oder Objekt des Satzes
verwendet. Die Jahreszahl wird dabei dem Autor*innenname in Klammern nachgestellt. Das zugehörige Makro ist
%
\begin{itemize}
\item[] \verb!\textcite{!\textit{keys}\verb!}!.
\end{itemize}
%
\textbf{Beispiel:}
\textcite{Daniel2018} geben eine kurze Einführung in das Thema \latex, wohingegen \textcite{Oetiker2021, Kopka2003}
bereits mehr ins Detail gehen.


\section{Narrative Verweise innerhalb von Klammern}

Soll ein Verweis \emph{innerhalb} von Klammern verwendet werden, so müssen diese bei der Quellenangabe
selbst entfallen. Dazu verwendet man das Makro 
%
\begin{itemize}
\item[] \verb!\nptextcite{!\textit{keys}\verb!}!.
\end{itemize}
%
\textbf{Beispiel:}
Auf jeden Fall empfiehlt es sich, Literatur zum Thema \latex zu besorgen (\zB \nptextcite{Daniel2018, Oetiker2021, Kopka2003}).

\section{Parenthetische Verweise}

Parenthetische Verweise (engl. "parenthetical citations") werden verwendet, wenn die Quelle am
Ende eines Satzes oder einer Aussage angegeben werden soll. Autor*innen\-name und Jahreszahl
werden dabei gemeinsam in Klammern gesetzt und durch einen Beistrich getrennt.
Das verwendete Makro ist
%
\begin{itemize}
\item[] \verb!\parencite{!\textit{keys}\verb!}!.
\end{itemize}
%
\textbf{Beispiel:}
Für \latex existieren sowohl kurze Einführungen \parencite{Daniel2018} wie auch umfangreichere Werke 
\parencite{Oetiker2021, Kopka2003}.

