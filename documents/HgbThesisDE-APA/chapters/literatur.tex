\chapter[Umgang mit Literatur]{Umgang mit Literatur und anderen Quellen}
\label{cha:Literatur}

Der APA-Zitierstil benötigt verschiedene Zitationsmakros, abhängig von der Verwendung \bzw. dem Auftreten der Quelle im Text.

\section{Narrative Verweise}

Bei narrativen Verweisen (engl. "narrative citations") wird die Quelle wie Subjekt oder Objekt des Satzes verwendet. Die Jahreszahl wird dabei dem Autor*innenname in Klammern nachgestellt.

Das verwendete Makro ist \verb|\textcite{bibkey}|.

\subsection{Beispiel}

\textcite{Daniel2018} geben eine kurze Einführung in das Thema \latex, wohingegen \textcite{Oetiker2018, Kopka2003} bereits mehr ins Detail gehen.

\section{Narrative Verweise innerhalb von Klammern}

Soll ein Verweis innerhalb von Klammern verwendet werden, so müssen diese bei der Quellenangabe selbst entfallen.

Das Makro \verb|\nptextcite{bibkey}| beachtet dies.

\subsection{Beispiel}

Auf jeden Fall empfiehlt es sich, Literatur zum Thema \latex besorgen (\zB \nptextcite{Daniel2018, Oetiker2018, Kopka2003}).

\section{Parenthetische Verweise}

Parenthetische Verweise (engl. "parenthetical citations") werden verwendet, wenn die Quelle am Ende eines Satzes oder einer Aussage angegeben werden soll. Die Autor*innenname und Jahreszahl werden dabei gemeinsam in Klammern gesetzt.

Das verwendete Makro ist \verb|\parencite{bibkey}|.

\subsection{Beispiel}

Für \latex existieren sowohl kurze Einführungen \parencite{Daniel2018}, als auch umfangreichere Werke \parencite{Oetiker2018, Kopka2003}.

