\chapter[References to Literature]{Handling References to Literature and Other Sources}
\label{cha:Literature}

\paragraph{Note:} The title of this chapter is intentionally lengthy, so it no
longer fits in the page header. For this case, a shortened text for the header
and the table of contents can be specified by providing an
optional argument \verb![..]! to the \verb!\chapter! statement:
%
\begin{LaTeXCode}[numbers=none]
\chapter[References to Literature]{Adding References to Literature and Other ...}
\end{LaTeXCode}


\section{General Remarks}

The correct use of references is essential when writing scientific documents
(see also Section~\ref{sec:plagiarism}). Various guidelines are available for
the design of references, determined among other things by the respective 
subject area or guidelines from publishers and universities. 
This template provides a scheme that is common in engineering and scientific
disciplines.%
\footnote{Adaptation to other specifications is relatively easy.}
Technically, this part is based on the program \texttt{biber}%
\footnote{\url{https://ctan.org/pkg/biber}}
in combination with the \latex\ package \texttt{biblatex} \cite{Kime2023}.

Reference management consists of two elements: \emph{citations} in the text
refer to entries in the \emph{bibliography} (or multiple bibliographies). The
bibliography is a compilation of all references, typically placed at the end of
the document. Each citation must have an associated, unique entry in the
bibliography; each item in the bibliography must likewise be referenced in the
text.


\section{Citations}

Citations can be specified in several ways. Scenarios range from citing a single
reference to references with additional information, such as a page number, to
citing several different references simultaneously.

\subsection{The \texttt{\textbackslash cite} Command}

To create an entry in the bibliography and refer to it in the text, \latex\
provides a central command. For citations in the running text, use the
statement
%
\begin{itemize}
\item[] \verb!\cite{!\textit{keys}\verb!}! \quad or \quad
				\verb!\cite[!\textit{text}\verb!]{!\textit{keys}\verb!}!.
\end{itemize}
%
Here \textit{keys} is a comma-separated list of one or more citation keys to
identify the corresponding entries in the bibliography, and \textit{text} can
specify a supplementary text to the current citation, such as chapter or page
references for books. Below are some examples:
%
\begin{itemize}
    \item For further details, see \cite{Kopka2003}.
\begin{LaTeXCode}[numbers=none]
For further details, see \cite{Kopka2003}.
\end{LaTeXCode}
%
    \item For further details, see \cite[Ch.~3]{Kopka2003}.
\begin{LaTeXCode}[numbers=none]
For further details, see \cite[Ch.~3]{Kopka2003}.
\end{LaTeXCode}
%
    \item The data in \cite[pp.~274--277]{BurgeBurger1999} appear to be outdated.
\begin{LaTeXCode}[numbers=none]
The data in \cite[pp.~274--277]{BurgeBurger1999} appear to be  outdated.
\end{LaTeXCode}
%
    \item Also important are \cite{Patashnik1988,Feder2006,Duden1997}.
\begin{LaTeXCode}[numbers=none]
Also important are \cite{Patashnik1988,Feder2006,Duden1997}.
\end{LaTeXCode}
\end{itemize}
%
In the last example, several references are listed in a single
\texttt{\textbackslash cite} command. Note that the entries are sorted
automatically (numerically or alphabetically). Multiple consecutive
\texttt{\textbackslash cite} commands should not be used for this.

\subsection{Multiple References With Additional Texts}

Attaching texts to several sources simultaneously, for example, to indicate the
respective page numbers, is complex. For this purpose, the
\texttt{hagenberg-thesis} package offers the additional command%
\footnote{\texttt{\textbackslash mcite} is defined in \texttt{hgbbib.sty} and
works similar to the \texttt{{\bs}cites} command of \texttt{biblatex}
(see \url{http://mirrors.ctan.org/macros/latex/contrib/biblatex/doc/biblatex.pdf}).}
%
\begin{itemize}
\item[]
\verb!\mcite[!\textit{text1}\verb!]{!\textit{key1}\verb!}!%
      \verb![!\textit{text2}\verb!]{!\textit{key2}\verb!}!\ldots%
			\verb![!\textit{textN}\verb!]{!\textit{keyN}\verb!}!,
\end{itemize}
%
where, for each given citation, key (\textit{key}) an associated \textit{text} can
also be specified, for example:
%
\begin{itemize}
    \item Similar results can be found in 
    \mcite[Ch.~2]{Loimayr2019}[Sec.~3.6]{Drake1948}[pp.~5--7]{Eberl1987}.
\begin{LaTeXCode}[numbers=none]
Similar results can be found in 
\mcite[Ch.~2]{Loimayr2019}[Sec.~3.6]{Drake1948}[pp.~5--7]{Eberl1987}.
\end{LaTeXCode}
\end{itemize}
%
For better readability, the output---unlike the regular
\texttt{\textbackslash cite}---includes a \emph{semicolon} (;) as a separator
between the entries. However, sorting the entries (if desired) must be done
manually; it is not done by the \texttt{\textbackslash mcite} command.


\subsection{Suppressing Back References in the Bibliography}

With the present setup, a list of the text pages on which the source was cited
is automatically appended to each entry in the bibliography. In rare cases, it
is helpful to suppress these back references for individual citations, by using
%
\begin{itemize}
    \item[] \verb!{\backtrackerfalse\cite{...}}!
\end{itemize}
%
This also works with \verb!\mcite! and other \verb!cite! commands; note that 
the outer brackets are important here. For example, in item
{\backtrackerfalse\parencite{Bezos2023}} of the bibliography 
the current page (\the\value{page}) should \emph{not} be listed.

\subsection{Common Mistakes}

Several common mistakes tend to occur when working with references, especially
for inexperienced authors. However, these can be easily avoided.

\subsubsection{Placing References Outside Sentences}

Citations should be placed \emph{within} or\emph{at the end} of a sentence (\ie, before the
period), not \emph{outside}:
%
\begin{quote}
	\begin{tabular}{rl}
		\textrm{Wrong:}  & \ldots this is the end of the sentence.
		\cite{Oetiker2021} And here it continues \ldots \\
		\textrm{Correct:} & \ldots this is the end of the sentence
		\cite{Oetiker2021}. And here it continues \ldots
	\end{tabular}
\end{quote}

\subsubsection{References Without a Preceding Space}

A citation is \emph{always} separated from the preceding word by a space, never
appended directly to the word (unlike a footnote):
%
\begin{quote}
	\begin{tabular}{rl}
		\textrm{Wrong:}  & \ldots here goes the
		citation\cite{Oetiker2021} and it continues \ldots  \\
		\textrm{Correct:} & \ldots here goes the citation
		\cite{Oetiker2021} and it continues \ldots
	\end{tabular}
\end{quote}

\subsubsection{Literal Quotes}

If an entire paragraph (or more) is quoted from a source, the associated reference
should
be placed in the preceding text and not \emph{within} the quote itself. As an
example, the following passage from \cite{Daniel2018}:
%
\begin{quote}
	\begin{german}
		Typographisches Design ist ein Handwerk, das erlernt werden muss.
		Ungeübte Autoren machen dabei oft gravierende Fehler. Fälschlicherweise
		glauben viele Laien, dass Textdesign vor allem eine Frage der Ästhetik
		ist -- wenn das Schriftstück vom künstlerischen Standpunkt aus "schön"
		aussieht, dann ist es schon gut "designed". Da Schriftstücke jedoch
		gelesen und nicht in einem Museum aufgehängt werden, sind die leichtere
		Lesbarkeit und bessere Verständlichkeit wichtiger als das schöne
		Aussehen.
	\end{german}
\end{quote}
%
For the quote itself, the \texttt{quote} environment should be used; it
demarcates the quote from one's text by employing indentations on both sides and
thus reduces the risk of ambiguities (where is the end of the quote?). The above
example also switches to German (see Sec.\ \ref{sec:language-switching}):%
\footnote{Note the German quotation marks inside the quote, which
are set by the \texttt{smartquotes} option.}
%
\begin{itemize}
    \item[] \verb!\begin{quote}\begin{german}! \emph{quoted text \ldots}
    \verb!\end{german}\end{quote}!
\end{itemize}
%
If desired, the quote can also be wrapped in quotation marks \emph{or}
italicized, but not both!

\subsubsection{Optional Extensions (Using Document Option
\texttt{smartquotes})}

The \texttt{csquotes} package%
\footnote{\url{https://ctan.org/pkg/csquotes}, see also section
\ref{sec:quotation-marks}.}
(automatically loaded by the \texttt{smartquotes} option) defines several
additional environments for quotes, \eg
%
\begin{itemize}
    \item[] \verb!\begin{displayquote}! \ldots \verb!\end{displayquote}!
\end{itemize}
%
(equivalent to \verb!\begin{quote}! \ldots \verb!\end{quote}!) and for foreign
language quotes the environment
%
\begin{itemize}
    \item[] \verb!\begin{foreigndisplayquote}{language}! \ldots
    \verb!\end{foreigndisplayquote}!.
\end{itemize}
%
This allows, for example, a German quote%
\footnote{Currently, only the languages \texttt{german} and \texttt{english}
are defined.}
\emph{without} explicit language switching:
%
\begin{itemize}
    \item[] \verb!\begin{foreigndisplayquote}{german}!\newline
       \verb!   !\emph{quoted text \ldots}\newline
    \verb!\end{foreigndisplayquote}!
\end{itemize}

\subsection{Dealing With Secondary Sources}

In rare cases, it happens that one wants (or needs) to cite a source \textrm{A}
that is not available (and thus cannot be read personally) but which is cited in
\emph{another} available source \textrm{B}. In this case, \textrm{A} is called
the \emph{original} or \emph{primary source}, and \textrm{B} is called the
\emph{secondary source}. In such a scenario the following basic rules should be
observed:
%
\begin{itemize}
    \item Secondary sources should be \emph{avoided} whenever
    possible.
    \item In order to quote a source in the usual form, one must always have
    \emph{personally accessed} (read) it!
    \item Only if one can absolutely \emph{not} get hold of the primary source, a
    reference via a secondary source is permissible. In this case, primary and
    secondary sources should be \emph{cited together}, as shown in the example
    below.
    \item \emph{Important}: Only the available source (\textrm{B}) and not
    the original work is included in the bibliography!
\end{itemize}
%
\paragraph{Example:} Suppose one would like to quote a passage from the famous
book (\textrm{A}) \emph{Dialogo} by Galileo Galilei (which is difficult to obtain),
referenced in a more recent work from 1969 (\textrm{A}). One could accomplish this,
\eg, with the following footnote (all page numbers are fictional).%
\footnote{Galileo Galilei, \emph{Dialogo sopra i due massimi sistemi del
mondo tolemaico e copernicano}, p.~314 (1632). Quoted from 
\cite[p.~59]{Hemleben1969}.}
Only the secondary source \cite{Hemleben1969} is included in the actual bibliography.


\section{Bibliography}

There are several options for creating the bibliography in \latex. The
traditional method is to use \texttt{BibTeX} \cite{Patashnik1988}. Another (more
modern) approach uses \texttt{biber}%
\footnote{\url{http://mirrors.ctan.org/biblio/biber/documentation/biber.pdf}}
and \texttt{biblatex}, as described below.


\subsection{Bibliographic Data in BibTeX and BibLaTex}
\label{sec:bibtex}

BibTeX is a stand-alone program that generates a bibliography suitable for
\latex from a "bibliographic database" (one or more text files with a given
structure). Literature on using BibTeX can be found online, \eg,
\cite{Feder2006, Patashnik1988}.

BibTeX files can, of course, be created manually with a text editor. For many
references, complete BibTeX entries are already available online. However, one
should be careful because these entries are (even when provided by large
institutions and publishers) \emph{often wrong or syntactically incorrect}!
Therefore, one should not adopt them unchecked and especially scrutinize the
final results. Furthermore, there are specialized applications for maintaining
BibTeX bibliographies, such as \emph{JabRef}.%
\footnote{\url{https://www.jabref.org/}}

\subsubsection{Using \texttt{biblatex} and \texttt{biber}}

This document uses \texttt{biblatex} (version 1.4 or higher) in conjunction with
the software \texttt{biber}, which addresses many shortcomings of the
traditional BibTeX workflow and significantly extends its capabilities.%
\footnote{In fact, \texttt{biblatex} is the first radical (and long-needed)
overhaul of the outdated BibTeX workflow and has already replaced the latter in
many documents.}
It adds many new entry types, which are indispensable, especially for
referencing modern multimedia sources. However, this means that the
bibliographic data used in \texttt{biblatex} are no longer fully backward
compatible with BibTeX. It is, therefore, usually necessary to manually revise
existing BibTeX data or data taken from online sources (see
Section~\ref{sec:tips-on-biblatex}).

In our setup, the interface to \texttt{biblatex} is contained in the style
file \nolinkurl{hgbbib.sty}. The typical usage in the main \latex file is as
follows:
%
\begin{LaTeXCode}[numbers=left]
\documentclass[master,english,smartquotes]{hgbthesis}
   ...
\bibliography{references} /+\label{tex:references1}+/
   ...
\begin{document}
   ...
\MakeBibliography /+\label{tex:references2}+/
   ...
\end{document}
\end{LaTeXCode}
%
In the "preamble",
\verb!\bibliography{references}!%
\footnote{The \texttt{{\bs}bibliography} command is actually a relic from BibTeX
and is replaced in \texttt{biblatex} by the \texttt{{\bs}addbibresource}
statement. Both statements are equivalent, but often only
\texttt{{\bs}bibliography} makes the associated \texttt{.bib} file visible in
the file structure of the editor environment.}
(line \ref{tex:references1}) refers to a BibLaTeX file \nolinkurl{references.bib},
which holdss all references.
If multiple BibLaTeX files are used, they can be specified in the same form.

The \verb!\MakeBibliography! statement near the end of the document
(line~\ref{tex:references2}) is responsible for the output of the bibliography,
here with the title "References". Two variants are possible:
%
\begin{description}
    \item[\texttt{{\bs}MakeBibliography}] ~ \newline
    Creates a bibliography divided into several \emph{categories} (see
    Section~\ref{sec:reference-categories}). This variant is used in the present
    document.
%
    \item[\texttt{{\bs}MakeBibliography[nosplit]}] ~ \newline
    Generates a traditional \emph{one-piece} bibliography.
\end{description}

\subsection{Reference Categories}
\label{sec:reference-categories}

The \texttt{hagenberg-thesis} package provides the following categories for split
bibliographies (see Table~\ref{tab:references-and-entry-types}):%.
\footnote{These categories are defined in the file \nolinkurl{hgbbib.sty}. Any
changes and the definition of additional categories are pretty straightforward
if required.}
%
\begin{itemize}
    \item[] \textsf{literature} -- for classic publications that are available
    in print or online;
    \item[] \textsf{avmedia} -- for movies, audio-visual media (on DVD,
    streaming, \etc);
    \item[] \textsf{software} -- for software, APIs, computer games;
    \item[] \textsf{online} -- for artifacts that are \emph{only} available
    online.
\end{itemize}
%
Each reference is automatically assigned to one of these categories based on the
specified BibLaTeX entry type (\texttt{@\emph{type}}) (see
Table~\ref{tab:reference-categories}). Only the most essential entry types are
listed here, but they should cover most cases in practice and are explained
below by examples. All entries that are not explicitly specified are assigned to
the category \textsf{literature}.


%%------------------------------------------------------

\begin{table}[htbp]
\caption{Defined reference categories and recommended BibLaTeX entry types.}
\label{tab:references-and-entry-types}
\centering
\begin{tabular}{@{}llc@{}}
	\toprule
	\emph{Literature} (\textsf{literature}) & Type & Page\\
	\midrule
	Book (textbook, monograph) & \texttt{@book} & \pageref{sec:@book}\\
	Collection (editor and multiple authors) & \texttt{@incollection} & \pageref{sec:@incollection} \\
	Conference proceedings & \texttt{@inproceedings} & \pageref{sec:@inproceedings}\\
	Article in a journal or magazine & \texttt{@article} & \pageref{sec:@article}\\
	Thesis (bachelor, master, diploma), dissertation & \texttt{@thesis} & \pageref{sec:@thesis}\\
	Technical report, lab report & \texttt{@report} & \pageref{sec:@report}\\
	Manual, product description & \texttt{@manual} & \pageref{sec:@manual}\\
	Norm, standard & \texttt{@standard} & \pageref{sec:@standard}\\
	Laws, bills, regulations, \etc & \texttt{@legislation} & \pageref{sec:@legislation}\\
	Compositions, sheet music & \texttt{@book}, \texttt{@incollection} & \pageref{sec:sheet-music}\\
	Pre-publications (\eg conference submissions) & \texttt{@unpublished} & \pageref{sec:@unpublished}\\
	\addlinespace
%
	\midrule
	\emph{Audiovisual media} (\textsf{avmedia}) & & \\
	\midrule
	Audio (CD) & \texttt{@audio} & \pageref{sec:@audio}\\
	Image, photo, graphic & \texttt{@image} & \pageref{sec:@image}\\
	Video (on DVD, Blu-ray disk, online) & \texttt{@video} & \pageref{sec:@video}\\
	Movie (cinema) & \texttt{@movie} & \pageref{sec:@movie}\\
	\addlinespace
%
	\midrule
	\emph{Software} (\textsf{software}) & & \\
	\midrule
	Software product or project & \texttt{@software} & \pageref{sec:@software}\\
	Computer game & \texttt{@software} & \pageref{sec:@software}\\
	\addlinespace
%
	\midrule
	\emph{Online sources} (\textsf{online}) & & \\
	\midrule
	Website, wiki entry, blog, \etc & \texttt{@online} & \pageref{sec:@online-www} \\
	\bottomrule
\end{tabular}
\end{table}

\begin{table}
\caption{Reference categories and associated BibLaTeX entry types. In the case
of a split bibliography, the entries of each category are combined in a
common section. Items marked in gray are synonyms for the respective
types above them.}
\label{tab:reference-categories}
\centering
\definecolor{midgray}{gray}{0.5}
\setlength{\tabcolsep}{4mm}
\begin{tabular}{@{}llll@{}}
	\toprule
	\textsf{literature} & \textsf{avmedia} & \textsf{software} & \textsf{online} \\
	\midrule
	\texttt{@book} & \texttt{@audio} & \texttt{@software} & \texttt{@online} \\
	\texttt{@incollection} & \texttt{\color{midgray}@music} & & \texttt{\color{midgray}@electronic} \\
	\texttt{@inproceedings} & \texttt{@video} & & \texttt{\color{midgray}@www} \\
	\texttt{@article} & \texttt{@movie} & & \\
	\texttt{@thesis} & \texttt{@software} & & \\
	\texttt{@report} & & & \\
	\texttt{@manual} & & & \\
	\texttt{@standard} & & & \\
	\texttt{@legislation} & & & \\
	\texttt{@misc} & & &  \\
	\texttt{@unpublished} &  & & \\
	\ldots & & & \\
	\bottomrule
\end{tabular}
\end{table}

%% ----------------------------------------------------

\subsection{Printed Sources (\textsf{literature})}
\label{sec:category-literature}

This category includes all works published in printed form, for example, in
books, conference proceedings, journal articles, theses, etc. In the following
examples, the BibLaTeX entry in the file \nolinkurl{references.bib} is given,
followed by the corresponding result in the bibliography.

\subsubsection{\texttt{\bfseries @book}}
\label{sec:@book}

A single-volume book (monograph) written in its entirety by one or more authors
and (typically) published by one publisher.
% 
\begin{itemize}
\item[]
\begin{GenericCode}[numbers=none]
@book{BurgerBurge2022,
	author={Burger, Wilhelm and Burge, Mark James},
	title={Digital Image Processing},
	subtitle={An Algorithmic Introduction},
	publisher={Springer},
	location={Cham},
	edition={3},
	date={2022},
	doi={10.1007/978-3-031-05744-1},
	langid={english}
}
\end{GenericCode}
\item[\cite{BurgerBurge2022}] \fullcite{BurgerBurge2022}
\end{itemize}
%
Note: The entry field \texttt{edition} is usually only specified if there
is \emph{more} than one, especially \emph{not for the first
edition} if this is the only one! ISBNs can be safely omitted.


%%------------------------------------------------------

\subsubsection{\texttt{\bfseries @incollection}}
\label{sec:@incollection}

A self-contained and titled contribution by one or more authors to a book or
collection. \texttt{title} is the title of the contribution, \texttt{booktitle}
the title of the collection, and \texttt{editor} the name of the editor.
%
\begin{itemize}
\item[]
\begin{GenericCode}[numbers=none]
@incollection{BurgeBurger1999,
  author={Burge, Mark and Burger, Wilhelm},
  title={Ear Biometrics},
  booktitle={Biometrics},
  booksubtitle={Personal Identification in Networked Society},
  publisher={Kluwer Academic Publishers},
  date={1999},
  location={Boston},
  editor={Jain, Anil K. and Bolle, Ruud and Pankanti, Sharath},
  chapter={13},
  pages={273-285},
  doi={10.1007/0-306-47044-6_13},
  langid={english}
}
\end{GenericCode}
\item[\cite{BurgeBurger1999}] \fullcite{BurgeBurger1999}
\end{itemize}


%%------------------------------------------------------

\subsubsection{\texttt{\bfseries @inproceedings}}
\label{sec:@inproceedings}

Conference paper, individual contribution in conference proceedings. Distinguish
between the fields \texttt{venue} to indicate the place of the conference and
\texttt{location} for the location of the publisher.
%
\begin{itemize}
\item[]
\begin{GenericCode}[numbers=none]
@inproceedings{Burger1987,
  author={Burger, Wilhelm and Bhanu, Bir},
  title={Qualitative Motion Understanding},
  booktitle={Proceedings of the Tenth International Joint Conference on Artificial Intelligence},
  date={1987-08},
  editor={McDermott, John P.},
  eventdate={1987-08-23/1987-08-28},
  venue={Milano},
  publisher={Morgan Kaufmann Publishers},
  location={San Francisco},
  pages={819-821},
  doi={10.1007/978-1-4615-3566-9},
  langid={english}
}
\end{GenericCode}
\item[\cite{Burger1987}] \fullcite{Burger1987}
\end{itemize}

%%------------------------------------------------------

\subsubsection{\texttt{\bfseries @article}}
\label{sec:@article}

Article in a magazine, scientific journal, or daily newspaper. \texttt{volume}
denotes the volume and \texttt{number} the number within that volume. The
journal name (\texttt{journaltitle}) should be abbreviated only in justifiable
cases to avoid misunderstandings.
%
\begin{itemize}
\item[]
\begin{GenericCode}[numbers=none]
@article{Mermin1989,
  author={Mermin, Nathaniel David},
  title={What's Wrong with these Equations?},
  journaltitle={Physics Today},
  volume={42},
  number={10},
  date={1989},
  pages={9-11},
  doi={10.1063/1.2811173},
  langid={english}
}
\end{GenericCode}
\item[\cite{Mermin1989}] \fullcite{Mermin1989}
\end{itemize}
%
Note: Specifying an issue for \emph{multiple} months (for example, in the
case of a combined issue) is not possible in \texttt{biblatex} with the
\texttt{month} field because it may only contain \emph{a single} number.
However, the \texttt{number} field can be used for this purpose, \eg,
\texttt{number=\{6-7\}} in \cite{Vardavoulia2001}.

%%------------------------------------------------------

\subsubsection{\texttt{\bfseries @thesis}}
\label{sec:@thesis}

This entry type can be used for any academic thesis. The exact type is specified via the
\texttt{type} attribute. The values \texttt{phdthesis}, \texttt{mathesis}, and
\texttt{bathesis} indicate doctoral, master's, and bachelor's theses,
respectively, and correctly specify the type of thesis, depending on the
document language and style. Alternatively, individual content can be stored in
this field.

\paragraph{Dissertation (Doctoral Thesis, PhD Thesis):}
%
\begin{itemize}
\item[]
\begin{GenericCode}[numbers=none]
@thesis{Eberl1987,
  author={Eberl, Gerhard},
  title={Automatischer Landeanflug durch Rechnersehen},
  type={phdthesis},
  date={1987-08},
  institution={Universität der Bundeswehr, Fakultät für Raum- und Luftfahrttechnik},
  location={München},
  langid={ngerman}
}
\end{GenericCode}
\item[\cite{Eberl1987}] \fullcite{Eberl1987}
\end{itemize}

\paragraph{Diploma Thesis:} ~ \newline
Equivalent to a dissertation (see above), but with
\texttt{type=\obnh\{Diploma thesis\}}.

%%------------------------------------------------------

\paragraph{Master Thesis:} ~ \newline
Equivalent to a dissertation (see above), but with \texttt{type=\{mathesis\}}.%

%
\begin{itemize}
\item[]
\begin{GenericCode}[numbers=none]
@thesis{Loimayr2019,
  author={Loimayr, Nora},
  title={Utilization of GPU-Based Smoothed Particle Hydrodynamics for Immersive Audiovisal Experiences},
  type={mathesis},
  date={2019-11-26},
  month={11},
  institution={University of Applied Sciences Upper Austria, Interactive Media},
  location={Hagenberg, Austria},
  url={https://theses.fh-hagenberg.at/thesis/Loimayr19},
  langid={english}
}
\end{GenericCode}
\item[\cite{Loimayr2019}] \fullcite{Loimayr2019}
\end{itemize}
%
The content of the \verb!url={..}! field will be automatically typeset as a URL
without any additional markup (using the \verb!\url{..}! command).

%%------------------------------------------------------

\paragraph{Bachelor Thesis:} ~ \newline
Bachelor theses are not usually considered "proper" publications, but it must
still be possible to reference them if necessary, for example:
%
\begin{itemize}
\item[]
\begin{GenericCode}[numbers=none]
@thesis{Bacher2004,
  author={Bacher, Florian},
  title={Interaktionsmöglichkeiten mit Bildschirmen und großflächigen Projektionen},
  type={bathesis},
  date={2004-06},
  institution={University of Applied Sciences Upper Austria, Medientechnik und {-design}},
  location={Hagenberg, Austria},
  langid={ngerman}
}
\end{GenericCode}
\item[\cite{Bacher2004}] \fullcite{Bacher2004}
\end{itemize}

%%------------------------------------------------------

\subsubsection{\texttt{\bfseries @report}}
\label{sec:@report}

These are typically numbered reports (\emph{technical reports} or \emph{research
reports}) from companies, university institutes, or research projects. They are
distinguished by the \texttt{type} attribute, which can take the values
\texttt{techreport} or \texttt{resreport}. Specifying the issuing organizational
unit (company, institute, faculty, \etc) and address is essential. It also makes
sense to provide a corresponding URL, if available.
%
\begin{itemize}
\item[]
\begin{GenericCode}[numbers=none]
@report{Drake1948,
  author={Drake, Hubert M. and McLaughlin, Milton D. and Goodman, Harold R.},
  title={Results obtained during accelerated transonic tests of the {Bell} {XS-1} airplane in flights to a {MACH} number of 0.92},
  type={techreport},
  institution={NASA Dryden Flight Research Center},
  date={1948-01},
  location={Edwards, CA},
  number={NACA-RM-L8A05A},
  url={https://www.nasa.gov/centers/dryden/pdf/87528main_RM-L8A05A.pdf},
  langid={english}
}
\end{GenericCode}
\item[\cite{Drake1948}] \fullcite{Drake1948}
\end{itemize}

%%------------------------------------------------------

\subsubsection{\texttt{\bfseries @manual}}
\label{sec:@manual}

This publication type is suitable for technical or other documentation, such as
product descriptions, manuals, presentations, white papers, \etc The
documentation does not need to exist in printed form.
%
\begin{itemize}
\item[]
\begin{GenericCode}[numbers=none]
@manual{Mittelbach2023,
  author={Mittelbach, Frank and Schöpf, Rainer and Downes, Michael and Jones, David M. and Carlisle, David},
  title={The \texttt{amsmath} package},
  date={2023-05-13},
  version={2.17o},
  url={http://mirrors.ctan.org/macros/latex/required/amsmath/amsmath.pdf},
  langid={english}
}
\end{GenericCode}
\item[\cite{Mittelbach2023}] \fullcite{Mittelbach2023}
\end{itemize}
%
Often no author is named in such documents. Instead, the name of the
\emph{company} or \emph{institution} is specified in the \texttt{author} field,
but within \emph{additional parentheses} \texttt{\{..\}}, so that the
argument is not misinterpreted as \emph{first name} + \emph{last name}.%.
\footnote{Unlike BibTeX, \texttt{biblatex} does not accept the
\texttt{organization} field as a replacement for \texttt{author} in
\texttt{@manual} entries.}
This trick is used in the following example (among others).

%%------------------------------------------------------

\subsubsection{\texttt{\bfseries @standard}}
\label{sec:@standard}

References to \emph{standards} are supported in \texttt{biblatex} through the
type \texttt{@standard}. Here is a typical example:
%
\begin{itemize}
\item[]
\begin{GenericCode}[numbers=none]
@standard{WHATWGHTMLLivingStandard,
  author={{Web Hypertext Application Technology Working Group}},
  shortauthor={WHATWG},
  title={HTML},
  titleaddon={Living Standard},
  date={2023-11-06},
  url={https://html.spec.whatwg.org/multipage/},
  langid={english}
}
\end{GenericCode}
\item[\cite{WHATWGHTMLLivingStandard}] \fullcite{WHATWGHTMLLivingStandard}
\end{itemize}
%

%%------------------------------------------------------

\subsubsection{\texttt{\bfseries @patent}}
\label{sec:@patent}

The special entry type \texttt{@patent} exists for patents, as the following
example shows. \texttt{year} and \texttt{month} refer to the date of the patent
grant; the specification of \texttt{holder} is optional:
%
\begin{itemize}
\item[]
\begin{GenericCode}[numbers=none]
@patent{Pike2008,
  author={Pike, Dion},
  title={Master-slave communications system and method for a network element},
  type={US Patent},
  holder={Alcatel-Lucent SAS},
  number={7,460,482},
  date={2008-12-02},
  url={https://patents.google.com/patent/US7460482},
  langid={english}
}
\end{GenericCode}
\item[\cite{Pike2008}] \fullcite{Pike2008}
\end{itemize}
%

%%------------------------------------------------------

\subsubsection{\texttt{\bfseries @legislation}}
\label{sec:@legislation}

Legal texts can be mapped in \texttt{biblatex} using the type
\texttt{@legislation}. Since this is a non-standard type, the driver for
\texttt{@misc} is used, so it is essential to specify details such as the type
of publication with the \texttt{howpublished} field. The following example shows
the application for a legal text (see also \cite{FhStG1993} and
\cite{EuRichtlinie2000}).
%
\begin{itemize}
\item[]
\begin{GenericCode}[numbers=none]
@legislation{OoeRaumordnungsgesetz1994,
  title={Landesgesetz vom 6. Oktober 1993 über die Raumordnung im Land Oberösterreich},
  titleaddon={Oö. Raumordnungsgesetz 1994 - Oö. ROG 1994},
  howpublished={LGBl.Nr. 114/1993 zuletzt geändert durch LGBl.Nr. 111/2022},
  date={1993-12-23},
  url={https://www.ris.bka.gv.at/GeltendeFassung.wxe?Abfrage=LrOO&Gesetzesnummer=10000370},
  langid={ngerman}
}
\end{GenericCode}
\item[\cite{OoeRaumordnungsgesetz1994}] \fullcite{OoeRaumordnungsgesetz1994}
\end{itemize}
%

%%------------------------------------------------------

\subsubsection{\texttt{\bfseries @misc}}
\label{sec:@misc}

If the entry types for printed publications listed so far do not suffice, one
should first look at the other types (not described in detail here) in the
\texttt{biblatex} manual \cite{Kime2023}. For example, \texttt{@collection} for
a complete collection (\ie, not just a single contribution).

If none of them fit, then the \texttt{@misc} type can be used. It provides a
\texttt{howpublished} text field where the type of publication can be specified
individually. Likewise, the \texttt{type} field can be used to specify what kind
of document is represented.

%%------------------------------------------------------

\subsubsection{Compositions and Sheet Music}
\label{sec:sheet-music}

For printed compositions (sheet music or musical scores), there is,
unfortunately, no particular entry type in BibLaTeX. For a \emph{specific}
edition, it is easiest to use the type \texttt{@book}, such as (see also
\cite{HaydnCelloConcerto2,ShostakovichOp110})
%
\begin{itemize}
\item[]
\begin{GenericCode}[numbers=none]
@book{BachBWV988,
  author={Bach, Johann Sebastian},
  title={Goldberg-Variationen für Streichquartett, BWV 988},
  editor={Anka, Dana},
  publisher={Musikverlag Hans Sikorski},
  location={Hamburg},
  date={2017},
  langid={ngerman}
}
\end{GenericCode}
\item[\cite{BachBWV988}] \fullcite{BachBWV988}
\end{itemize}
%
For compositions that have been published in a collection, one can use---as for
a scientific collection---the type \texttt{@incollection}:
%
\begin{itemize}
\item[]
\begin{GenericCode}[numbers=none]
@incollection{GershwinSummertime,
  author={Gershwin, George and Heyward, DuBose},
  title={Summertime},
  booktitle={The Greatest Songs of George Gershwin},
  publisher={Chappel Music},
  location={London},
  pages={40-43},
  date={1979},
  langid={english}
}
\end{GenericCode}
\item[\cite{GershwinSummertime}] \fullcite{GershwinSummertime}
\end{itemize}

\subsubsection{\texttt{\bfseries @unpublished}}
\label{sec:@unpublished}

It is becoming increasingly common for manuscripts to be published online by
authors a long time before actual publication, for example, on platforms such as
\textsf{arXiv.org}%
\footnote{\url{https://arxiv.org/}}
or \textsf{researchgate.net}.%
\footnote{\url{https://www.researchgate.net/}}
It should be noted that releasing something online does \emph{not formally
count as a publication}, as these platforms are no publishers. Some of the
works (uploaded by the authors) are \emph{never} officially published, \eg,
unaccepted submissions to conferences. When citing this reference it is crucial
to determine whether the paper was accepted and actually published:
%
\begin{itemize}
\item[a)]
\emph{The work was indeed published:} Here the corresponding original
publication must be searched and used! The citation is done in conventional form
(for example, with \texttt{@inproceedings} in case of a conference volume), but
\emph{without} a reference to the online publication.
\item[b)]
\emph{The work was \emph{not} published:} If indeed no publication (or an
associated \emph{technical report}, see above) can be found, the
\texttt{@unpublished} tag can be helpful:
\item %
\begin{GenericCode}[numbers=none]
@unpublished{Dai2016,
  author={Dai, Jifeng and Li,Yi and He, Kaiming and Sun, Jian},
  title={{R-FCN:} Object Detection via Region-Based Fully Convolutional Networks},
  date={2016},
  pubstate={prepublished},
  doi={10.48550/arXiv.1605.06409},
  langid={english}
}
\end{GenericCode}
\begin{itemize}
\item[\cite{Dai2016}] \fullcite{Dai2016}
\end{itemize}
%
\end{itemize}
%
In the latter case, the specification of the link (via \texttt{doi} or alternatively
\texttt{url}) is essential. Details on the \texttt{pubstate} field
(\texttt{prepublished}) can be found in the \texttt{biblatex} documentation
\cite[Sec.\ 4.9.2.11]{Kime2023}. Otherwise (if unknown or not published),
%
\begin{itemize}
\item[]\texttt{note=\{unpublished\}} \quad or
   \quad \texttt{note=\{unververöffentlicht\}}
\end{itemize}
%
can be used instead of \texttt{pubstate}.


\subsection{Movies and Audio-Visual Media (\textsf{avmedia})}
\label{sec:category-avmedia}

This category covers audio-visual productions such as movies, sound recordings,
CDs, DVDs, and VHS tapes, thus meaning works published in physical (but not
printed) form. It does not include audio-visual creations (sound recordings,
images, videos) that are exclusively available online---these should be
referenced with the entry type \texttt{@online} (see
Table~\ref{tab:reference-categories} and Section~\ref{sec:category-online}).

The \texttt{@audio}, \texttt{@video}, and \texttt{@movie} types described below
are \emph{not} BibTeX standard types. However, they are represented in
\texttt{biblatex} (implicitly replaced by \texttt{@misc}) and are recommended
here to allow the automatic assignment of items in the bibliography.

\subsubsection{\texttt{\bfseries @audio}}
\label{sec:@audio}
Here is an example of the specification of an audio CD:
%
\begin{itemize}
\item[]
\begin{GenericCode}[numbers=none]
@audio{Zappa1995,
  author={Zappa, Frank},
  title={Freak Out!},
  type={audiocd},
  date={1995-05},
  organization={Rykodisc, New York},
  langid={english}
}
\end{GenericCode}
\item[\cite{Zappa1995}] \fullcite{Zappa1995}
\end{itemize}
%
Instead of \verb!type={audiocd}! one could also use \verb!howpublished={Audio CD}!.

\subsubsection{\texttt{\bfseries @image}}
\label{sec:@image}

The following example shows the reference to a digitally available photo, which
is also used in Figure~\ref{fig:CocaCola}:
%
\begin{itemize}
\item[]
\begin{GenericCode}[numbers=none]
@image{CocaCola1940,
  author={Wolcott, Marion Post},
  title={Natchez, Miss.},
  note={Library of Congress Prints and Photographs Division Washington, Farm Security Administration/Office of War Information Color Photographs},
  date={1940-08},
  url={https://www.loc.gov/pictures/item/2017877479/},
  langid={english}
}
\end{GenericCode}
\item[\cite{CocaCola1940}] \fullcite{CocaCola1940}
\end{itemize}

\subsubsection{\texttt{\bfseries @video}}
\label{sec:@video}

The example below shows a link to a YouTube video:
%
\begin{itemize}
\item[]
\begin{GenericCode}[numbers=none]
@video{HistoryOfComputers2008,
  title={History of Computers},
  date={2008-09-24},
  url={https://www.youtube.com/watch?v=LvKxJ3bQRKE},
  langid={english}
}
\end{GenericCode}
\item[\cite{HistoryOfComputers2008}] \fullcite{HistoryOfComputers2008}
\end{itemize}

\noindent
Here is an example of referencing a DVD edition:
%
\begin{itemize}
\item[]
\begin{GenericCode}[numbers=none]
@video{Futurama1999,
  author={Groening, Matt},
  title={Futurama},
  titleaddon={Season 1 Collection},
  howpublished={DVD},
  date={2002-02},
  organization={Twentieth Century Fox Home Entertainment},
  langid={english}
}
\end{GenericCode}
\item[\cite{Futurama1999}] \fullcite{Futurama1999}
\end{itemize}
%
In this case, the specified date is the \emph{date of issue}. If no unique
author can be named, omit the \texttt{author} field and wrap the corresponding
information in the \texttt{note} field, as shown in the example below.

\subsubsection{\texttt{\bfseries @movie}}
\label{sec:@movie}

This entry type is intended for movies. No author is specified because it is
usually not possible to name a specific author for a film production. In the
following example (see also \cite{Psycho1960}) the relevant data are provided
in the \texttt{note} field:%
\footnote{By the way, \texttt{biblatex} nicely ensures that the dot at the end
of the \texttt{note} text is not duplicated in the output.}
%
\begin{itemize}
\item[]
\begin{GenericCode}[numbers=none]
@movie{Nosferatu1922,
  title={Nosferatu -- A Symphony of Horrors},
  howpublished={Film},
  date={1922},
  note={Drehbuch/Regie: F.\ W.\ Murnau. Mit Max Schreck, Gustav von Wangenheim, Greta Schröder.},
  langid={english}
}
\end{GenericCode}
\item[\cite{Nosferatu1922}] \fullcite{Nosferatu1922}
\end{itemize}
%
The specification \verb!howpublished={Film}! makes sense here to avoid confusion
with a possible book of the same name.

\subsubsection{Time Codes for Music Recordings and Movies}

A reference to a specific passage in a music or movie can be accomplished
similarly to the page reference in a printed work. Particularly legendary
(and often parodied), for example, is the shower scene in \emph{Psycho}
\cite[T=00:32:10]{Psycho1960}:
%
\begin{itemize}
    \item[] \verb!\cite[T=00:32:10]{Psycho1960}!
\end{itemize}
%
As an alternative to the simple timecode "T=\emph{hh}:\emph{mm}:\emph{ss}",
it is possible to specify the position of a specific frame by using the
corresponding \emph{timecode} "TC=\emph{hh:mm:ss:ff}", \eg,
\cite[TC=00:32:10:12]{Psycho1960} for frame \emph{ff}=12:
%
\begin{itemize}
    \item[] \verb!\cite[TC=00:32:10:12]{Psycho1960}!
\end{itemize}

\subsection{Software (\texttt{\bfseries @software})}
\label{sec:@software}

This entry type is especially suitable for computer games (in the absence of a
particular entry type).
%
\begin{itemize}
\item[]
\begin{GenericCode}[numbers=none]
@software{LegendOfZelda1998,
  author={Miyamoto, Shigeru and Aonuma, Eiji and Koizumi, Yoshiaki},
  title={The Legend of Zelda: Ocarina of Time},
  howpublished={N64 Cartridge},
  publisher={Nintendo},
  date={1998-11},
  langid={english}
}
\end{GenericCode}
\item[\cite{LegendOfZelda1998}] \fullcite{LegendOfZelda1998}
\end{itemize}

\noindent
The following is an example of the reference to a typical software project:
%
\begin{itemize}
\item[]
\begin{GenericCode}[numbers=none]
@software{SpringFramework,
  title={Spring Framework},
  url={https://github.com/spring-projects/spring-framework},
  langid={english}
}
\end{GenericCode}
\item[\cite{SpringFramework}] \fullcite{SpringFramework}
\end{itemize}



\subsection{Online Sources (\texttt{\bfseries @online})}
\label{sec:category-online}

In the case of references to online resources, three cases need to be
distinguished:
%
\begin{itemize}
    \item[A.] Referring to a web page in general, such as the "Panasonic
    products for business" page.%
    \footnote{\url{http://business.panasonic.co.uk/}}
    In this case, there is no reference to a specific "work"; therefore, it is
    \emph{not included} in the bibliography. Instead, as shown in the previous
    sentence, it is sufficient to use a simple footnote with
    \verb!\footnote{\url{..}}!.
%
    \item[B.] Some piece of printed or audio-visual work (see Secs.~%
    \ref{sec:category-literature} and \ref{sec:category-avmedia}) is
    \emph{additionally} available online. In this case, however, the primary
    publication is \emph{not} "online", and it is sufficient to specify the
    associated link in the \texttt{url} field, which can be specified for any
    entry type.
%
    \item[C.] A publication in the broadest sense that is
    \emph{exclusively} available online, such as a wiki or blog entry. The \emph{online} category
    is precisely (and \emph{only}) intended for this type of source.
\end{itemize}

\subsubsection{Example: Wiki Entry}
\label{sec:@online-www}

Due to the volume and increasing quality of these entries, their inclusion in
a bibliography seems justified. For example, a \emph{relic shrine} (German:
"Reliquienschrein") is a casket in which the relics of one or more saints are
kept \cite{WikiReliquienschrein2023}.
%
\begin{itemize}
\item[]
\begin{GenericCode}[numbers=none]
@online{WikiReliquienschrein2023,
  title={Reliquienschrein},
  url={https://de.wikipedia.org/wiki/Reliquienschrein},
  date={2023-09-22},
  urldate={2023-11-06},
  langid={ngerman}
}
\end{GenericCode}
\item[\cite{WikiReliquienschrein2023}] \fullcite{WikiReliquienschrein2023}
\end{itemize}
%
In this case, the reference consists mainly of a URL. With \texttt{date}, one
can specify the current version at that time. The (optional) specification of
\texttt{urldate} (in \texttt{YYYY-MM-DD} format) automatically inserts the
information when the online document was accessed.

Technically, only the \texttt{url} field is required for online
sources, but specifying further details (\eg, \texttt{author}) is possible.
However, if \emph{no} author is available, then---as shown in the examples
above---at least a meaningful \emph{title} (\texttt{title}) should be added,
which may be used for sorting the bibliography.

\subsection{Tips for Creating BibLaTex Files}
\label{sec:tips-on-biblatex}

The following items should be considered for creating correct BibLaTeX
files.

\subsubsection{\texttt{date} Attribute}

While in classic BibTeX, the year and month of a publication are specified using
the attributes \texttt{year} and \texttt{month}, for pure BibLaTeX
bibliographies (as in this document), the attribute \texttt{date} is better
suited. Entries are made in the format \texttt{YYYY-MM-DD}, which can also
consist of only the year (\texttt{YYYY}) or year and month (\texttt{YYYY-MM}).
Likewise, durations can also be defined in the format
\texttt{YYYY-MM-DD/YYYY-MM-DD}. Related fields are \texttt{origdate} (original
publication date for, \eg, a reprint or translation), \texttt{eventdate} (date
of a conference), and \texttt{urldate} (access date of an URL).

Should \texttt{year} and \texttt{month} nevertheless be used, note that the
latter is numeric in BibLaTeX (in contrast to BibTeX) and is specified,
for example, in the form \verb!month={8}! (for August).

\subsubsection{\texttt{langid} Attribute}

The \texttt{langid} attribute enables the correct typesetting of multilingual
bibliographies. If possible, it should be specified for every reference, \eg,
%
\begin{quote}
\verb!langid={english}! \quad or \quad \verb!langid={ngerman}!
\end{quote}
%
for an English- or German-language source, respectively.

\subsubsection{\texttt{edition} Attribute}

The numeric \texttt{edition} field is used to specify the edition of a
reference. Only the number itself has to be specified, \eg, \verb!edition={3}!
for a third edition. The complete text is added automatically depending on the
language setting (\eg, "3rd edition" or "3.\ Auflage"). As already described on
page \pageref{sec:@book} (under \texttt{@book}), in case of a
\emph{first} edition, the \texttt{edition} field should
\emph{not} be specified if there is no other edition!

\subsubsection{Caution When Incorporating Existing Bibtex Entries}

Many publishers and literature brokers offer ready-made BibTeX entries for
download. However, great care should be taken because these entries are often
incomplete, inconsistent, or syntactically incorrect! They should \emph{always}
be checked for correctness when importing them! Particular attention should be
paid to the correct specification of the first names (\texttt{\textit{fn}}) and
last names (\texttt{\textit{ln}}), preferably in the form%
\footnote{Note how the commas are placed! The keyword \underline{\texttt{and}}
separates the names of the individual authors.}
%
\begin{itemize}
\item[]
\texttt{author=\{\textit{ln1}, \textit{fn1a} \emph{fn1b} \underline{and}
\textit{ln2}, \textit{fn2a} \ldots \}}.
\end{itemize}
%
This is especially important for multi-part surnames because otherwise, the
first and last names cannot be correctly assigned, \eg,
%
\begin{itemize}
\item[]
\texttt{author=\{van Beethoven, Ludwig \underline{and} ter Linden, Jaap\}}
\end{itemize}
%
for a (fictional) piece of joint work by \emph{Ludwig van Beethoven} and \emph{Jaap ter
Linden}.

Omissions or errors are often found in entries of \texttt{volume},
\texttt{number}, and \texttt{pages}, especially in collections
(\texttt{@incollection}) and conference proceedings (\texttt{@inproceedings}).
Also, the names of conferences and meeting venues are often not specified
correctly (even in official ACM and IEEE sources). ISBN and ISSN are usually
redundant and should be omitted. However, a DOI (Digital Object Identifier)
entry is useful. This unique number is assigned a hyperlink by BibLaTeX that
points to the source of the work (usually that of the publisher). To avoid
duplicate hyperlinks in the entry, the \texttt{hagenberg-thesis} package
automatically removes the \texttt{url} field if a DOI is present.

Since imported entries are almost always in BibTeX and not BibLaTeX notation,
they should be adjusted to correct types if necessary and to make use of the
current detail fields.

\subsubsection{Common Citation Errors}

Check the finished bibliography carefully for \emph{completeness} and
\emph{consistency}. For each reference, is it clear how and where it was
published? Are the details sufficient to locate the source? Here is a list of
the most critical provisions related to the bibliography:
%
\begin{itemize}
    \item Check all entries for missing or misinterpreted items!
    \item Check all authors' names and first names, are the abbreviations (of
    first names) consistent?
    \item Check the capitalization and punctuation of all entries and correct
    them if necessary.
    \item Books: Check the publisher's name and the place of publication for 
    completeness, consistency, and redundancies.
    \item \emph{Omit} URLs unless they are indispensable! This is especially
    true for books and conference papers. Instead, provide a DOI if
    available---it will also be linked to the source in the PDF.
    \item Journal articles: Always specify the full name of the journal, \eg,
    "ACM Transactions on Computer-Human Interaction" instead of "ACM Trans.\
    Comput.-Hum.\ Interact."! Do not forget the page numbers!
    \item Conference proceedings: Refer to conference proceedings consistently
    in the form "Proceedings of the \emph{XY Conference on Something} \ldots".
    Indicate the conference venue, and remember to specify the page numbers!
    \item For technical reports, master theses, and dissertations, the
    institution (university and department, company, \etc) \emph{must} be
    provided!
\end{itemize}

\subsubsection{Listing of All References}

Using the \verb!\nocite{*}! statement---placed anywhere in the document---will
list \emph{all} existing entries of the BibLaTeX file in the bibliography,
including those for which there is no explicit \verb!\cite{}! call in the text.
This is useful for getting an overview of all references while writing the thesis.
Normally, however, all cited sources must also be referenced in the text!


\section{Using the APA Citation Style}

As an alternative to the numeric citation style set in this document
(\texttt{numeric-comp}), the style of the American Psychological Association%
\footnote{\url{https://apastyle.apa.org/style-grammar-guidelines/references/}}
(APA) may be used. This should be in accordance with the institute's guidelines
and agreed upon with the supervisor. This type of reference formatting uses the
author's name and the year of publication instead of a number in square
brackets. The formatting of the entries in the bibliography is also different.

To use APA as the citation style throughout the document, the document option
"\texttt{apa}" must be specified in the main file, \eg,
%
\begin{LaTeXCode}[numbers=none]
\documentclass[master,english,smartquotes,apa]{hgbthesis}
\end{LaTeXCode}
%
Depending on the type of use, several \emph{different commands} are necessary
to reference sources in the text, as described below.

\subsection{Narrative Citations}

For narrative citations, the reference is used as the subject or object of the
sentence. The year is placed after the author's name in parentheses. To create
this kind of citation use 
%
\begin{itemize}
    \item[] \verb!\textcite{!\textit{keys}\verb!}!.
\end{itemize}
%
Example:
%
\begin{itemize}
\item[]
\begin{LaTeXCode}[numbers=none,breakindent=0pt]
\textcite{Daniel2018} give a brief introduction to \latex, whereas
\textcite{Oetiker2021, Kopka2003} go into more detail.
\end{LaTeXCode}
\item[]
    Daniel et al. (2018) give a brief introduction to LaTeX, whereas Kopka and
    Daly (2003) and Oetiker et al. (2021) go into more detail.
\end{itemize}


\subsection{Narrative Citations Within Parentheses}

If a reference is to be used \emph{within} parentheses, these must be omitted
from the citation. The associated command is
%
\begin{itemize}
    \item[] \verb!\nptextcite{!\textit{keys}\verb!}!.
\end{itemize}
%
Example:
%
\begin{itemize}
\item[]
\begin{LaTeXCode}[numbers=none,breakindent=0pt]
In any case, it is recommended to obtain literature on the topic of \latex
(\eg, \nptextcite{Daniel2018, Oetiker2021, Kopka2003}).
\end{LaTeXCode}
\item[]
    In any case, it is recommended to obtain literature on the topic of \latex
    (\eg, Daniel et al., 2018; Kopka \& Daly, 2003; Oetiker et al., 2021).
\end{itemize}

\subsection{Parenthetical Citations}

Parenthetical citations are used when the reference should be cited at the end
of a sentence or statement. The author's name and year are enclosed in
parentheses and separated by a comma. In this case, use
%
\begin{itemize}
    \item[] \verb!\parencite{!\textit{keys}\verb!}!.
\end{itemize}
%
Example:
%
\begin{itemize}
\item[]
\begin{LaTeXCode}[numbers=none,breakindent=0pt]
For \latex, there are short introductions \parencite{Daniel2018}, as well as
more comprehensive works \parencite{Oetiker2021, Kopka2003}.
\end{LaTeXCode}
%
\item[]
    For \latex, there are short introductions (Daniel et al., 2018), as well as
    more comprehensive works (Kopka \& Daly, 2003; Oetiker et al., 2021).
\end{itemize}


\subsubsection{Appearance in the Bibliography}

\begin{sloppypar}
The references used above are displayed in the bibliography as follows:
%
\begin{list}{}{
\setlength{\leftmargin}{1.3cm}
\setlength{\itemindent}{-1.3cm}
\setlength{\itemsep}{-0.15cm}
\urlstyle{rm}
}
\item
Daniel, M., Gundlach, P., Schmidt, W., Knappen, J., Partl, H. \& Hyna, I.
(2018, April 8). \textit{\LaTeX2e-Kurzbeschreibung.} Version 3.0c.
\textrm{\url{http://mirrors.ctan.org/info/lshort/german/l2kurz.pdf}}.
(cit. on pp.\ 63, 64)
\item
Kopka, H. \& Daly, P. W. (2003). \textit{Guide to \LaTeX} (4th ed.).
Addison-Wesley. (Cit. on pp.\ 63, 64).
\item
Oetiker, T., Partl, H., Hyna, I. \& Schlegl, E. (2021, March 9). \textit{The
Not So ShortIntroduction to \LaTeX2e: Or \LaTeX2e in 139 minutes.} Version 6.4.
\url{http://mirrors.ctan.org/info/lshort/english/lshort.pdf} (cit. on pp.\
63, 64).
\end{list}
\end{sloppypar}


\section{Plagiarism and Paraphrasing}
\label{sec:plagiarism}

\emph{Plagiarism} means representing somebody else's work as one's own creation, in part
or whole, whether consciously or unconsciously. Plagiarism is not a new problem
in higher education but has increased dramatically in recent years due
to the widespread availability of electronic sources. Many universities now also use electronic
resources (some of which are not accessible to students) as a countermeasure.
Therefore, one should expect any submitted work to be routinely checked for
plagiarism!

Plagiarism is by no means considered a trivial offense. Even if discovered at some later
point, this can, in the worst case, lead to the subsequent (and final) revocation of the
academic degree. To avoid such problems, one should better be overcautious and at
least observe the following rules:
%
\begin{itemize}
    \item
    Adopting short text passages is only permitted with correct citation,
    whereby the text quotation's scope (beginning and end) must be clear in each
    case.
    \item
    In particular, it is not acceptable to mention a reference only at the
    beginning and then repeatedly adopt non-cited text passages as
    one's own creation.
    \item
    Under no circumstances tolerable is to directly adopt or \emph{paraphrase}
    longer text passages, whether with or without citation. Passages used
    indirectly or translated from another language must also be cited properly!
\end{itemize}
%
In case of doubt, more detailed rules can be found in any good book on
scientific writing. To be safe, asking your thesis advisor is never a mistake.
