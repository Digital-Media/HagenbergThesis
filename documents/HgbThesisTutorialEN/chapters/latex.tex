\chapter{Working with \latex}
\label{cha:WorkingWithLatex}


\section{Getting Started}
\label{sec:LatexGettingStarted}

\latex is a widespread and classic document preparation system for creating
large and complicated documents with professional requirements. Working with
\latex appears---at least for inexperienced users---at first, more complex
than with conventional tools for word processing.

First, unlike most common word processors, \latex is not \textsc{WYSIWYG}.%
\footnote{"What You See Is What You Get." There were some \textsc{WYSIWYG}
editors for \latex, but they all have disappeared in the last years.}
However, it is a \emph{markup} language (like HTML) that can be somewhat
complicated for beginners together with a set of associated tools. The
supposedly strong restrictions of \latex, especially concerning the choice of
fonts and layout, certainly also seem unfamiliar. While at first, the impression
arises that this rigidity limits one's creativity, it is noticeable after a
while that it is precisely through this that one concentrates more on the
content of the work than on its outer form. The fact that the form is still
correct in the end, however, is only guaranteed if one imposes extreme
restraint on oneself when it comes to modifying the formats and parameters;
unless, of course, one has already become a \latex \emph{expert} in the
meantime.

In the end, the effort is worth it, especially since the thesis is a substantial
effort (with or without \latex). However, with the help of \latex, a
professional-looking result should be easier to achieve, and it should also save
some trouble with errors and limitations of standard software. In addition, one's
(semi-){\obnh}professional eye for the subtleties of book typesetting might
further develop along the way.%
\footnote{By the way, this final text element was set like this to enable a line
break after the parenthesis: \texttt{\ldots (semi-)\{{\bs}obnh\}professional\ldots}
The non-standard command \texttt{{\bs}obnh} ("optional break with no hyphen") is
defined in \texttt{hgbabbrev.sty}.}

\subsection{Software}
\label{sec:Software}

To work with \latex, one needs---besides a computer---the necessary software. In
the past, the individual components of \latex often had to be painstakingly
gathered and configured for one's environment. Nowadays, ready-made
distributions are available for the most important platforms (Windows, macOS,
Linux) that contain everything needed. The current version of \latex\ is
\LaTeXe\ (pronounced "LaTeX two e"). Working locally with \latex requires two
things:
%
\begin{itemize}
    \item a \latex installation (distribution),
    \item a text editor or authoring environment (front end).
\end{itemize}
%
All components are free of charge and available for all common platforms.

Alternatively, an \emph{online} editor can be used, which allows working in the browser
and does not require any installation on one's computer. In addition, the work
can easily be shared with other people, such as the supervisor. Details about
recommended setups and possible alternatives can be found in
Appendix~\ref{app:TechnicalDetails}.

\subsection{Literature}
\label{sec:literature}

It is tedious to start with \latex without relevant literature; even advanced
users will often require help. Fortunately, many helpful resources are available
online. Good starting points are, \eg,
%
\begin{itemize}
    \item \citetitle{Oetiker2021} by \textcite{Oetiker2021} or
	\item \citetitle{Daniel2018} by \textcite{Daniel2018} (only available in German).

\end{itemize}
%
\noindent
As a well-known and often referenced manual to \latex 
%
\begin{itemize}
    \item \citetitle{Kopka2003} by \textcite{Kopka2003}
\end{itemize}
%
can be recommended. Numerous other documents on \latex and related topics can be
found online at the \emph{Comprehensive TeX Archive Network} (CTAN) at
%
\begin{quote}
    \url{https://ctan.org/}.
\end{quote}
%
Particularly useful are \citetitle{Pakin2024} \cite{Pakin2024} and the
descriptions of important \latex packages, such as
%
\begin{itemize}
    \item[] \texttt{babel} \cite{Bezos2025},
    \item[] \texttt{graphics}, \texttt{graphicx} \cite{Carlisle2024},
    \item[] \texttt{fancyhdr} \cite{Oostrum2025},
    \item[] \texttt{caption} \cite{Sommerfeldt2023}.
\end{itemize}


\section{Typesetting}

When working with a \latex document, one of the first things is to specify the
font used. Text passages can then be accentuated by changing the font style
using different kinds of markup.

\subsection{Fonts}

\latex normally uses the fonts of the \emph{Computer Modern} (CM) series, which,
like the \emph{TeX} software itself, were developed by Donald Knuth.%
\footnote{\url{https://www-cs-faculty.stanford.edu/~knuth/}}
The three basic CM series fonts in \latex are
%
\begin{quote}
    \begin{tabular}{lcl}
        \textrm{Roman}      & & \verb!\textrm{Roman}!,      \\
        \textsf{Sans Serif} & & \verb!\textsf{Sans Serif}!, \\
        \texttt{Typewriter} & & \verb!\texttt{Typewriter}!.
    \end{tabular}
\end{quote}
%
In the eyes of many users, the quality and timelessness of these fonts alone is
a reason to use \latex for professional purposes. Another advantage of
\emph{TeX} fonts is that the different font families and weights are very
well-matched in size.

In addition, any \emph{PostScript} font (Type 1) can be used in \latex, but this
requires some finetuning in practice. Frequently used are, \eg, \emph{Times}
and \emph{Palatino}, but there is a trend back to using the classic CM fonts.

\subsection{Text Effects}

Text can be formatted in different ways.
%
\begin{itemize}
    \item \textit{Italicization} (\verb!\textit{..}!) is especially suitable for
    emphasizing and quotations, but also for product names, foreign words, and
    mathematical variables in the text, for example,
    \begin{quote}
        \verb!\textit{Variable}! $\rightarrow$ \textit{Variable}
    \end{quote}
%
    \item \textsl{Slanted} (\verb!\textsl{..}!) denotes a slanted typeface and
    thus differs significantly from \textit{italic}; for comparison:
    \begin{quote}
        \verb!\textrm{Daimler-Chrysler}! $\rightarrow$
        \textrm{Daimler-Chrysler} \newline%
        \verb!\textsl{Daimler-Chrysler}! $\rightarrow$
        \textsl{Daimler-Chrysler} \newline%
        \verb!\textit{Daimler-Chrysler}! $\rightarrow$ \textit{Daimler-Chrysler}
    \end{quote}
%
    \item \textbf{Boldface} (\verb!\textbf{..}!) is used for \textbf{headings},
    labels of \textbf{figures} and \textbf{tables}, but only in rare cases in
    continuous text:
    \begin{quote}
        \verb!\textbf{Headings}! $\rightarrow$ \textbf{Headings}
    \end{quote}
%
    \item \emph{Emphasize} (\verb!\emph{..}!) is usually equivalent to
    \verb!\textit!, but \verb!\emph! also does the "right thing" for nested
    emphases and in combination with other font styles:
    \begin{quote}
        \setlength{\tabcolsep}{0pt}%
        \begin{tabular}{lcl}
            \verb!\textrm{You're \emph{also} here?}! & $\;\rightarrow\;$ &
            \textrm{You're \emph{also} here?} \\
            \verb!\textit{You're \emph{also} here?}! & $\;\rightarrow\;$ &
            \textit{You're \emph{also} here?} \\
            \verb!\textsl{You're \emph{also} here?}! & $\;\rightarrow\;$ &
            \textsl{You're \emph{also} here?} \\
            \verb!\textbf{You're \emph{also} here?}! & $\;\rightarrow\;$ &
            \textbf{You're \emph{also} here?} \\
            \verb!\texttt{You're \emph{also} here?}! & $\;\rightarrow\;$ &
            \texttt{You're \emph{also} here?}
        \end{tabular}
    \end{quote}
%
    \item \underline{Underlining} is a relic of the typewriter era and is
    \underline{dispensable} in modern typesetting. It should therefore be used
    only in exceptional cases, \eg
    \begin{quote}
        \verb!\underline{dispensable}!%
        \footnote{Also, underlined texts are not automatically
        hyphenated.}
    \end{quote}
%
\end{itemize}


\section{Text Structure}

\latex provides several commands for structuring the text.

\subsection{Paragraph Breaks}

Paragraphs are separated in {\latex} source text exclusively by inserting one or
more \emph{blank lines} from each other, so \emph{no other commands} are
necessary!
%
\begin{center}
	\setlength{\fboxrule}{0.2mm}
	\setlength{\fboxsep}{2mm}
	\fbox{%
		\begin{minipage}{0.9\textwidth}
			Especially the use of \texttt{\textbackslash\textbackslash} and
			\texttt{\textbackslash{newline}} commands for line breaks is a
			common \emph{error}. Also, the statement
			\texttt{\textbackslash{paragraph}\{\}} must \emph{not} be used
			in this context; it is---unlike in HTML---a command to define
			sub-headings with titles in \latex\ (see below).
	\end{minipage}}
\end{center}

Usually, \latex inserts \emph{no} additional vertical spacing between
consecutive paragraphs.%
\footnote{This is the default setting in \latex. It depends on parameters
such as the document class and style.}
However, the \emph{first} line of each paragraph (except in
the first paragraph of a section) is indented to define the paragraph
boundaries. This scheme has proven successful in traditional book typesetting%
\footnote{Those who do not believe it should search their bookshelf (or their
parents' bookshelf if necessary) for counterexamples.}
and should be retained unless there are very good reasons against it. Headings
(see below) are provided for all other outlines in the vertical text flow.

\subsection{Headings}
\label{sec:headings}

\latex provides---depending on the document class used---a set of predefined
heading formats in the following order:
%
\begin{quote}
    \verb!\part{!\texttt{\em Title}\verb!}!%
    \footnote{\texttt{part} is intended for splitting a large document
    into several parts and is typically not used in a thesis (and not in this
    document).}
    \newline%
    \verb!\chapter{!\texttt{\em Title}\verb!}! \newline%
    \verb!\section{!\texttt{\em Title}\verb!}! \newline%
    \verb!\subsection{!\texttt{\em Title}\verb!}! \newline%
    \verb!\subsubsection{!\texttt{\em Title}\verb!}! \newline%
    \verb!\paragraph{!\texttt{\em Title}\verb!}! \newline%
    \verb!\subparagraph{!\texttt{\em Title}\verb!}!
\end{quote}
%
\paragraph{Frequent error:} When using \verb!\paragraph{}! and
\verb!\subparagraph{}!---as seen in this paragraph---the text following the
title continues on the same line without a line break; care should be taken to
use appropriate punctuation in the title (here, \eg, \underline{\texttt{:}}).
The horizontal spacing after the title alone would not make it recognizable as a
heading.

\subsubsection{Title Capitalization}

While the English language usually only uses capital letters at the beginning of
a sentence and for proper nouns, these rules are different for titles and
headings. Varying requirements apply depending on the style guide (\eg, APA,
MLA, or The Chicago Manual of Style). While one is most welcome to delve into
these style guides, this might be too much detail when writing a thesis.
Therefore, choose one of the styles mentioned above and use a
title capitalization tool%
\footnote{\url{https://capitalizemytitle.com/} formats titles according to
several style guides, and it is easy to use.}
to get the correct output. This template uses the APA style for headings and
titles, by the way.

\subsection{Lists}

Lists are a popular means of structuring text. In \latex---similar to HTML---
three types of formatted lists are available: unordered lists ("bullet lists"),
ordered lists (enumerations), and description lists:
%
\begin{verbatim}
    \begin{itemize}     ... \end{itemize}
    \begin{enumerate}   ... \end{enumerate}
    \begin{description} ... \end{description}
\end{verbatim}
%
List entries are marked using \verb!\item!, for \texttt{description} lists with
\verb!\item[!\texttt{\em title}\verb!]!. Lists can be nested; for
\texttt{itemize} and \texttt{enumerate} lists, the bullets change with nesting
depth (see the \latex documentation for details).

\subsection{Paragraph Formatting and Line Spacing}

A thesis is---like a book---usually formatted in one column and justified, which
makes sense for the continuous text due to the considerable line length.
However, there are often problems with hyphenation and justification within
tables because of the small column width. Using ragged-right aligment (\eg, in
Table~\ref{tab:synthesis-techniques} on page \pageref{tab:synthesis-techniques})
is advisable in such cases.

\subsection{Footnotes}

Footnotes can be placed in \latex at almost any position, but definitely in
normal paragraphs, using the command
%
\begin{LaTeXCode}[numbers=none]
\footnote{/+\emph{Footnote text}+/}
\end{LaTeXCode}
%
There should \emph{never be a space} between the \verb!\footnote! command and
the preceding text (comment out any line breaks with \verb!%!). Numbering
and placement of footnotes is done automatically. Note that large footnotes
may be wrapped over two consecutive pages if necessary.

\subsubsection{Footnotes in Headings}

This may be necessary from time to time, but is no simple task because the
footnote in a heading must only appear next to the title and not in the \emph{table
of contents}! A concrete example is the heading for Chapter~\ref{cha:Closing},
which is defined as follows:
%
\begin{LaTeXCode}[numbers=none]
\chapter[Closing Remarks]%
        {Closing Remarks%
        \protect\footnote{This note ....}}%
\end{LaTeXCode}
%
The first (optional) title \verb![Closing Remarks]! is the entry in the table of
contents and the page header. The second (identical) title \texttt{\{Closing
Remarks\}} appears on the current page and also contains the \verb!\footnote{}!
entry, which, at this point, must be "protected" by the \verb!\protect!
directive. The \verb!%! characters are necessary here to eliminate possible
spaces caused by line breaks in the source text (this trick is often needed in
\latex, see Section~\ref{sec:comments}). All in all, this is quite complicated,
and thus another reason to \emph{avoid} footnotes altogether in such places.

In general, footnotes should be used sparingly, as they interrupt the flow of
the text and distract the reader. In particular, footnotes should not take up a
large part of the page and thus form a second document (as seen in some social
science publications).%
\footnote{In documents with many footnotes, this allegedly leads some readers to
the point where they regularly start reading the footnotes out of curiosity (or
by mistake) and then laboriously search for the associated small-print
references in the main text.}

\subsection{Cross-References}
\label{sec:cross-references}

To manage cross-references within a document, \latex\ provides a straightforward
mechanism. First, each location (chapter, section, figure, table, \etc) must be
marked by
%
\begin{LaTeXCode}[numbers=none]
\label{/+\emph{key}+/}
\end{LaTeXCode}
%
where \texttt{\em key} must be a valid \latex symbol.
To prevent confusion about labels (which are just numbers), it is common to give
them a different prefix depending on their meaning, for example,
%
\begin{quote}
    \tabcolsep0pt
    \begin{tabular}{ll}
        \verb!cha:!\texttt{\em chapter} & \ \ldots\ for chapters, \\
        \verb!sec:!\texttt{\em section} & \ \ldots\ for sections and
        subsections, \\
        \verb!fig:!\texttt{\em figure} & \ \ldots\ for figures, \\
        \verb!tab:!\texttt{\em table} & \ \ldots\ for tables, \\
        \verb!equ:!\texttt{\em equation} & \ \ldots\ for formulae and equations.
    \end{tabular}
\end{quote}
%
\noindent
Examples:\ \verb!\label{cha:Introduction}! or \verb!\label{fig:Screen-1}!.
Using the commands
%
\begin{LaTeXCode}[numbers=none]
\ref{/+\emph{key}+/} /+\quad oder \quad+/ \pageref{/+\emph{key}+/}
\end{LaTeXCode}
%
allows the item or page number associated with \texttt{\emph{key}} to be
inserted anywhere in the document, \eg,
%
\begin{LaTeXCode}[numbers=none]
.. as mentioned in Chapter~\ref{cha:Introduction} ..
.. the screenshot on page \pageref{fig:Screen-1} ..
\end{LaTeXCode}
%
By the way, the terms \emph{chapter} and \emph{section} are frequently misused.
Chapters \emph{always} have whole numbers:
%
\begin{quote}
    \begin{tabular}{ll}
        \textrm{Correct:\ } & Chapter 7 and Section 2.3.4 \\
        \textrm{Wrong:\ } & Chapter 7.2 and Section 5
    \end{tabular}
\end{quote}
%
Also, \emph{Chapter} and \emph{Section}, as well as \emph{Figure}, \emph{Table},
\emph{Program}, or \emph{Equation}, should always be capitalized when used in
connection with a cross-reference:
%
\begin{quote}
	\begin{tabular}{lll}
		\textrm{Correct:\ } & \ldots\ see Figure 4.1 & \ldots\ as stated in
		Chapter 4 \ldots \\
		\textrm{Wrong:\ }  & \ldots\ see figure 4.1  & \ldots\ as stated in
		chapter 4 \ldots
	\end{tabular}
\end{quote}

\subsection{Hyperlinks and E-mail Addresses}

Hyperlinks (URLs) present a particular challenge for typesetting, especially
when line breaks occur. The command
%
\begin{LaTeXCode}[numbers=none]
\url{/+\texttt{\emph{address}}+/}
\end{LaTeXCode}
%
allows line breaks at certain address characters and should always be used when a
hyperlink is specified in the main text or inside a footnote.
%
For e-mail addresses the macro
%
\begin{LaTeXCode}[numbers=none]
\email{/+\texttt{\emph{e-mail address}}+/}
\end{LaTeXCode}
%
is defined in \texttt{hgb.sty}. It creates a correct link in the document with
a \texttt{mailto:} prefix using \verb|\url{}|. The statement can also be used
within the \verb|\author{}| command in the preamble of a document to
additionally specify an e-mail address on the title page:
%
\begin{LaTeXCode}[numbers=none]
\author{%
    Alex A. Wiseguy \\%
    \email{alex@wiseguy.org}%
}
\end{LaTeXCode}



\section{Word Spacing and Punctuation}

While \latex automatically tries to achieve the best possible result in many
typesetting areas, punctuation requires the author's care.

\subsection{\emph{French Spacing}}

In English-language typesetting, it is customary to insert an increased space
(compared to the usual space between words) after the end each sentence.
Although this is not enabled by default for this document (it is also not
traditionally done in German and French), it is sometimes preferred because of
improved readability. If the English ("non-French") sentence separation with
additional spacing is desired, only the line
%
\begin{LaTeXCode}[numbers=none]
\nonfrenchspacing
\end{LaTeXCode}
%
needs to be added at the beginning of the document. In this case, however, the
punctuation within sentences (after .\ and :) should be carefully observed. For
example, "Dr.\ Mabuse" is written in the form
%
\begin{LaTeXCode}[numbers=none]
Dr.\ Mabuse! /+\quad\textrm{or}\quad+/ Dr.~Mabuse
\end{LaTeXCode}
%
In the second example, the \verb!~! symbol also prevents a line break at the
space character.

\subsection{Dashes and Hyphens}
\label{sec:dash}

Confusing dashes with hyphens or similar punctuation marks (with and without
spaces) is a common mistake. The following types should be distinguished:
%
\begin{itemize}
    \item Hyphens (as in "tech-savvy").
    \item Minus signs, \eg $-7$ (created with \verb!$-7$!).
    \item Dashes---such as the em dash here (generated with \verb!---!).
\end{itemize}
%
\noindent
Dashes are used for pauses or indicating ranges. There are clear conventions
for using them:%
\footnote{Both versions also have corresponding special characters in
\emph{Word}.}
%
\begin{enumerate}
	\item In \emph{English} texts, the \emph{em dash} is used \emph{without}
	extra spaces---\emph{as we should know by now} (in \latex by typing
	{\verb*!---!}).
	\item In \emph{German}, the slightly shorter \emph{en dash} surrounded by
	two spaces is usually used -- wie hier zu sehen (in \latex by typing
	{\verb*! -- !})). This dash is also used in both languages to indicate
	intervals of numbers (pages 12--19), but in this case without spaces.
\end{enumerate}

\subsection{Comments}
\label{sec:comments}

Text parts can be commented out line by line in \latex\ with \verb!%!. The text
after a \verb!%! character is ignored until the following end-of-line:
%
\begin{LaTeXCode}[numbers=none]
This will be printed. % And this text will be ignored.
\end{LaTeXCode}
%
Comment characters are frequently used to hide \emph{white space}, \ie, spaces
and line breaks. The following example shows how \verb!%! at the end of a line
can be used to avoid the occurrence of a space before a subsequent footnote
marker:
%
\begin{LaTeXCode}[numbers=none]
In Austria, people eat Schnitzel on Sundays.%
\footnote{Which explains their good health.}
\end{LaTeXCode}
%
Similarly, the occurrence of unwanted paragraph space can be avoided by the
selective use of comment lines, \eg, before and after a centered text section:
%
\begin{LaTeXCode}[numbers=none]
... normal text.
%
\begin{center}
   This text is centered.
\end{center}
%
And now it continues normally ...
\end{LaTeXCode}
%
In addition, the \verb!comment! environment can be used to hide larger text blocks
in one piece:
\begin{LaTeXCode}[numbers=none]
\begin{comment}
This text ...
   ... is ignored.
\end{comment}
\end{LaTeXCode}



\subsection{Quotation Marks}
\label{sec:quotation-marks}

Quotation marks are a common (and often unnoticed) source of error; again, the
differences between English and German (among other languages) should be noted.

\subsubsection{Version 1: Quotation Marks Using \latex's Default Setting}

With \latex's default setting (\ie, \emph{without} using the document option
\texttt{smartquotes}, see below), input of leading and trailing
quotation marks must strictly follow the appropriate conventions. Here is the
correct \latex notation for English and German texts, respectively:
%
\begin{quote}
    \verb!``English''! $\rightarrow$ ``English'',\\
    \verb!"`Deutsch"'! $\rightarrow$ {\glqq}Deutsch{\grqq}.
\end{quote}
%
Note the subtle typographical differences between the two languages.%
\footnote{Some editors (\eg, \textsf{TeXstudio}) can be configured to use the
corresponding quotation marks \emph{automatically} (context- and
language-dependent) when typing a double quote character
(\texttt{\textquotedbl}). However, this is currently not possible in
\textsf{Overleaf}.}

\emph{Single} quotation marks are generated analogously in English. In German,
however, the macros \verb!\glq! and \verb!\grq! (German left/right quote) are
required:
%
\begin{quote}
    \verb!`English'! $\rightarrow$ `English',\\
    \verb!{\glq}Deutsch{\grq}! $\rightarrow$ {\glq}Deutsch{\grq}.
\end{quote}

\subsubsection{Version 2: Quotation Marks Using the \texttt{smartquotes} Option}

Setting the \texttt{smartquotes} document option (as done in \emph{this} document)
activates a \emph{special setup} based on the \texttt{csquotes} package.%
\footnote{\url{https://ctan.org/pkg/csquotes}}
This clearly simplifies the right use of quotation marks because the
correct versions are inserted automatically depending on the current language
setting and the position of the quote character. It is sufficient to use a double
quote character \texttt{\textquotedbl} to achieve this, for example,
%
\begin{quote}
	\verb!"English"! $\rightarrow$ "English" (with language setting
	\texttt{english}),\\
	\begin{german}%
		\verb!"Deutsch"! $\rightarrow$ "Deutsch",
	\end{german}
	(with language setting \texttt{german}).
\end{quote}
%
It should be noted that the standard quotation marks (Version~1, see
above) are \emph{not} available in this case and thus the combined use of
Versions 1 and 2 is not possible! With this setting, all other shorthands of
the \texttt{babel} package%
\footnote{\url{https://ctan.org/pkg/babel}}
(\eg, \verb!"a!, \verb!"o!, \verb!"u!) are also \emph{permanently disabled}
and cannot be reactivated locally either.%
\footnote{The use of the \texttt{\textquotedbl} character as the double-sided
"outer quote" character is considered "dangerous"---especially in combination
with the German language---because the \texttt{babel} package uses the double
quote character for special \emph{shorthand} macros. We bravely ignore this,
though the \texttt{babel}-shorthands are generally disabled in the current
setup to avoid trouble.}

\subsubsection{Additional Features of the \texttt{csquotes} Package}

The \texttt{csquotes} package (automatically loaded with the
\texttt{smartquotes} option) provides many more possibilities for entering
quoted text (quotations), especially the command
%
\begin{itemize}
    \item[] \verb!\enquote{text}!,
\end{itemize}
%
which typesets the given \texttt{text} in the correct form (among other things
depending on the language setting and nesting depth) as a citation, \eg,
%
\begin{itemize}
    \item[] \verb|\enquote{I have a dream.}|
    \item[] $\rightarrow$ \enquote{I have a dream.}
\end{itemize}
%
The advantage of this construct is especially apparent in \emph{nested}
quotations, as, for example in
%
\begin{itemize}
    \item[] \verb|\enquote{Napoleon just said \enquote{Keep going!} and left.}|
    \item[] $\rightarrow$ \enquote{Napoleon just said \enquote{Keep going!} and
    left.}
\end{itemize}
%
Another handy feature is the command \verb!\foreignquote! which makes it very
easy to insert foreign quotations in the text without explicitly changing the
language setting, for example,%
\footnote{Currently only the language settings \texttt{english} and
\texttt{german} are available.}
%
\begin{itemize}
    \item[] \verb|\foreignquote{german}{Da sprach der Herr zu Kain:|\newline
    \verb|   \enquote{Wo ist dein Bruder Abel?} Er entgegnete: \ldots}|
    \item[] $\rightarrow$ \foreignquote{german}{Da sprach der Herr zu Kain:
    \enquote{Wo ist dein Bruder Abel?}  Er entgegnete: \ldots}
\end{itemize}


\section{Hyphenation}
\label{subsec:hyphenation}

Hyphenation is essential to achieve a clean typography, especially for long words.
It is done either \emph{automatically} or \emph{manually} by inserting optional
hyphens.

\subsection{Automatic Line Break}

In \latex, hyphenation is generally done automatically. The language is set at
the beginning of the document, and appropriate hyphenation rules are applied to
the entire text.

Especially for narrow text columns, \latex may not find a suitable place to
break the line and lets the text run beyond the right margin. This is intentional
and meant to indicate a problem that needs to be repaired through manual
intervention.

\subsection{Manual Line Break}

Generally, one should be suspicious of automatic hyphenation and always
check the final result carefully. Especially words with umlauts or compound
words with hyphens (see below) are often split incorrectly by \latex.

\paragraph{Optional line breaks:} If required, additional hyphenation
points can be specified with \verb!\-!, as for example in
%
\begin{itemize}
    \item[] \verb!in\-com\-pre\-hen\-si\-ble!.
\end{itemize}

\paragraph{Compound words:} An unpleasant peculiarity of \latex is that in a
\emph{hyphenated} word, the individual parts are generally \emph{not
automatically} hyphenated! This is quite common, especially (but not only) in
German texts, and thus annoying; for example, \latex will not hyphenate
\emph{either} of the two parts of the word
%
\begin{itemize}
    \item[] \verb!anti-intellectualism!
\end{itemize}
%
but, if necessary, let it run beyond the right margin! Manual hyphenation by
inserting \verb!\-! can once again be helpful.

\paragraph{"Sloppy" formatting:} In real problem cases---for example, text
elements that must not or cannot be wrapped---\latex\ can be told to be less
strict about formatting specific paragraphs. This is achieved as follows:
%
\begin{LaTeXCode}[numbers=none]
\begin{sloppypar}
    This paragraph is set "sloppy" ...
\end{sloppypar}
\end{LaTeXCode}
%
The last resort is to rewrite the passage in question in such a way that it
results in a decent line break---after all, it is one's own work, and no
justification is owed to anyone (except perhaps the supervisor).%
\footnote{It is said that such independent changes to the text by typesetters were
quite common even in the metal type days.}



\section{The \texttt{hagenberg-thesis} Package}

The \texttt{hagenberg-thesis} package provides several \latex\ files that are
required for this document:
%
\begin{itemize}
  \item \nolinkurl{hgbthesis.cls} (class file): Defines the document
    structure, layout, and the entire preamble of the document (title page,
    \etc).
  \item \nolinkurl{hgbpdfa.sty} (style file): Definitions for producing
    PDF/A-compliant output. Must be loaded by the main document before
    the \verb!\documentclass! command (see also Sec.\ \ref{sec:PDFA}).
  \item \nolinkurl{hgb.sty} (style file): Contains central definitions and
    settings. This file is automatically loaded by \nolinkurl{hgbthesis.cls},
    but it can also be used for other documents.
  \item Additional style files imported by \nolinkurl{hgbthesis.cls}:
    \begin{itemize}
        \item[] \nolinkurl{hgbabbrev.sty} (various abbreviations),
        \item[] \nolinkurl{hgbalgo.sty} (algorithms),
        \item[] \nolinkurl{hgbbib.sty} (reference management),
        \item[] \nolinkurl{hgbheadings.sty} (page headers),
        \item[] \nolinkurl{hgblistings.sty} (code listings),
        \item[] \nolinkurl{hgbmath.sty} (mathematical functionalities).
    \end{itemize}
\end{itemize}


\subsection{Settings}
\label{sec:hagenberg-settings}

All sample (\verb!.tex!) documents of this package start with the statement
%
\begin{itemize}
    \item[] \verb!\documentclass[!\texttt{type=\emph{type}},%
    \texttt{language=\emph{language}}\verb!]{hgbthesis}!.
\end{itemize}
%
The \texttt{\emph{type}} option specifies the type of document:
%
\begin{itemize}
    \item[] \texttt{master} (master thesis = \emph{default}),
    \item[] \texttt{diploma} (diploma thesis),
    \item[] \texttt{bachelor} (bachelor thesis),
    \item[] \texttt{internship} (internship report).
\end{itemize}
%
The \texttt{\emph{language}} option can be used to specify the primary language
of the document; possible values are:
%
\begin{itemize}
    \item[] \texttt{german} (\emph{default}),
    \item[] \texttt{english}.
\end{itemize}
%
Additional options:
%
\begin{itemize}
		\item[] \texttt{proposal} (exposé, in connection with 
			\texttt{type=bachelor} or \texttt{master}).
    \item[] \texttt{smartquotes} (use of upright double quotes, see
			Section~\ref{sec:quotation-marks}).
		\item[] \texttt{oneside} (\emph{default}) or \texttt{twoside} for 
			a one- or two-page document layout, respectively.
\end{itemize}
%
If \emph{none} of the above options is specified, the default settings are 
\begin{itemize}
	\item[] \texttt{[type=master,language=german,oneside]} % theme=default,
\end{itemize}
are used.%
\footnote{The full source code for a corresponding \texttt{.tex} main file is listed in
Appendix~\ref{app:latex}.}



\subsubsection{Details of the Thesis or Report}

The document class is intended for different types of works that differ only in
the structure of the title pages. Different elements are required for the title
pages depending on the selected document type (see
Table~\ref{tab:TitleElements}). The following information is required for
\emph{all} document types:
%
\begin{itemize}
    \item[] %
    \verb!\title{!\texttt{\em Thesis or report title}\verb!}!, \newline%
    \verb!\author{!\texttt{\em Author}\verb!}!, \newline%
    \verb!\programtype{!\texttt{\em Type of program}\verb!}!, \newline%
    \verb!\programname{!\texttt{\em Program}\verb!}!, \newline%
    \verb!\placeofstudy{!\texttt{\em Place of study}\verb!}!, \newline%
    \verb!\dateofsubmission{!\texttt{\em yyyy}\verb!}{!\texttt{\em
    mm}\verb!}{!\texttt{\em dd}\verb!}!, \newline%
    \verb!\advisor{!\texttt{\em Supervisor name}\verb!}! --
    optional.
\end{itemize}
%
\noindent
For \emph{internship reports}, the following elements are considered in
addition to the basic information:
%
\begin{itemize}
    \item[] %
    \verb!\companyName{!\texttt{\em Name and address of the company}\verb!}!,
    \newline%
    \verb!\companyUrl{!\texttt{\em Company website}\verb!}!.
\end{itemize}

\begin{table}
    \caption{Elements in title pages for different document options.}
    \label{tab:TitleElements}
    \centering\small
    \begin{tabular}{@{}lcccc@{}}
        \toprule
        \emph{Element} & \texttt{master} & \texttt{bachelor} & 
				\texttt{diploma} & \texttt{internship} \\
        \midrule
        \verb!\title!            & $+$ & $+$ & $+$ & $+$ \\
        \verb!\author!           & $+$ & $+$ & $+$ & $+$ \\
        \verb!\programtype!      & $+$ & $+$ & $+$ & $+$ \\
        \verb!\programname!      & $+$ & $+$ & $+$ & $+$ \\
        \verb!\placeofstudy!     & $+$ & $+$ & $+$ & $+$ \\
        \verb!\dateofsubmission! & $+$ & $+$ & $+$ & $+$ \\
        \verb!\advisor!          & $+$ & $+$ & $+$ & $+$ \\
        \verb!\companyName!      & $-$ & $-$ & $-$ & $+$ \\
        \verb!\companyPhone!     & $-$ & $-$ & $-$ & $+$ \\
        \bottomrule
    \end{tabular}
\end{table}

\subsubsection{Title Pages}

The first pages of the work, including the title page, are created by the
command
%
\begin{itemize}
    \item[] \verb!\maketitle!
\end{itemize}
%
depending on the above settings:
%
\begin{center}
    \begin{tabular}{@{}cl@{}}
        \toprule
        \emph{Page} & \emph{Content} \\
        \midrule
        \textrm{i}   & Title page \\
        \textrm{ii}  & Advisor page (only when \verb!\advisor! is specified) \\
        \textrm{iii} & Copyright page \\
        \textrm{iv}  & Affidavit \\
        \bottomrule
    \end{tabular}
\end{center}
%
The copyright page%
\footnote{If the \texttt{proposal} option is used, the pages for copyright and
affidavit are omitted.}
also notes the conditions for the use and distribution of the work. The
following settings can determine the associated text at the beginning of the
document.
%
\begin{description}
    \item[\normalfont\texttt{{\bs}cclicense}] ~ \newline
    Publication under a Creative Commons license%
    \footnote{\url{https://creativecommons.org/licenses/by-nc-nd/4.0/}}
    that permits free redistribution of the work with attribution to the author,
    but no commercial use or adaptation. This is the \emph{recommended} default
    setting.
    \item[\normalfont\texttt{{\bs}strictlicense}] ~ \newline
    Classic restriction of usage rights (\emph{All Rights Reserved}).
    \item[\normalfont\texttt{{\bs}license\{\emph{License text}\}}] ~ \newline
    If necessary, this can specify an alternative \text{\emph{license text}}.
    Such changes should, of course, be coordinated with the university.
\end{description}


\subsection{Defined Abbreviations}

There are also several abbreviation commands%
\footnote{Similar to the \texttt{jkthesis} style by J. Küpper
(\url{https://ctan.org/pkg/jkthesis}).}
defined in the \texttt{hagenberg-thesis} package, which simplify writing and
provide consistent spacing (see Table~\ref{tab:abbreviations}). When using any
command, it should generally be noted that they sometimes "gobble" trailing
spaces so that no spacing is created before the subsequent text.%
\footnote{For almost all macros defined in \texttt{hgbabbrev.sty} this
is prevented by the use of \texttt{\textbackslash xspace}.}
If necessary, this can be prevented with a subsequent "\verb*!\ !" or by wrapping
the expression in \verb!{}! brackets. When using commands with a trailing period
at the end of a sentence, care should also be taken to avoid \emph{double}
periods.


\begin{table}
    \caption{English and German abbreviation commands defined in
    \texttt{hgbabbrev.sty}.}
    \label{tab:abbreviations}
    \centering\small
    \begin{tabular}{@{}llp{2cm}ll@{}}
        \toprule
        \verb+\ie+   & \ie   & & \verb+\eg+  & \eg  \\
        \verb+\wrt+  & \wrt  & & \verb+\Eg+  & \Eg  \\
        \midrule
        \verb+\bzw+  & \bzw  & & \verb+\ua+  & \ua  \\
        \verb+\bzgl+ & \bzgl & & \verb+\Ua+  & \Ua  \\
        \verb+\ca+   & \ca   & & \verb+\uae+ & \uae \\
        \verb+\dah+  & \dah  & & \verb+\usw+ & \usw \\
        \verb+\Dah+  & \Dah  & & \verb+\uva+ & \uva \\
        \verb+\ds+   & \ds   & & \verb+\uvm+ & \uvm \\
        \verb+\evtl+ & \evtl & & \verb+\va+  & \va  \\
        \verb+\ia+   & \ia   & & \verb+\vgl+ & \vgl \\
        \verb+\sa+   & \sa   & & \verb+\zB+  & \zB  \\
        \verb+\so+   & \so   & & \verb+\ZB+  & \ZB  \\
        \verb+\su+   & \su   & & \verb+\etc+ & \etc \\
				\verb+\oa+   & \oa   & &   									\\
        \bottomrule
    \end{tabular}
\end{table}

\subsection{Language Switching}
\label{sec:language-switching}

For German-language sections (\eg, the \emph{Kurzfassung} or German quotations), the
language should be switched from English to German to get the correct
hyphenation. To avoid accidentally forgetting to reset the language, two special
\latex\ environments are provided for this purpose:
%
\begin{LaTeXCode}[numbers=none]
\begin{german}
    An dieser Stelle steht eine Zusammenfassung der Arbeit.
\end{german}
\end{LaTeXCode}
%
and, analogously,
%
\begin{LaTeXCode}[numbers=none]
\begin{english}
    Text in English (if the surrounding language is German).
\end{english}
\end{LaTeXCode}
%
The current language setting can be displayed with the command
\verb!\languagename!. At this point, this results in
"\texttt{\languagename}".



\subsection{Additional \latex\ Packages}

Several additional \latex\ packages are required to use this document
(Table~\ref{tab:packages}). These packages are automatically loaded by the
\texttt{hagenberg-thesis} package. All packages used are part of the \latex\
standard installation, as, for example, in MikTeX, where corresponding
documentation can also be found (mainly as DVI files). The current versions of
the packages are available online on the CTAN sites given in
Section~\ref{sec:literature}.

\begin{table}
    \caption{The most important of the \latex extensions used in the
    \texttt{hagenberg-thesis} package. These are already included in typical
    \latex\ standard installations (\eg, MikTeX).}
    \label{tab:packages}
    \centering\small
    \begin{tabular}{@{}ll@{}}
        \toprule
        \emph{Package}        & \emph{Function} \\
        \midrule
        \texttt{algorithmicx} & Description of algorithms \\
        \texttt{amsfonts},
        \texttt{amsbsy}       & Mathematical symbols \\
        \texttt{amsmath}      & Mathematical typesetting \\
        \texttt{babel}        & Language switching \\
        \texttt{biblatex}     & Bibliography management \\
        \texttt{caption}      & More flexible captions \\
        \texttt{csquotes}     & Context-sensitive quotation marks \\
        \texttt{marvosym}     & \euro symbol (\verb!\euro!) \\
        \texttt{exscale}      & Correct font sizes in math mode \\
        \texttt{fancyhdr}     & Controlling headers and footers \\
        \texttt{float}        & Improved float handling \\
        \texttt{fontenc}      & Use of T1 (Western European) character encoding \\
        \texttt{graphicx}     & Including of graphics \\
        \texttt{hyperref}     & Handling of cross-references in the PDF document \\
        \texttt{ifthen},
        \texttt{xifthen}      & Conditional commands in \latex \\
        \texttt{inputenc}     & Extended input encodings \\
        \texttt{listings}     & Code listings \\
        \texttt{upquote}      & Realistic quotes in \texttt{verbatim} texts \\
        \texttt{url}          & URL handling in text \\
        \texttt{verbatim}     & Better \texttt{verbatim} environments \\
        \texttt{xcolor}       & Colored text elements and background colors \\
        \bottomrule
    \end{tabular}
\end{table}


\section{\latex\ Error Messages and Warnings}

During a run \latex\ outputs tons of messages and one should not be
confused by their abundance, \eg:

\begin{GenericCode}[numbers=none]
...
Underfull \hbox (badness 2744) in paragraph at lines 568--572
\T1/lmr/m/n/10.95 be-tween in-di-vid-ual
[]
Underfull \hbox (badness 5607) in paragraph at lines 580--581
\T1/lmr/m/n/10.95 only as a place-
[]
...
\end{GenericCode}

\noindent
All \emph{errors} must be corrected, but \latex\ does not make this job easy.
Sometimes (\eg, when a closing bracket \verb!}! was forgotten), the problem may
not be located until much later in the text. In such cases, inspecting the
generated output document can be useful to determine at what point the result
gets out of hand. In case of capital errors, the \latex processor stops
completely and does not generate any output at all (often in connection with a highly
cryptic error message). In this case, a detailed analysis of the source code or
the immediate steps before the error will usually help. A detailed error log can be
found in the \verb!.log! file for the main document.

If no errors are shown, at least the syntactical structure of the document is
okay. However, one should take a closer look at the list of messages at the
latest when finishing the thesis. This is highly recommended to eliminate any
remaining problems, such as overlong lines of text (overfull boxes), unresolved
references, and similar issues. In any case, the console output should look like 
this eventually:
%
\begin{GenericCode}[numbers=none]
LaTeX-Result: 0 Error(s), 0 Warning(s), ...
\end{GenericCode}


