\chapter[Closing Remarks]{Closing Remarks%
\protect\footnote{This note only demonstrates the (in rare cases reasonable)
usage of footnotes in headings.}}%
\label{cha:Closing}

An dieser Stelle sollte eine Zusammenfassung der Abschlussarbeit stehen, in
der auch auf den Entstehungsprozess, persönliche Erfahrungen, Probleme bei
der Durchführung, Verbesserungsmöglichkeiten, mögliche Erweiterungen \usw\
eingegangen werden kann. War das Thema richtig gewählt, was wurde konkret
erreicht, welche Punkte blieben offen und wie könnte von hier aus
weitergearbeitet werden?


\section{Lesen und lesen lassen}

Wenn die Arbeit fertig ist, sollten Sie diese zunächst selbst nochmals
vollständig und sorgfältig durchlesen, auch wenn man vielleicht das mühsam
entstandene Produkt längst nicht mehr sehen möchte. Zusätzlich ist sehr zu
empfehlen, auch einer weiteren Person diese Arbeit anzutun -- man wird
erstaunt sein, wie viele Fehler man selbst überlesen hat.

Auch der Einsatz von KI-unterstützten Schreibassistenten wie \zB
\emph{Grammarly}%
\footnote{\url{https://grammarly.com/}} oder \emph{LanguageTool}%
\footnote{\url{https://languagetool.org/}} kann durchaus sinnvoll sein.
Jedoch sollten die Vorschläge dieser Werkzeuge nicht einfach blind, sondern
mit Bedacht angenommen werden.


\section{Checkliste}

Abschließend noch eine kurze Liste der wichtigsten Punkte, an denen
erfahrungsgemäß die häufigsten Fehler auftreten (Tab.\ \ref{tab:checkliste}).

\begin{table}
	\caption{Checkliste. Diese Punkte bilden auch die Grundlage der
	routine\-mäßigen Formbegutachtung in Hagenberg.}
	\label{tab:checkliste}
	\centering%
	\setlength{\fboxrule}{0.2mm}%
	\setlength{\fboxsep}{2mm}%
	\fbox{%
		\begin{minipage}{0.95\textwidth}
			\begin{itemize}
				\item[$\Box$] \textbf{Titelseite:}
				Länge des Titels (Zeilenumbrüche), Name, Studiengang, Datum.
				\item[$\Box$] \textbf{Erklärung:}
				vollständige Unterschrift.
				\item[$\Box$] \textbf{Inhaltsverzeichnis:}
				balancierte Struktur, Tiefe, Länge der Überschriften.
				\item[$\Box$] \textbf{Kurzfassung/Abstract:}
				präzise Zusammenfassung, passende Länge, gleiche Inhalte und
				Struktur.
				\item[$\Box$] \textbf{Überschriften:}
				Länge, Stil, Aussagekraft.
				\item[$\Box$] \textbf{Typographie:}
				sauberes Schriftbild, keine "manuellen" Abstände zwischen
				Absätzen oder Einrückungen, keine überlangen Zeilen,
				Hervorhebungen, Schriftgröße, Platzierung von Fußnoten.
				\item[$\Box$] \textbf{Sprache:}
				geschlechtergerechte Formulierungen (kein generisches
				Maskulinum oder Generalklausel), neutraler, sachlicher Stil,
				keine übermäßigen Anglizismen.
				\item[$\Box$] \textbf{Interpunktion:}
				Binde- und Gedankenstriche richtig gesetzt, Abstände nach
				Punkten (\va\ nach Abkürzungen), korrekte (vordere/hintere)
				Hochkommas.
				\item[$\Box$] \textbf{Abbildungen:}
				Qualität der Grafiken und Bilder, Schriftgröße und -typ in
				Abbildungen, Platzierung von Abbildungen und Tabellen, Captions.
				Sind \emph{alle} Abbildungen (Tabellen) im Text referenziert?
				\item[$\Box$] \textbf{Gleichungen/Formeln:}
				mathem.\ Elemente auch im Fließtext richtig gesetzt, explizite
				Gleichungen richtig verwendet, Verwendung von mathem.\ Symbolen.
				\item[$\Box$] \textbf{Quellenangaben:}
				Zitate richtig referenziert, Seiten- oder Kapitelangaben.
				\item[$\Box$] \textbf{Literaturverzeichnis:}
				Art der Publikation muss in jedem Fall klar sein, konsistente
				und vollständige Einträge, Online-Quellen (URLs) sauber
				angeführt.
				\item[$\Box$] \textbf{Sonstiges:}
				ungültige Querverweise (\textbf{??}), Anhang, Papiergröße der
				PDF-Datei (A4 = $8.27 \times 11.69$ Zoll), Druckgröße und -qualität.
			\end{itemize}
		\end{minipage}%
	}
\end{table}


