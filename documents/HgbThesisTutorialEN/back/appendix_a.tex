\chapter{Technical Information}
\label{app:TechnicalDetails}

\section{Current Package Version}

\begin{center}
	\begin{tabular}{@{}ll@{}}
		\toprule
		Date   & File            \\
		\midrule
		\hgbDate & \texttt{hgb.sty} \\
		\bottomrule
	\end{tabular}
\end{center}


\section{Additional Details}

This package is designed for \mbox{UTF-8} encoded source files and supports 
\latex\ in direct PDF mode only.%
\footnote{The "classic" DVI-PS-PDF process is no longer supported.}


\subsection{Technical Requirements}

A current \latex\ installation including
%
\begin{itemize}
		\item \texttt{biber} (modern replacement for BibTeX, Version $\geq 1.5$),
		\item \texttt{biblatex} package (version $\geq 2.5$, 2013/01/10),
		\item Latin Modern fonts (package \texttt{lmodern}).%
			\footnote{\url{https://ctan.org/pkg/lm},
				\url{https://tug.org/FontCatalogue/latinmodernroman/}}
\end{itemize}
%
In addition, a text editor for {UTF-8} encoded (Unicode) files, as well 
as software for opening and viewing PDF files.



\subsection{Use Under Windows}
\label{sec:UseUnderWindows}

A typical installation under Windows looks like this:
%
\begin{enumerate}
\item \textbf{MikTeX}%
	\footnote{\url{https://miktex.org/} -- 
  Generally, the \emph{complete installation} of MikTeX ("Complete MiKTeX") is 
	recommended, as it already contains all necessary additional packages and font files.
	During installation, make sure that the automatic installation of required packages
	is enabled by "\emph{Install missing packages on-the-fly: = Yes}" 
	(NOT "\emph{Ask me first}")!
	It is also recommended to update the installed packages immediately after installing MikTeX 
	and periodically thereafter using the \texttt{MikTeX Console} program.} 
	(basic \latex\ environment),
\item \textbf{TeXstudio}%
	\footnote{\url{https://www.texstudio.org/}}
	(text editor, supports UTF-8 and includes an integrated PDF viewer).
\end{enumerate}
%
Alternative editors and PDF viewers are:
%
\begin{enumerate}
	\item Visual Studio Code%
	\footnote{\url{https://code.visualstudio.com/}}
	with LaTeX Workshop Extension,%
	\footnote{\url{https://marketplace.visualstudio.com/items?itemName=James-Yu.latex-workshop}}
	\item IntelliJ IDEA,%
	\footnote{\url{https://www.jetbrains.com/idea/}}
	with TeXiFy-IDEA plugin,%
	\footnote{\url{https://plugins.jetbrains.com/plugin/9473-texify-idea}
	\item Lyx,%
	\footnote{\url{https://www.lyx.org/}}
	\item TeXworks,%
	\footnote{\url{https://www.tug.org/texworks/}}
	\item WinEdt,%
	\footnote{\url{https://www.winedt.com/}}
	\item Sumatra PDF ("\latex\ friendly" PDF viewer).%
	\footnote{\url{https://www.sumatrapdfreader.org/}}
\end{enumerate}


\subsection{Use Under macOS}

For macOS, the following configuration is recommended:
%
\begin{enumerate}
\item 
	\textbf{MacTex}%
	\footnote{\url{https://tug.org/mactex/} -- 
	Current MacTeX distributions usually require a mostly up-to-date version of macOS. 
	On older versions, \emph{TeXLive} can alternatively be installed with a special
	installation script. To keep the packages of the \latex\ distribution up-to-date, the 
	\emph{TeX Live Utility} program should be run regularly.}
	(basic \latex\ environment),
\item \textbf{TeXstudio} (text editor, supports UTF-8 and includes an integrated PDF viewer).
\end{enumerate}
%
Alternative editors and PDF viewers are:
%
\begin{enumerate}
	\item Visual Studio Code with LaTeX Workshop Extension,%
	\item Lyx,%
	\item TeXworks,%
	\item Skim ("\latex\ friendly" PDF viewer).%
	\footnote{\url{https://skim-app.sourceforge.io/}}
\end{enumerate}


\subsection{Use Under Linux}

Under Linux the following setup can be used:
%
\begin{enumerate}
	\item 
	\textbf{TeX Live}%
	\footnote{\url{https://tug.org/texlive/} -- An installation under Linux is---depending
	on the distribution used---most easily done with the help of the associated package
	management system (\eg, \texttt{apt-get}).}
	(basic \latex\ environment),
	\item \textbf{TeXstudio} (text editor, supports UTF-8 and includes an
	integrated PDF viewer).
\end{enumerate}
%
Alternative editors and PDF viewers are:
%
\begin{enumerate}
	\item Visual Studio Code with LaTeX Workshop Extension,%
	\item Lyx,%
	\item TeXworks,%
	\item qpdfview ("\latex\ friendly" PDF viewer).%
	\footnote{\url{https://launchpad.net/qpdfview}}
\end{enumerate}


\subsection{Using Online \latex\ Environments}

Besides using a local \latex installation and editor, there are now also good online 
environments that allow to create \latex documents directly in the browser.
The \latex environment is installed on the servers of the provider. Documents can be
created in the online editor or existing templates (such as this document) uploaded and 
edited. Most platforms also allow collaborative work on the same document.

When using such environments, it is highly recommended to perform regular \emph{backups} of your
online data while working, so that in the worst case you don't have to start all over 
again.

\subsubsection{Overleaf}

The most popular editor (tested with this template) is \emph{Overleaf}%.
\footnote{\url{https://www.overleaf.com/}}.
To quickly import template documents from the \texttt{hagenberg-thesis} package, the 
import links in the \emph{readme} section of this project's Github repository%
\footnote{\url{https://github.com/Digital-Media/HagenbergThesis}}
can be used directly.

Please note that free Overleaf accounts have reduced their compile timeouts to only 20 seconds since the end of 2023. This means that larger theses and this template document can no longer be created with it because the PDF generation aborts after 20 seconds.

The only workaround is to purchase a paid account (license) or ditch Overleaf entirely and switch to a local setup. In educational institutions that provide licenses for teachers, they can create a document and share it with students to collaborate, ensuring a smooth workflow.


\subsubsection{Other Online Services}

Besides, there are other online environments for \latex\ and their number is constantly growing, for example:
%
\begin{enumerate}
	\item Papeeria,%
	\footnote{\url{https://papeeria.com/}}
	\item CoCalc.%
	\footnote{\url{https://cocalc.com/}}
\end{enumerate}
