%%% Einfaches Template für einen Abschlussbericht zum Berufspraktikum
%%% äöüÄÖÜß  <-- keine deutschen Umlaute hier? UTF-faehigen Editor verwenden!

%%% Magic Comments zum Setzen der korrekten Parameter in kompatiblen IDEs
% !TeX encoding = utf8
% !TeX program = pdflatex 
% !TeX spellcheck = de_DE
% !BIB program = biber

\RequirePackage{hgbpdfa}      % PDF/A output

\documentclass[internship,german,smartquotes]{hgbthesis}
% Zulässige Optionen in [..]: 
%    Typ der Arbeit: 'diploma', 'master' (default), 'bachelor', 'internship'
%		 Zusätzlich für ein Thesis-Exposé: 'proposal' (für 'bachelor' und 'master')
%    Hauptsprache: 'german' (default), 'english'
%    Option zur Umwandlung in typografische Anführungszeichen: 'smartquotes'
%    APA Zitierstil: 'apa'
%%%-----------------------------------------------------------------------------

\RequirePackage[utf8]{inputenc} % bei Verw. von lualatex oder xelatex entfernen!

\graphicspath{{images/}}  % Verzeichnis mit Bildern und Grafiken
\logofile{logo}           % Logo-Datei: images/logo.pdf (kein Logo: \logofile{})
\bibliography{references} % Biblatex-Literaturdatei (references.bib)

%%%-----------------------------------------------------------------------------
\begin{document}
%%%-----------------------------------------------------------------------------

%%%-----------------------------------------------------------------------------
% Angaben für die Titelei (Titelseite, Erklärung etc.)
%%%-----------------------------------------------------------------------------

\title{Endbericht zum Berufspraktikum bei Mogulovich International}
\author{Alex A.\ Schlaumeier}

\programtype{Fachhochschul-Bachelorstudiengang}
\programname{Medientechnik und -design}
\placeofstudy{Hagenberg}

\dateofsubmission{2023}{06}{27}
\advisor{Pjotr I.~Czar, M.A.}
\companyName{%
   Mogulovich International Media GmbH\\
   Online Division\\
   Hubertusgasse 3a, 1020 Wien
}
\companyUrl{www.mogul.at}

%%%-----------------------------------------------------------------------------
\frontmatter                                       % Titelei (röm. Seitenzahlen)
%%%-----------------------------------------------------------------------------

\maketitle
\tableofcontents

%%%-----------------------------------------------------------------------------

\chapter{Kurzfassung}

Umfang der Kurzfassung: ca.\ 200 Worte.

Zum allgemeinen Inhalt des Berichts: Dieser Bericht beschreibt den Ablauf des
Praktikums, die Aufgaben und durchgeführten Projekte und Erfahrungen. Die
eigenen Aktivitäten (Projekte) stehen dabei natürlich im Mittelpunkt und bilden
den Hauptteil des Berichts. Wenn viele Kleinprojekte bearbeitet wurden, sollten
einige davon exemplarisch genauer beschrieben werden. Neben der eigentlichen
Arbeit sollten aber auch folgende weitere Aspekte berücksichtigt werden:
%
\begin{itemize}
	\item Abläufe (Workflows) innerhalb des Unternehmens bzw.\ in Projekten
	(grafische Darstellungen können dabei nützlich sein),
	\item Arbeits- und Führungsstil, Kommunikation innerhalb des Unternehmens,
	\item Kommunikation nach außen (Kund*innen, Partner*innen),
	\item Zeitsituation, Terminprobleme,
	\item Einbettung in das Team, soziale Erfahrungen,
	\item Einsatz von speziellen Techniken, Methoden und Werkzeugen,
	\item wichtige Herausforderungen oder Schwierigkeiten,
	\item Anforderungen in Bezug auf die Ausbildung im Studium (gut
	einsetzbare Kenntnisse, Defizite).
\end{itemize}
%
Die nachfolgenden Kapitelüberschriften sollen nur zur Orientierung für die
Struktur des Berichts dienen, über die konkrete Einteilung und den Wortlaut
kann man natürlich selbst entscheiden.

%%%-----------------------------------------------------------------------------
\mainmatter                             % Hauptteil (ab hier arab. Seitenzahlen)
%%%-----------------------------------------------------------------------------

\chapter{Das Unternehmen}

Umfang: 1--2 Seiten

%%%-----------------------------------------------------------------------------

\chapter{Projekte und Tätigkeiten während des Praktikums}

Umfang: 2--3 Seiten (Projektziel(e), Projektumfeld)

%%%-----------------------------------------------------------------------------
     
\chapter{Projektbeispiele}

Umfang: 5--6 Seiten (Umsetzung, grober Terminplan, Ergebnisse,
Qualitätssicherungsmaßnahmen)

%%%-----------------------------------------------------------------------------

\chapter{Erfahrungen und Zusammenfassung}

Umfang: 1--2 Seiten

%%%-----------------------------------------------------------------------------
%\backmatter        % Schlussteil (nur bei Verwendung des Quellenverzeichnisses)
%%%-----------------------------------------------------------------------------

%\MakeBibliography % Quellenverzeichnis (sofern notwendig, sonst weglassen)

\end{document}
