\chapter[Mathem.\ Formeln etc.]{Mathematische Formeln, Gleichungen und Algorithmen}
\label{cha:Mathematik}

Das Setzen von mathematischen Elementen gehört sicher zu den Stär\-ken
von \latex. Man unterscheidet zwischen mathematischen Elementen im Fließtext
und freistehenden Gleichungen, die in der Regel fortlaufend nummeriert werden
. Analog zu Abbildungen und Tabellen sind dadurch Querverweise zu Gleichungen
leicht zu realisieren. Hier nur einige Beispiele und spezielle Themen, vieles
weitere dazu findet sich \zB in \cite[Kap.\ 7]{Kopka2003} und~\cite{Voss2014}.


\section{Mathematische Elemente im Fließtext}

Mathematische Symbole, Ausdrücke, Gleichungen etc.\ werden im Fließtext durch
paarweise \verb!$! \ldots \verb!$! markiert. Hier ein simples Beispiel:
%
\begin{itemize}
	\item[]
	Der Nah-Unendlichkeitspunkt liegt bei
	$\bar{a} = f' \cdot (f' / (K \cdot u_{\max}) + 1)$,
	sodass bei einem auf $\infty$ eingestellten Objektiv von der Entfernung
	$\bar{a}$ an alles scharf ist. Fokussiert man das Objektiv auf die
	Entfernung $\bar{a}$ (\dah, $a_0 = \bar{a}$), dann wird im Bereich
	$[\frac{\bar{a}}{2}, \infty]$ alles scharf.
\end{itemize}
%
Dabei sollte unbedingt darauf geachtet werden, dass die Höhe der einzelnen
Elemente im Text nicht zu groß wird.

\paragraph{Häufiger Fehler:}
Im Fließtext wird bei einfachen Variablen oft auf die Verwendung der richtigen,
mathematischen Zeichen vergessen, wie etwa in "X-Achse" anstelle von "$X$-Achse"
(\verb!$X$-Achse!).

\paragraph{Zeilenumbrüche:}
Bei längeren mathematischen Elementen im Fließtext sind Probleme mit
Zeilenumbrüchen vorprogrammiert. In der Regel ermöglicht \latex nur am "="
einen Zeilenumbruch, an anderer Stelle kann man Umbrüche mit
\texttt{{\bs}allowbreak} ermöglichen. Hier ein kleines Beispiel:
%
\begin{itemize}
	\item[a)] Einen einfachen Zeilenvektor definiert man beispielsweise in der
	Form $\boldsymbol{x} = (x_0, x_1, \ldots, x_{n-1})$.
	\item[b)] Einen einfachen Zeilenvektor definiert man beispielsweise in der
	Form $\boldsymbol{x} = (x_0,\allowbreak x_1,\allowbreak\ldots,\allowbreak
	x_{n-1})$.
\end{itemize}
%
Die Zeile in a) sollte über den Seitenrand hinauslaufen, b) hingegen enthält
\texttt{{\bs}allowbreak} an mehreren Stellen und sollte daher sauber umbrechen.


\section{Freigestellte Ausdrücke}

Freigestellte mathematische Ausdrücke können in \latex\ durch \verb!\[!
\ldots \verb!\]! erzeugt werden. Das Ergebnis wird zentriert, erhält jedoch
keine Nummerierung. So ist \zB\ \[y = 4 x^2\] das Ergebnis von
\verb!\[y = 4 x^2\]!.


\subsection{Einfache Gleichungen}

Meistens wird in solchen Fällen jedoch die \texttt{equation}-Umgebung zur
Herstellung nummerierter Gleichungen verwendet, auf die im Text jederzeit
verwiesen werden kann. Zum Beispiel erzeugt
%
\begin{LaTeXCode}[numbers=none]
\begin{equation}
	f(k) = \frac{1}{N} \sum_{i=0}^{k-1} i^2 .
	\label{eq:MyFirstEquation}
\end{equation}
\end{LaTeXCode}
%
die Gleichung
%
\begin{equation}
	f(k) = \frac{1}{N} \sum_{i=0}^{k-1} i^2 .
	\label{eq:MyFirstEquation}
\end{equation}
%
Mit \verb!\ref{eq:MyFirstEquation}! erhält man wie üblich die Nummer
(\ref{eq:MyFirstEquation}) dieser Gleichung (siehe dazu auch Abschn.\
\ref{sec:VerweiseAufGleichungen}). Dieselbe Gleichung \emph{ohne}
Nummerierung kann übrigens mit der \texttt{equation*}-Umgebung erzeugt werden.
%
\begin{center}
	\setlength{\fboxrule}{0.2mm}
	\setlength{\fboxsep}{2mm}
	\fbox{%
		\begin{minipage}{0.9\textwidth}
		Man beachte, dass \textbf{Gleichungen} inhaltlich ein \textbf{Teil
		des Texts} sind und daher neben der sprachlichen
		\textbf{Überleitung} auch die \textbf{Interpunktion} (wie in Gl.\
		\ref{eq:MyFirstEquation} gezeigt) beachtet werden muss. Bei
		Unsicherheiten sollte man sich passende Beispiele in einem
		guten Mathematik\-buch ansehen.
		\end{minipage}}
\end{center}
%
Für Interessierte findet sich mehr zum Thema Mathematik und Prosa in
\cite{Mermin1989} und \cite{Higham2020}.

\subsection{Mehrzeilige Gleichungen}

Für mehrzeilige Gleichungen bietet \latex\ die \verb!eqnarray!-Umgebung, die
allerdings etwas eigenwillige Zwischenräume erzeugt. Es empfiehlt sich, dafür
gleich auf die erweiterten Möglichkeiten des \texttt{amsmath}-Pakets%
\footnote{American Mathematical Society (AMS). \texttt{amsmath} ist Teil der
\latex\ Standardinstallation und wird von \texttt{hgb.sty} bereits importiert.}
\cite{Mittelbach2022} zurückzugreifen. Hier ein Beispiel mit zwei am $=$
Zeichen ausgerichteten Gleichungen,
%
\begin{align}
	f_1 (x,y) &= \frac{1}{1-x} + y , \label{eq:f1} \\
	f_2 (x,y) &= \frac{1}{1+y} - x , \label{eq:f2}
\end{align}
%
erzeugt mit der \texttt{align}-Umgebung aus dem \texttt{amsmath}-Paket:
%
\begin{LaTeXCode}[numbers=none]
\begin{align}
  f_1 (x,y) &= \frac{1}{1-x} + y , \label{eq:f1} \\
  f_2 (x,y) &= \frac{1}{1+y} - x , \label{eq:f2}
\end{align}
\end{LaTeXCode}


\subsection{Fallunterscheidungen}

Mit der \texttt{cases}-Umgebung aus \texttt{amsmath} sind Fallunterscheidungen,
\ua\ innerhalb von Funktionsdefinitionen, sehr einfach zu bewerkstelligen.
Beispielsweise wurde die rekursive Definition
%
\begin{equation}
	f(i) =
	\begin{cases}
		0             & \text{für $i = 0$,}\\
		f(i-1) + f(i) & \text{für $i > 0$.}
	\end{cases}
\end{equation}%
%
mit folgenden Anweisungen erzeugt:
%
\begin{LaTeXCode}[numbers=none]
\begin{equation}
	f(i) =
	\begin{cases}
	  0             & \text{für $i = 0$,}\\
	  f(i-1) + f(i) & \text{für $i > 0$.}
	\end{cases}
\end{equation}
\end{LaTeXCode}
%
Man beachte dabei die Verwendung des sehr praktischen \verb!\text{..}!-Makros,
mit dem im Mathematik-Modus gewöhnlicher Text eingefügt werden kann, sowie
wiederum die Interpunktion innerhalb der Gleichung.


\subsection{Gleichungen mit Matrizen}

Auch hier bietet \texttt{amsmath} einige Vorteile gegenüber der Verwendung
der \latex\ Standardkonstrukte. Dazu ein einfaches Beispiel für die
Verwendung der \texttt{pmatrix}-Umgebung für Vektoren und Matrizen,
%
\begin{equation}
	\begin{pmatrix}
		x' \\ y'
	\end{pmatrix}
	=
	\begin{pmatrix}
		\cos \phi & -\sin \phi           \\
		\sin \phi & \phantom{-}\cos \phi
	\end{pmatrix}
	\cdot
	\begin{pmatrix}
		x \\ y
	\end{pmatrix} ,
\end{equation}
%
das mit den folgenden Anweisungen erzeugt wurde:
%
\begin{LaTeXCode}
\begin{equation}
	\begin{pmatrix} 
			x' \\ 
			y' 
	\end{pmatrix}
	= 
	\begin{pmatrix}
		  \cos \phi &           -\sin \phi \\
		  \sin \phi & \phantom{-}\cos \phi /+ \label{lin:phantom} +/
	\end{pmatrix} 
	\cdot
	\begin{pmatrix} 
			x \\ 
			y 
	\end{pmatrix} ,
\end{equation}
\end{LaTeXCode}
%
Ein nützliches Detail darin ist das \tex-Makro \verb!\phantom{..}! (in Zeile
\ref{lin:phantom}), das sein Argument unsichtbar einfügt und hier als
Platzhalter für das darüberliegende Minuszeichen verwendet wird. Alternativ
zu \texttt{pmatrix} kann mit der \texttt{bmatrix}-Umgebung Matrizen und
Vektoren mit eckigen Klammern erzeugt werden. Zahlreiche weitere
mathematische Konstrukte des \texttt{amsmath}-Pakets sind in
\cite{Mittelbach2022} beschrieben.


\subsection{Verweise auf Gleichungen}
\label{sec:VerweiseAufGleichungen}

Beim Verweis auf nummerierte Formeln und Gleichungen genügt grundsätzlich die
Angabe der entsprechenden Nummer in runden Klammern, \zB\
\begin{center}
	"\ldots\ wie aus (\ref{eq:f1}) abgeleitet werden kann \ldots"
\end{center}
%
Um Missverständnisse zu vermeiden, sollte aber -- \va\ in Texten mit nur
wenigen mathematischen Elementen -- "Gleichung \ref{eq:f1}", "Gl
.~\ref{eq:f1}" oder "Gl.~(\ref{eq:f1})" geschrieben werden (natürlich
konsistent).
%
\begin{center}
	\setlength{\fboxrule}{0.2mm}
	\setlength{\fboxsep}{2mm}
	\fbox{%
		\begin{minipage}{0.9\textwidth}
			\textbf{Achtung:} Vorwärtsverweise auf (im Text weiter hinten
			liegende) Gleichungen sind \textbf{äußerst ungewöhnlich}
			und sollten vermieden werden! Glaubt man dennoch so etwas zu
			benötigen, dann wurde meistens ein Fehler in der Anordnung gemacht.
		\end{minipage}}
\end{center}


\section{Spezielle Symbole}

Für einen Großteil der mathematischen Symbole werden spezielle Makros
benötigt. Im Folgenden werden einige der gebräuchlichsten aufgelistet.

\subsection{Zahlenmengen}

Einige häufig verwendete Symbole sind leider im ursprünglichen mathematischen
Zeichensatz von \latex nicht enthalten, \zB die Symbole für die reellen und
natürlichen Zahlen. Im \texttt{hagenberg-thesis}-Paket sind diese Symbole als
Makros \verb!\R!, \verb!\Z!, \verb!\N!, \verb!\Cpx!, \verb!\Q! ($\R, \Z, \N,
\Cpx, \Q$) mithilfe der \emph{AMS Blackboard Fonts} definiert, \zB:
%
\begin{center}
	$x \in \R$ , $k \in \N_0$, $z = (a + \mathrm{i} \cdot b) \in \Cpx$.
\end{center}


\subsection{Operatoren}

In \latex\ sind Dutzende von mathematischen Operatoren für spezielle
Anwendungen definiert. Am häufigsten werden natürlich die arithmetischen
Operatoren $+$, $-$, $\cdot$ und $/$ benötigt. Ein dabei oft beobachteter
Fehler (der wohl aus der Programmierpraxis resultiert) ist die Verwendung von
$*$ für die einfache Multiplikation -- richtig ist $\cdot$ (\verb!\cdot!).%
\footnote{Das Zeichen $*$ ist üblicherweise für den \emph{Faltungsoperator}
vorgesehen.}
%
Für Angaben wie \zB\ "ein Feld mit $25 \times 70$ Metern" (aber auch fast
\emph{nur} dafür) wird sinnvollerweise der $\times$ (\verb!\times!) Operator
und \emph{nicht} einfach das Textzeichen~"x" verwendet!


\subsection{Variable (Symbole) mit mehreren Zeichen}

Vor allem bei der mathematischen Spezifikation von Algorithmen und Programmen
ist es häufig notwendig, Symbole (Variablennamen) mit mehr als einem Zeichen
zu verwenden, \zB
%
	\[Scalefactor\leftarrow Scalefactor^2 \cdot 1.5 \; ,\]
%
\textbf{fälschlicherweise} erzeugt durch
%
\begin{quote}
	\verb!$Scalefactor \leftarrow Scalefactor^2! \verb!\cdot 1.5$!.
\end{quote}
%
Dabei interpretiert \latex allerdings die Zeichenkette "Scalefactor" als 11
einzelne, aufeinanderfolgende Symbole $S$, $c$, $a$, $l$, $e$, \ldots und
setzt dazwischen entsprechende Abstände. \textbf{Richtig} ist, diese
Buchstaben mit \verb!\mathit{..}! zu \emph{einem} Symbol zusammenzufassen.
Der Unterschied ist in diesem Fall deutlich sichtbar:
%
\begin{center}
	\setlength{\tabcolsep}{4pt}
	\begin{tabular}{llll}
		\text{Falsch:}  & $Scalefactor^2$          & $\leftarrow$ &
		\verb!$Scalefactor^2$!          \\
		\text{Richtig:} & $\mathit{Scalefactor}^2$ & $\leftarrow$ &
		\verb!$\mathit{Scalefactor}^2$!
	\end{tabular}
\end{center}
%
Grundsätzlich sollten derart lange Symbolnamen aber ohnehin vermieden und
stattdessen möglichst kurze (gängige) Symbole verwendet werden (\zB\
Brennweite $f = 50 \, \mathrm{mm}$ statt $\mathit{Brennweite} = 50 \,
\mathrm{mm}$).

\subsection{Funktionen}

Während Symbole für Variablen traditionell (und in \latex\ automatisch)
\emph{italic} gesetzt werden, wird für die Namen von Funktionen und
Operatoren üblicherweise \emph{roman} als Schrifttyp verwendet, wie \zB in
%
\begin{center}
	\begin{tabular}{lcl}
		$\sin \theta = \sin(\theta + 2 \pi)$ &
		$\leftarrow$ & \verb!$\sin \theta = \sin(\theta + 2 \pi)$! \\
	\end{tabular}
\end{center}
%
Das ist bei den bereits vordefinierten Standardfunktionen (wie \verb!\sin!,
\verb!\cos!, \verb!\tan!, \verb!\log!, \verb!\max! \uva) automatisch der Fall.
Diese Konvention sollte auch bei selbstdefinierten Funktionen befolgt werden,
wie etwa in
%
\begin{center}
	\begin{tabular}{lcl}
	$\mathrm{dist}(A,B) := |A-B|$ & $\leftarrow$ & 
	\verb!$\mathrm{dist}(A,B) := |A-B|$! \\
	\end{tabular}
\end{center}


\subsection{Maßeinheiten und Währungen}

Bei der Angabe von Maßeinheiten wird üblicherweise Normalschrift (keine
Italics) verwendet, \zB:
%
\begin{quote}
	Die Höchstgeschwindigkeit der \textit{Bell XS-1} beträgt 345~m/s bei
	einem Startgewicht von 15~t. Der Prototyp kostete über 25.000.000 US\$,
	also ca.\ 19.200.000 \euro\ nach heutiger Umrechnung.
\end{quote}
%
Der Abstand zwischen der Zahl und der Maßeinheit ist dabei gewollt. Das
\$-Zeichen wird mit \verb!\$! und das Euro-Symbol (\euro) mit dem Makro
\verb!\euro! erzeugt.%
\footnote{Das \euro\ Zeichen ist nicht im ursprünglichen \latex-Zeichensatz
enthalten sondern wird mit dem \texttt{eurosym}-Paket erzeugt.}


\subsection{Kommas in Dezimalzahlen (Mathematik-Modus)}

\latex\ setzt im Mathematik-Modus (also innerhalb von \verb!$! \ldots \verb!$!,
\verb!\[! \ldots \verb!\]! oder in Gleichungen) nach dem angloamerikanischen
Stil in Dezimalzahlen grundsätzlich den \emph{Punkt} (\verb!.!) als Trennsymbol
voraus. So wird etwa mit \verb!$3.141$! normalerweise die Ausgabe "3.141"
erzeugt. Um das in Europa übliche Komma in Dezimalzahlen zu verwenden, genügt
es \emph{nicht}, einfach \verb!.! durch \verb!,! zu ersetzen. Das Komma wird
in diesem Fall als \textbf{Satzzeichen} interpretiert und sieht dann so aus:
%
\begin{quote}
	\verb!$3,141$! $\quad \rightarrow \quad 3,141$
\end{quote}
%
(man beachte den Leerraum nach dem Komma). Dieses Verhalten lässt sich in
\latex\ zwar global umdefinieren, was aber wiederum zu einer Reihe
unangenehmer Nebeneffekte führt. Eine einfache (wenn auch nicht sehr
elegante) Lösung ist, Kommazahlen im Mathematik-Modus so zu schreiben:
%
\begin{quote}
	\verb!$3{,}141$! $\quad \rightarrow \quad 3{,}141$
\end{quote}


\subsection{Mathematische Werkzeuge}

Für die Erstellung komplizierter Gleichungen ist es mitunter hilfreich, auf
spezielle Software zurückzugreifen. Unter anderem können aus dem Microsoft
\emph{Equation Editor} und aus {\em Mathematica} auf relativ einfache Weise
\latex-An\-wei\-sun\-gen für mathematische Gleichungen exportiert und direkt
(mit etwas manueller Nacharbeit) in das eigene \latex-Dokument übernommen
werden.


\section{Algorithmen}

Die algorithmische Darstellung ist ein wichtiges Mittel zur präzisen
Beschreibung von Berechnungsabläufen. Durch die Verwendung von
\emph{mathematischer Notation} (Symbolen und Operatoren) einerseits und den
aus der Programmierung gewohnten \emph{Ablaufstrukturen} (Entscheidungen,
Schleifen, Prozeduren \etc) sind Algorithmen ein bewährtes Bindeglied
zwischen der mathematischen Formulierung und dem zugehörigen Programmcode.

Ein wesentlicher Aspekt der algorithmischen Beschreibung -- die idealerweise
der Implementierung zumindest strukturell möglichst ähnlich sein sollte --
ist die weitgehende \emph{Unabhängigkeit} von einer spezifischen
Programmiersprache. Dadurch ergibt sich eine bessere Lesbarkeit, breitere
Anwendbarkeit und erhöhte Nachhaltigkeit (möglicherweise über die Lebensdauer
einer Programmiersprache hinaus). Bei der Formulierung von Algorithmen sollte
man \ua\ folgendes beachten:%
\footnote{Siehe auch
\url{http://mirrors.ctan.org/macros/latex/contrib/algorithms/algorithms.pdf}
(Abschnitt~7).}
%
\begin{itemize}
	\item
	Verwende in Algorithmen die gleichen kurze Symbole (wie $a, i, x, S,
	\alpha \ldots$), wie man sie auch in mathematischen Definitionen und
	Gleichungen verwendet.
	\item
	Verwende nach Möglichkeit mathematische Operatoren, wie \zB\
	$=$ (\verb!$=$!) statt \texttt{==},
	$\leq$ (\verb!$\leq$!) statt \texttt{<=},
	$\cdot$ (\verb!$\cdot$!) statt \texttt{*},
	$\wedge$ (\verb!$\wedge$!) statt \texttt{\&\&},
	\usw
	\item
	Verwende keine Elemente oder Syntax einer spezifischen Programmiersprache
	(so ist etwa ein "\texttt{;}" am Ende einer Anweisung unnötig).
	\item
	Wenn ein Algorithmus für eine Seite zu lang wird, überlege, wie man ihn
	sinnvoll auf kleinere Module aufteilen kann (meist ist dann auch die
	zugehörige Programmstruktur nicht optimal).
\end{itemize}


Für die Notation von Algorithmen in mathematischer Form oder auch für
Pseudo\-code ist in \latex selbst keine spezielle Unterstützung vorgesehen.
Dazu gibt es jedoch eine Reihe von brauchbaren \latex-Paketen, \ua\
\texttt{algorithmicx}, das wegen seiner einfachen Syntax auch hier verwendet
wird, allerdings in der verbesserten Version \texttt{algpseudocodex}.%
\footnote{Die Datei \nolinkurl{hgbalgo.sty} des
\texttt{hagenberg-thesis}-Pakets erweitert die Pakete \texttt{algorithmicx}
\bzw\ \texttt{algpseudocodex} (s.\ \url{https://ctan.org/pkg/algpseudocodex})
durch verbesserte Einrückung, Farben \etc}
%
Das Beispiel in Alg.~\ref{alg:Example} wurde mit der Float-Umgebung
\texttt{algorithm} und dem \texttt{algpseudocodex}-Paket erstellt (s.\
Quellcode in Prog.\ \ref{prog:AlgExample}). Zur besseren Lesbarkeit werden
hier vertikale Einrückungslinien verwendet (\texttt{indLines=true}) und auf
das Schlüsselwort \texttt{end} am Ende von Blöcken wird verzichtet
(\texttt{noEnd=true}).

%%--------------------------------------------------------------------

\begin{algorithm}
\caption{Beispiel für einen mit dem Paket \texttt{algpseudocodex} gesetzten
Algorithmus zur bikubischen Interpolation in 2D (aus \cite{BurgerBurge2022}).
Die in den Zeilen \ref{alg:wcub1} und \ref{alg:wcub2} verwendete Funktion
$\Call{Cubic1D}{x}$ berechnet die Gewichtung des Werts für die eindimensionale
Position $x$.}
\label{alg:Example}

\begin{algorithmic}[1]     % [1] = all lines are numbered
\Function{BicubicInterpolation}{$I, x, y$} \Comment{two-dimensional interpolation}
	\Input{$I$, original image; $x,y \in \R$, continuous position.}
	\Returns{the interpolated pixel value at position $(x,y)$.\algsmallskip}
	
	\State $\mathit{val} \gets 0$
	
	\For{$j \gets 0, \ldots, 3$} \Comment{iterate over 4 lines}
		\State $v \gets \lfloor y \rfloor - 1 + j$
		\State $p \gets 0$
		\For{$i \gets 0, \ldots, 3$} \Comment{iterate over 4 columns}
			\State $u \gets \lfloor x \rfloor - 1 + i$
			\State $p \gets p + I(u,v) \cdot \Call{Cubic1D}{x - u}$	\label{alg:wcub1}
		\EndFor

		\StateNN[2]{Sometimes it is useful to insert a longer, \emph{unnumbered}
		statement extending over multiple lines with proper indentation. This
		can be done with the (non-standard) command
		\texttt{{\bs}StateNN[]\{..\}}. For long \emph{numbered} (multi-line)
		statements use the standard \texttt{{\bs}State} command.}
		
		\State $\mathit{val} \gets \mathit{val} + p \cdot \Call{Cubic1D}{y - v}$
				\label{alg:wcub2}
	\EndFor
	\State\Return $\mathit{val}$
\EndFunction

\medskip	% \medskip can be used here, because we are in vertical mode
\hrule

\Function{Cubic1D}{$x$} \Comment{piecewise cubic polynomial (1D)}
	\State $z \gets 0$
		\If{$|x| < 1$}
			\State $z \gets |x|^3 - 2 \cdot |x|^2 + 1$
		\ElsIf{$|x| < 2$}
			\State $z \gets -|x|^3 + 5 \cdot |x|^2 - 8 \cdot |x| + 4$
		\EndIf
		\State\Return{$z$}
\EndFunction

\end{algorithmic}
\end{algorithm}

%%--------------------------------------------------------------------

\begin{program}
\caption{Quellcode zu Algorithmus \ref{alg:Example}. Wie ersichtlich, können
hier auch beliebig Leerzeilen verwendet werden, was die Lesbarkeit deutlich
verbessert.}
\label{prog:AlgExample}
\begin{LaTeXCode}
\begin{algorithm}
\caption{Beispiel für einen mit dem Paket \texttt{algpseudocodex}
gesetzten Algorithmus zur bikubischen Interpolation in 2D (aus
\cite{BurgerBurge2022}). Die in den Zeilen \ref{alg:wcub1} und
\ref{alg:wcub2} verwendete Funktion $\Call{Cubic1D}{x}$ berechnet 
die Gewichtung des Werts für die eindimensionale Position $x$.}
\label{alg:Example}

\begin{algorithmic}[1]     % [1] = all lines are numbered
\Function{BicubicInterpolation}{$I, x, y$} \Comment{two-dimensional interpolation}
	\Input{$I$, original image; $x,y \in \R$, continuous position.}
	\Returns{the interpolated pixel value at position $(x,y)$.\algsmallskip}
	
	\State $\mathit{val} \gets 0$
	
	\For{$j \gets 0, \ldots, 3$} \Comment{iterate over 4 lines}
		\State $v \gets \lfloor y \rfloor - 1 + j$
		\State $p \gets 0$
		\For{$i \gets 0, \ldots, 3$} \Comment{iterate over 4 columns}
			\State $u \gets \lfloor x \rfloor - 1 + i$
			\State $p \gets p + I(u,v) \cdot \Call{Cubic1D}{x - u}$	\label{alg:wcub1}
		\EndFor		
		
		\StateNN[2]{Sometimes it is useful to insert a longer, ...}
		
		\State $\mathit{val} \gets \mathit{val} + p \cdot \Call{Cubic1D}{y - v}$
				\label{alg:wcub2}
	\EndFor
	\State\Return $\mathit{val}$
\EndFunction

\medskip	% \medskip can be used here, because we are in vertical mode
\hrule

\Function{Cubic1D}{$x$} \Comment{piecewise cubic polynomial (1D)}
	\State $z \gets 0$
		\If{$|x| < 1$}
			\State $z \gets |x|^3 - 2 \cdot |x|^2 + 1$
		\ElsIf{$|x| < 2$}
			\State $z \gets -|x|^3 + 5 \cdot |x|^2 - 8 \cdot |x| + 4$
		\EndIf
		\State\Return{$z$}
\EndFunction

\end{algorithmic}
\end{algorithm}
\end{LaTeXCode}
\end{program}
