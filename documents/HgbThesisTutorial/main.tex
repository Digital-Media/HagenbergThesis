% !TeX document-id = {d6373e26-6fc2-494b-a9ec-b221ee39738f}
%%% File encoding: UTF-8
%%% äöüÄÖÜß  <-- keine deutschen Umlaute hier? UTF-faehigen Editor verwenden!

%%% Magic Comments zum Setzen der korrekten Parameter in kompatiblen IDEs
% !TeX encoding = utf8
% !TeX program = pdflatex 
% !TeX spellcheck = de_DE
% !BIB program = biber

\documentclass[master,german,smartquotes]{hgbthesis}
% Zulässige Optionen in [..]: 
%    Typ der Arbeit: 'diploma', 'master' (default), 'bachelor', 'internship' 
%    Hauptsprache: 'german' (default), 'english'
%    Turn on smart quote handling: 'smartquotes'
%    Use APA bibliography style: 'apa'
%%%----------------------------------------------------------

\RequirePackage[utf8]{inputenc}		% bei der Verw. von lualatex oder xelatex entfernen!

\graphicspath{{images/}}    % Verzeichnis mit Bildern und Grafiken
\logofile{logo}				% Logo-Datei = images/logo.pdf (\logofile{}, wenn kein Logo gewünscht)
\bibliography{references}  	% Biblatex-Literaturdatei (references.bib)

\usepackage{acro}
\acsetup{
	first-style=long-short, % long-short|short-long|short|long|footnote
	subsequent-style=short, % long-short|short-long|short|long|footnote
	single=false, % set to "true" if an acronym that has only been used once should not be part of the list of acronyms
	make-links=true, % set to "true" to link acronyms with their entry in the list of acronyms
	list/display=used, % set to "all" to print all defined acronyms, no matter if they are referenced or not
	list/template=toc
	%list/heading=chapter
}
% Acronyms (tagged "acro")

\DeclareAcronym{jpg}{
	short=JPEG,
	long=Joint Photographic Experts Group,
	alt=JPG,
	sort=jpeg,
	tag=acro
}

\DeclareAcronym{ufo}{
	short=UFO,
	long=unidentified flying object,
	foreign=unbekanntes Flugobjekt,
	foreign-plural-form=unbekannte Flugobjekte,
	foreign-babel=ngerman,
	long-indefinite=an,
	tag=acro
}

% Nomenclature (tagged "nomencl")

\DeclareAcronym{pi}{
	short=\ensuremath{\pi},
	long=The ratio of a circle's circumference to its diameter,
	sort=pi,
	first-style=short,
	tag=nomencl
}

\DeclareAcronym{c}{
	short=\ensuremath{c},
	long=The speed of light in vacuum,
	sort=c,
	first-style=short,
	tag=nomencl
}


%%%----------------------------------------------------------
% Angaben für die Titelei (Titelseite, Erklärung etc.)
%%%----------------------------------------------------------

%%% Einträge für ALLE Arbeiten: -----------------------------
\title{Partielle Lösungen zur allgemeinen Problematik}
\author{Alex A.\ Schlaumeier}
\programname{Universal Computing}

% \programtype{Fachhochschul-Bachelorstudiengang}		% select/edit
\programtype{Fachhochschul-Masterstudiengang}

\placeofstudy{Hagenberg}
\dateofsubmission{2021}{07}{15}	% {YYYY}{MM}{DD}

\advisor{Alois B.~Treuer, Päd.\ Phil.}	% optional

%\strictlicense		%%% restrictive license instead of Creative Commons (discouraged!)

%%%----------------------------------------------------------
\begin{document}
%%%----------------------------------------------------------

%%%----------------------------------------------------------
\frontmatter                    % Titelei (röm. Seitenzahlen)
%%%----------------------------------------------------------

\maketitle
\tableofcontents

\chapter{Vorwort} 	% engl. Preface


Dies ist \textbf{Version \hgbthesisDate} der \latex-Dokumentenvorlage für 
verschiedene Abschlussarbeiten an der Fakultät für Informatik, Kommunikation
und Medien der FH Oberösterreich in Hagenberg, die mittlerweile auch 
an anderen Hochschulen im In- und Ausland gerne verwendet wird.

Das Dokument entstand ursprünglich auf Anfragen von Studierenden,
nachdem im Studienjahr 2000/01 erstmals ein offizieller
\latex-Grundkurs im Studiengang Medientechnik und -design an der
FH Hagenberg angeboten wurde. Eigentlich war die Idee, die bereits
bestehende \emph{Word}-Vorlage für Diplomarbeiten "`einfach"' in
\latex\ zu übersetzen und dazu eventuell einige spezielle
Ergänzungen einzubauen. Das erwies sich rasch als wenig
zielführend, da \latex, \va was den Umgang mit Literatur und
Grafiken anbelangt, doch eine wesentlich andere Arbeitsweise
verlangt. Das Ergebnis ist -- von Grund auf neu geschrieben und
wesentlich umfangreicher als das vorherige Dokument --
letztendlich eine Anleitung für das Schreiben mit \latex, ergänzt
mit einigen speziellen (mittlerweile entfernten) Hinweisen für \emph{Word}-Benutzer.
Technische Details zur aktuellen Version finden sich in Anhang \ref{app:TechnischeInfos}.

Während dieses Dokument anfangs ausschließlich für die Erstellung
von Diplomarbeiten gedacht war, sind nunmehr auch  
\emph{Masterarbeiten}, \emph{Bachelor\-arbeiten} und \emph{Praktikumsberichte} 
abgedeckt, wobei die Unterschiede bewusst gering gehalten wurden.

Bei der Zusammenstellung dieser Vorlage wurde versucht, mit der
Basisfunktionalität von \latex das Auslangen zu finden und -- soweit möglich --
auf zusätzliche Pakete zu verzichten. Das ist nur zum Teil gelungen;
tat\-säch\-lich ist eine Reihe von ergänzenden "`Paketen"' notwendig, wobei jedoch
nur auf gängige Erweiterungen zurückgegriffen wurde.
Selbstverständlich gibt es darüber hinaus eine Vielzahl weiterer Pakete,
die für weitere Verbesserungen und Finessen nützlich sein können. Damit kann
sich aber jeder selbst beschäftigen, sobald das notwendige Selbstvertrauen und
genügend Zeit zum Experimentieren vorhanden sind.
Eine Vielzahl von Details und Tricks sind zwar in diesem Dokument nicht explizit
angeführt, können aber im zugehörigen Quelltext jederzeit ausgeforscht
werden.

Zahlreiche KollegInnen haben durch sorgfältiges Korrekturlesen und
konstruktive Verbesserungsvorschläge wertvolle Unterstützung
geliefert. Speziell bedanken möchte ich mich bei Heinz Dobler für
die konsequente Verbesserung meines "`Computer Slangs"', bei
Elisabeth Mitterbauer für das bewährte orthographische Auge und
bei Wolfgang Hochleitner für die Tests unter Mac~OS.

Die Verwendung dieser Vorlage ist jedermann freigestellt und an
keinerlei Erwähnung gebunden. Allerdings -- wer sie als Grundlage
seiner eigenen Arbeit verwenden möchte, sollte nicht einfach
("`ung'schaut"') darauf los werken, sondern zumindest die
wichtigsten Teile des Dokuments \emph{lesen} und nach Möglichkeit
auch beherzigen. Die Erfahrung zeigt, dass dies die Qualität der
Ergebnisse deutlich zu steigern vermag.

Der Quelltext zu diesem Dokument sowie das zugehörige
\latex-Paket sind in der jeweils aktuellen Version online
verfügbar unter
%
\begin{itemize}
\item[]\url{https://github.com/Digital-Media/HagenbergThesis}.
\end{itemize}
%
Trotz großer Mühe enthält dieses Dokument zweifellos Fehler und Unzulänglichkeiten
-- Kommentare, Verbesserungsvorschläge und passende Ergänzungen
sind daher stets willkommen, am einfachsten per E-Mail direkt an mich:
\begin{itemize}
\item[]%

Dr.\ Wilhelm Burger, Department für Digitale Medien,\newline
Fachhochschule Oberösterreich, Campus Hagenberg (Österreich)\newline
\nolinkurl{wilhelm.burger@fh-hagenberg.at}
\end{itemize}

\noindent
Übrigens, hier im Vorwort (das bei Diplom- und Masterarbeiten üblich, bei Bachelorarbeiten 
aber entbehrlich ist) kann kurz auf die Entstehung des Dokuments eingegangen werden.
Hier ist auch der Platz für allfällige Danksagungen (\zB an den Betreuer, 
den Begutachter, die Familie, den Hund, \ldots), Widmungen und philosophische 
Anmerkungen. Das sollte allerdings auch nicht übertrieben werden und sich auf 
einen Umfang von maximal zwei Seiten beschränken.




 				% Vorwort ist optional
\chapter{Kurzfassung}

\begin{german} %switch to German language rules
	Dies sollte eine maximal 1-seitige Zusammenfassung Ihrer Arbeit in deutscher
	Sprache sein.
	%here goes the rest of the Kurzfassung...
\end{german}

The German "Kurzfassung" should contain the same content as the English abstract.
Therefore, try to translate the abstract precisely but not word for word. When
translating, remember that certain idioms from English have no counterpart in
German or must be formulated differently. Also, word order in German is very
different from English (more on this in Section \ref{sec:german}). Without
knowledge of the German language, it is acceptable to resort to translators.
Nevertheless, hiring a skillful person for proofreading is recommended
even with the highest confidence in one's German knowledge.

The correct translation for "diploma thesis" is \emph{Diplomarbeit}, a "master
thesis" is called \emph{Masterarbeit}. For "bachelor's thesis",
\emph{Bachelorarbeit} is the appropriate translation.

By the way, for this section, the \emph{language setting} in \latex\ should be
switched from English to German to get the correct form of hyphenation. However,
the correct quotation marks must be set manually (see Sections
\ref{sec:language-switching} and \ref{sec:quotation-marks}).
		
\chapter{Abstract}

Here goes an abstract of the work, with a maximum of 1 page. Unlike other
chapters, the abstract is usually not divided into sections and subsections.
Footnotes are also not used here.

By the way, abstracts are often included in literature databases with the author
and title of the work. It is, therefore, essential to ensure that the
information in the abstract is coherent and complete in itself (\ie, without
other parts of the work). In particular, \emph{no literature references} are
typically used at this point (as is the case also in the \emph{title} of the
thesis and the German \emph{Kurzfassung})!
If such is needed---for example, because the paper is a further development of a
particular, earlier paper---then \emph{full} references are necessary for the
abstract itself, e.g., [\textsc{Zobel} J.: \textit{Writing for Computer Science
-- The Art of Effective Commu\-nica\-tion}. Springer, Singapore, 1997].

It should also be noted that special characters or list items are usually lost
when records are added to a database. The same applies, of course, to the German
\emph{Kurzfassung}.

In terms of content, the abstract should not be a list of the individual
chapters (the introduction chapter is intended for this purpose). However, it
should provide the reader with a compact, substantial overview of the work.
Therefore, the structure used here is necessarily different from that used in
the introduction.
			

%%%----------------------------------------------------------
\mainmatter          % Hauptteil (ab hier arab. Seitenzahlen)
%%%----------------------------------------------------------

\chapter{Einleitung}
\label{cha:Einleitung}

\section{Zielsetzung}
Dieses Dokument ist als vorwiegend technische Starthilfe für das
Erstellen einer Masterarbeit (oder Bachelorarbeit) mit \latex
gedacht und ist die Weiterentwicklung einer früheren
Vorlage\footnote{Nicht mehr verfügbar.} für das Arbeiten mit
Microsoft \emph{Word}. Während ursprünglich daran gedacht war, die
bestehende Vorlage einfach in \latex zu übernehmen, wurde rasch
klar, dass allein aufgrund der großen Unterschiede zum Arbeiten
mit \emph{Word} ein gänzlich anderer Ansatz notwendig wurde. Dazu
kamen zahlreiche Erfahrungen mit Diplomarbeiten in den
nachfolgenden Jahren, die zu einigen zusätzlichen Hinweisen Anlass gaben.

Das vorliegende Dokument dient einem zweifachen Zweck: 
\emph{erstens} als Erläuterung und Anleitung, \emph{zweitens} als
direkter Ausgangspunkt für die eigene Arbeit. Angenommen wird,
dass der Leser bereits über elementare Kenntnisse im Umgang mit
\latex verfügt. In diesem Fall sollte -- eine einwandfreie
Installation der Software vorausgesetzt -- der Arbeit nichts mehr
im Wege stehen. Auch sonst ist der Start mit \latex\ nicht
schwierig, da viele hilfreiche Informationen auf den zugehörigen
Webseiten zu finden sind (s.\ Kap.~\ref{cha:ArbeitenMitLatex}).





\section{Warum {\latex}?}

Diplomarbeiten, Dissertationen und Bücher im
technisch-natur\-wissen\-schaft\-lichen Bereich werden
traditionell mithilfe des Textverarbeitungssystems \latex
\cite{Lamport1994, Lamport1995} gesetzt. Das hat gute Gründe, denn
\latex ist bzgl.\ der Qualität des Druckbilds, des Umgangs mit
mathematischen Elementen, Literaturverzeichnissen etc.\
unübertroffen und ist noch dazu frei verfügbar. Wer mit \latex
bereits vertraut ist, sollte es auch für die Abschlussarbeit
unbedingt in Betracht ziehen, aber auch für den Anfänger sollte
sich die zusätzliche Mühe am Ende durchaus lohnen.

Für den professionellen elektronischen Buchsatz wurde früher
häufig \emph{Adobe Framemaker} verwendet, allerdings ist diese
Software teuer und komplex. Eine modernere Alternative dazu ist
\emph{Adobe InDesign}, wobei allerdings die Erstellung
mathematischer Elemente und die Verwaltung von Literaturverweisen
zur Zeit nur rudimentär unterstützt werden.%
\footnote{Angeblich werden aber für den (sehr sauberen) Schriftsatz 
in \emph{InDesign} ähnliche Algorithmen wie in \latex\ verwendet.}

Microsoft \emph{Word} gilt im Unterschied zu \latex, 
\emph{Framemaker} und \emph{InDesign} übrigens nicht als professionelle
Textverarbeitungssoftware, obwohl es immer häufiger auch von
großen Verlagen verwendet wird.%
\footnote{Siehe auch \url{http://latex.tugraz.at/mythen.php}.}
Das Schriftbild in \emph{Word}
lässt -- zumindest für das geschulte Auge -- einiges zu wünschen
übrig und das Erstellen von Büchern und ähnlich großen Dokumenten
wird nur unzureichend unterstützt. Allerdings ist \emph{Word} sehr
verbreitet, flexibel und vielen Benutzern zumindest oberflächlich
vertraut, sodass das Erlernen eines speziellen Werkzeugs wie
\latex\ ausschließlich für das Erstellen einer Abschlussarbeit
manchen verständlicherweise zu mühevoll ist. Es sollte daher
niemandem übel genommen werden, wenn er/sie sich auch bei der Abschlussarbeit
auf \emph{Word} verlässt. Im Endeffekt lässt sich mit etwas
Sorgfalt (und ein paar Tricks) auch damit ein durchaus akzeptables
Ergebnis erzielen. 
Ansonsten sollten auch für \emph{Word}-Benutzer 
einige Teile dieses Dokuments von Interesse sein, insbesondere die
Abschnitte über Abbildungen und Tabellen
(Kap.~\ref{cha:Abbildungen}) und mathematische Elemente
(Kap.~\ref{cha:Mathematik}).


\section{Aufbau der Arbeit}

Hier am Ende des Einleitungskapitels (und nicht
etwa in der Kurzfassung) ist der richtige Platz, um die
inhaltliche Gliederung der nachfolgenden Arbeit zu beschreiben.
Hier sollte man darstellen, welche Teile (Kapitel) der Arbeit
welche Funktion haben und wie sie inhaltlich zusammenhängen. Auch
die Inhalte des \emph{Anhangs} -- sofern vorgesehen -- sollten hier
kurz beschrieben werden.

Zunächst sind in Kapitel \ref{cha:Abschlussarbeit} einige wichtige
Punkte zu Abschlussarbeiten im Allgemeinen zusammengefasst.
Kapitel \ref{cha:ArbeitenMitLatex} beschreibt die Idee und die
grundlegenden technischen Eigenschaften von \latex.
Kapitel \ref{cha:Abbildungen} widmet sich der Erstellung von Abbildungen
und Tabellen sowie der Einbindung von Quellcode.
Mathematische Elemente und Gleichungen sind das Thema in Kapitel \ref{cha:Mathematik} 
\usw
Anhang \ref{app:TechnischeInfos} enthält technische Details zu
dieser Vorlage, 
Anhang \ref{app:cdrom} enthält eine Auflistung von zugehörigen Materialien
auf einem beigelegten Speichermedium, und 
Anhang \ref{app:Fragebogen} zeigt ein Beispiel für die
Einbindung eines mehrseitigen PDF-Dokuments.







\chapter{Die Abschlussarbeit}
\label{cha:Abschlussarbeit}

Jede Abschlussarbeit%
\footnote{Die meisten der folgenden Bemerkungen gelten gleichsam für Bachelor-, Master- und Diplomarbeiten.} 
ist anders und dennoch sind sich gute
Arbeiten in ihrer Struktur meist sehr ähnlich, \va\ bei
technisch-natur\-wissen\-schaft\-lichen Themen. 

\section{Elemente der Abschlussarbeit}

Als Ausgangspunkt bewährt hat sich der folgende Grundaufbau, der natürlich 
vari\-iert und beliebig verfeinert werden kann:
%
\begin{enumerate}
\item \textbf{Einführung und Motivation}: Was ist die Problem- oder Aufgabenstellung und
warum sollte sich jemand dafür interessieren?
\item \textbf{Präzisierung des Themas}: Hier wird der aktuelle Stand der Technik
oder Wissenschaft ("`State-Of-The-Art"') beschrieben, es werden bestehende
Defizite oder offene Fragen aufgezeigt und daraus die
Stoßrichtung der eigenen Arbeit entwickelt.
\item \textbf{Eigener Ansatz}: Das ist natürlich der Kern der Arbeit. Hier
wird gezeigt, wie die vorher beschriebene Aufgabenstellung gelöst und --
häufig in Form eines Programms%
\footnote{\emph{Prototyp} ist in diesem Zusammenhang ein gerne benutzter Begriff, der im Deutschen
allerdings oft unrichtig dekliniert wird. Richtig ist: der \emph{Prototyp}, des \emph{Prototyps}, dem/den \emph{Protototyp} -- falsch hingegen \zB: des \emph{Prototyp\underline{en}}!
} --
realisiert wird, ergänzt durch illustrative Beispiele.
\item \textbf{Zusammenfassung}: Was wurde erreicht und welche Ziele sind
noch offen geblieben, wo könnte weiter gearbeitet werden?
\end{enumerate}
%
Natürlich ist auch ein gewisser dramaturgischer Aufbau der Arbeit
wichtig, wobei zu bedenken ist, dass der Leser in der Regel nur
wenig Zeit hat und -- anders als etwa bei einem Roman -- seine
Geduld nicht auf die lange Folter gespannt werden darf. Erklären
Sie bereits in der Einführung (und nicht erst im letzten Kapitel),
wie Sie an die Sache herangehen, welche Lösungen Sie vorschlagen
und wie erfolgreich Sie damit waren.

Übrigens, auch Fehler und Sackgassen dürfen (und sollten)
beschrieben werden; ihre Kenntnis hilft oft doppelte Experimente und
weitere Fehler zu vermeiden und ist damit sicher nützlicher als
jede Schönfärberei.
Und natürlich ist es auch nicht verboten, seine eigene Meinung 
in sachlicher Form zu äußern.


\section{Arbeiten in Englisch}
\label{sec:englisch}

Diese Vorlage ist zunächst darauf abgestellt, dass die
Abschlussarbeit in deutscher Sprache erstellt wird. Vor allem bei
Arbeiten, die in Zusammenarbeit mit größeren Firmen oder
internationalen Instituten entstehen, ist es häufig erwünscht,
dass die Abschlussarbeit zu besseren Nutzbarkeit in englischer
Sprache verfasst wird, und viele Hochschulen%
\footnote{Die FH Oberösterreich macht hier keine Ausnahme. 
Der Begriff "`Fachhochschule"' wird dabei entweder gar nicht
übersetzt oder -- wie im deutschsprachigen Raum mittlerweile üblich -- 
mit \emph{University of Applied Sciences}.
%Die offizielle englische Übersetzung von "`Medientechnik und -design"'
%ist übrigens \emph{Media Technology and Design}.
} 
lassen dies in
der Regel auch zu.

Beachtet sollte allerdings werden, dass das Schreiben dadurch nicht
einfacher wird, auch wenn einem Worte und Sätze im Englischen
scheinbar leichter "`aus der Feder"' fließen. Gerade im Bereich
der Informatik erscheint durch die Dominanz englischer
Fachausdrücke das Schreiben im Deutschen mühsam und das Ausweichen
ins Englische daher besonders attraktiv. Das ist jedoch
trügerisch, da die eigene Fertigkeit in der Fremdsprache
(trotz der meist langjährigen Schulbildung) häufig überschätzt wird.
Prägnanz und Klarheit gehen leicht verloren und bisweilen ist das
Resultat ein peinliches Gefasel ohne Zusammenhang und soliden
Inhalt. Sofern die eigenen Englischkenntnisse nicht wirklich gut sind, ist
es ratsam, zumindest die wichtigsten Teile der Arbeit zunächst in
Deutsch zu verfassen und erst nachträglich zu übersetzen. Besondere Vorsicht ist bei der Übersetzung von scheinbar
vertrauten Fachausdrücken angebracht. Zusätzlich ist es immer zu
empfehlen, die fertige Arbeit von einem "`native speaker"'
korrigieren zu lassen.



Technisch ist, außer der Spracheinstellung und den
unterschiedlichen Anführungszeichen (s.\
Abschn.~\ref{sec:anfuehrungszeichen}), für eine englische Arbeit
nicht viel zu ändern, allerdings sollte Folgendes beachtet werden:
%
\begin{itemize}
\item  Die Titelseite (mit der Bezeichnung "`Diplomarbeit"' oder "`Masterarbeit"') 
ist für die einzureichenden Exemplare jedenfalls in \emph{deutsch} zu halten,
auch wenn der Titel englisch ist. 
\item Ebenso muss neben dem
englischen \emph{Abstract} auch eine deutsche \emph{Kurzfassung}
enthalten sein. %
\item Akademische Titel von Personen haben im Englischen offenbar
weniger Bedeutung als im Deutschen und werden daher meist
weggelassen.
\end{itemize}

\chapter{Zum Arbeiten mit \latex}
\label{cha:ArbeitenMitLatex}


\chapter{Abbildungen, Tabellen, Quellcode}
\label{cha:Abbildungen}

\section{Allgemeines}

Abbildungen (\emph{figures}) und Tabellen (\emph{tables}) werden üblicherweise
zusammen mit einem nummerierten Titel (\emph{caption}) zentriert
angeordnet (siehe Abb.~\ref{fig:CocaCola}).
Im Text \emph{muss} es zu jeder Abbildung einen Verweis geben und die eigentliche Abbildung
sollte erst \emph{nach} dem ersten Verweis platziert werden.

\begin{figure}
\centering
\includegraphics[width=.75\textwidth]{cola-public-domain-photo-p} %{CS0031}
\caption{Coca-Cola Werbung 1940 \cite{CocaCola1940}.}
\label{fig:CocaCola}
\end{figure}



\section{\emph{Let Them Float!}}

Das Platzieren von Abbildungen und Tabellen gehört zu den
schwierigsten Aufgaben im Schriftsatz, weil diese meist viel Platz
benötigen und häufig nicht auf der aktuellen Seite im laufenden
Text untergebracht werden können. Diese Elemente müssen daher an
eine geeignete Stelle auf nachfolgenden Seiten verschoben werden,
was manuell sehr mühsam (jedoch in \emph{Word} beispielsweise unerlässlich) ist.

In \latex funktioniert das weitgehend automatisch, indem
Abbildungen, Tabellen und ähnliche als "`Floating Bodies"'
behandelt werden. Bei der Positionierung dieser Elemente wird
versucht, einerseits im Textfluss möglichst wenig Leer\-raum
entstehen zu lassen und andererseits die Abbildungen und Tabellen
nicht zu weit von der ursprünglichen Textstelle zu entfernen.

Der Gedanke, dass etwa Abbildungen kaum jemals genau an der
ge\-wünsch\-ten Stelle und möglicherweise nicht einmal auf
derselben Seite Platz finden, ist für viele Anfänger aber offenbar sehr
ungewohnt oder sogar beängstigend. Dennoch sollte zunächst einmal
getrost \latex\ diese Arbeit überlassen und \emph{nicht} manuell
eingegriffen werden. Erst am Ende, wenn das gesamte Dokument "`steht"' und
die automatische Platzierung wirklich nicht zufriedenstellend erscheint, sollte (durch gezielte Platzierungsanweisungen
\cite[S.~49]{Oetiker2015}) \textbf{in Einzelfällen} eingegriffen werden.



\section{Captions}

Bei Abbildungen steht der Titel üblicherweise \emph{unten}, bei
Tabellen hingegen -- je nach Konvention -- \emph{oben} (wie in diesem Dokument) 
oder ebenfalls \emph{unten}. In \latex\ erfolgt
auch die Nummerierung der Abbildungen automatisch, ebenso der
Eintrag in das (optionale)
Abbildungsverzeichnis%
\footnote{Ein eigenes Verzeichnis der Abbildungen am Anfang des Dokuments
ist zwar leicht erstellt, in einer Abschlussarbeit aber (und eigentlich
überall sonst auch) überflüssig. Man sollte es daher weglassen.}
am Beginn des Dokuments.

Die Markierung der Captions%
\footnote{Ausnahmsweise wird das Wort "`Caption"' im Folgenden
ohne deutsche Übersetzung verwendet.} erfolgt in \latex mithilfe
der \verb!\label{}! Anweisung, die unmittelbar auf die
\verb!\caption{}! Anweisung folgen muss:
%
\begin{LaTeXCode}[numbers=none]
\begin{figure}
\centering
\includegraphics[width=.95\textwidth]{cola-public-domain-photo-p}
\caption{Coca-Cola Werbung 1940 \cite{CocaCola1940}.}
\label{fig:CocaCola}
\end{figure}
\end{LaTeXCode}
%
Der Name des Labels (\texttt{fig:CocaCola}) kann beliebig gewählt werden. 
Die Kennzeichnung \texttt{fig:} ist (wie in Abschn.\ \ref{sec:querverweise} 
erwähnt) nur eine nützliche Hilfe, um beim Schreiben verschiedene Arten 
von Labels besser unterscheiden zu können.

Die Länge der Captions kann dabei sehr unterschiedlich sein. Je
nach Anwendung und Stil ergibt sich manchmal eine sehr kurze
Caption (Abb.~\ref{fig:CocaCola}) oder eine längere
(Abb.~\ref{fig:ibm360}).
Man beachte, wie bei kurzen Captions ein
zentrierter Satz und bei langen Captions ein Blocksatz verwendet
wird (\latex macht das automatisch).
Captions sollten \emph{immer} mit einem Punkt abgeschlossen sein.%
\footnote{Kurioserweise verlangen manche Anleitungen
genau das Gegenteil, angeblich, weil beim klassischen Bleisatz 
die abschließenden Punkte im Druck häufig "`weggebrochen"' sind. 
Das kann man glauben oder nicht, im Digitaldruck 
spielt es jedenfalls keine Rolle.}

\begin{figure}
\centering
\fbox{\includegraphics[width=.85\textwidth]{ibm-360-color}}  %{CS1065}}
%\FramePic{\includegraphics[width=.85\textwidth]{ibm-360-color}} 
\caption{Beispiel für einen langen Caption-Text. \textsc{Univac}
brachte 1961 mit dem Modell 751 den ersten Hochleistungsrechner
mit Halbleiterspeicher auf den Markt. Von diesem Computer wurden
in den U.S.A.\ bereits im ersten Produktionsjahr über fünfzig
Exemplare verkauft, vorwiegend an militärische Dienststellen,
Versicherungen und Großbanken. Die Ablöse erfolgte zwei Jahre
später durch das zusammen mit \textsc{Sperry} entwickelte Modell 820.
Das klingt vielleicht plausibel, ist aber völliger Unsinn, und das
Bild zeigt in Wirklichkeit eine System/360 Anlage von IBM. 
Bildquelle~\cite{IBM360}.} 
\label{fig:ibm360}
\end{figure}





\section{Abbildungen}

Für die Einbindung von Grafiken in \latex wird die Verwendung des Stan\-dard-Pakets
\texttt{graphicx} \cite{Carlisle2016} empfohlen 
(wird durch das \texttt{hagenberg}-Paket bereits eingebunden). 
Mit dem aktuell verwendeten Workflow (\texttt{pdflatex})
können Bild- bzw.\ Grafikformate ausschließlich 
in folgenden Formaten eingebunden werden:
%
\begin{itemize}
	\item \textbf{PNG}: für Grau-, S/W- und Farb-Rasterbilder (bevorzugt),
	\item \textbf{JPEG}: für Fotos (wenn nicht anders vorhanden),
	\item \textbf{PDF}: für Vektorgrafiken (Illustrationen, Strichzeichnungen etc.).
\end{itemize}
%
Bei Rasterbildern sollte wenn möglich PNG verwendet werden, weil die darin 
enthaltenen Bilder verlustfrei komprimiert sind und daher keine sichtbaren Kompressionsartefakte
aufweisen. Im Gegensatz dazu sollte JPEG nur dann verwendet werden, wenn das Originalmaterial
(Foto) bereits in dieser Form vorliegt.


\subsection{Wo liegen die Grafikdateien?} 

Die Bilder werden üblicherweise in einem Unterverzeichnis (oder in mehreren Unterverzeichnissen) abgelegt,
im Fall dieses Dokuments in \nolinkurl{images/}.
Dazu dient die folgende Anweisung
am Beginn des Hauptdokuments \nolinkurl{_DaBa.tex} (\sa\ Anhang \ref{app:latex}):
%
\begin{quote}
\verb!\graphicspath{{images/}}!
\end{quote}
%
Der (zum Hauptdokument relative) Pfad \texttt{graphicspath} kann innerhalb des
Dokuments jederzeit geändert werden, was durchaus nützlich ist, wenn
\zB\ die Grafiken einzelner Kapitel getrennt in entsprechenden Verzeichnissen
abgelegt werden sollen.
Die Größe der Abbildung im Druck kann durch Vorgabe einer bestimmten
Breite oder Höhe oder eines Skalierungsfaktors gesteuert werden, {\zB}:
%
\begin{quote}
\verb!\includegraphics[width=.85\textwidth]{ibm-360-color}! \\
\verb!\includegraphics[scale=1.5]{ibm-360-color}!
\end{quote}
%
Man beachte, dass dabei die Dateiendung nicht explizit angegeben werden muss. 
Das ist \va\ dann praktisch, wenn verschiedene Workflows mit jeweils
unterschiedlichen Dateitypen verwendet werden.


\subsection{Grafiken einrahmen} 

%Mit dem Makro \verb!\FramePic{}! (definiert in \texttt{hgb.sty}) kann optional ein dünner 
%Rahmen rund um die Grafik erzeugt werden, \zB:
Mit dem Makro \verb!\fbox{...}! kann optional ein dünner 
Rahmen rund um die Grafik erzeugt werden, \zB:
%
\begin{quote}
%\verb!\FramePic{\includegraphics[height=50mm]{ibm-360-color}}!
\verb!\fbox{\includegraphics[height=50mm]{ibm-360-color}}!
\end{quote}
%
Das wird üblicherweise nur bei Rasterbildern nötig sein, insbesondere wenn sie zum Rand hin sehr hell sind
und ohne Rahmen nicht vom Hintergrund abgrenzbar wären.

\subsection{Rasterbilder (Pixelgrafiken)}

Generell sollten Bilder bereits vorher so aufbereitet werden,
dass sie später beim Druck möglichst wenig an Qualität verlieren.
Es empfiehlt sich daher, die Bildgröße (Auflösung) bereits im Vorhinein
(\zB mit \emph{Photoshop})
richtig einzustellen.
Brauchbare Auflösungen bezogen auf die endgültige Bildgröße sind:
%
\begin{itemize}
  \item \textbf{Farb- und Grauwertbilder:} 150--300 dpi
  \item \textbf{Binärbilder (Schwarz/Weiß):} 300--600 dpi
\end{itemize}
%
Eine wesentlich höhere Auflösung macht aufgrund der beim Laserdruck notwendigen
Rasterung keinen Sinn, auch bei 1200 dpi-Druckern.
Speziell \emph{Screen\-shots} sollten nicht zu klein dargestellt werden,
da sie sonst schlecht lesbar sind (max.\ 200 dpi, besser 150 dpi).
Dabei ist zu bedenken, dass die Arbeit auch als Kopie in allen
Details noch gut lesbar sein sollte.

\subsubsection{JPEG-Problematik}

In der Regel sollten Bilder, die für den Einsatz in
Druckdokumenten gedacht sind, nicht mit verlustbehafteten
Kompressionsverfahren abgespeichert werden. Insbesondere sollte die Verwendung
von JPEG möglichst vermieden werden, auch wenn viele Dateien dadurch
wesentlich kleiner werden. 
Eine Ausnahme ist, wenn die Originaldaten nur in JPEG vorliegen und für die 
Einbindung nicht bearbeitet oder verkleinert wurden. Ansonsten sollte immer
PNG verwendet werden.

Besonders gerne werden farbige \textbf{Screenshots} einer JPEG-Kompression%
\footnote{Das JPEG-Verfahren ist für natürliche Fotos konzipiert und dafür auch gut geeignet,
seine undifferenzierte Verwendung ist aber zu einer globalen Plage geworden.}
unter\-zogen, obwohl deren verheerende Folgen für jeden Laien sichtbar sein sollten
(Abb.~\ref{fig:jpeg-pfusch}).

\begin{figure}
\centering\small
\begin{tabular}{@{}cc@{}}
\fbox{\includegraphics[width=0.475\textwidth]{screenshot-dirty}} &		% JPEG file
\fbox{\includegraphics[width=0.475\textwidth]{screenshot-clean}} \\	% PNG file
(a) & (b) 
\end{tabular}
\caption{Typischer JPEG-Pfusch. Screenshots und ähnliche im Original
verfügbare Rasterbilder sollten für Druckdokumente \emph{keinesfalls} mit
JPEG komprimiert werden. Das Ergebnis~(a) sieht gegenüber dem
unkomprimierten Original~(b) nicht nur schmutzig aus, sondern wird
im Druck auch schnell unleserlich.} 
\label{fig:jpeg-pfusch}
\end{figure}



\subsection{Vektorgrafiken}

Für schematische Abbildungen (\zB Flussdiagramme, Entity-Relationship-Diagramme
oder sonstige strukturelle Darstellungen) sollten unbedingt
Vektorgrafiken (PDF) verwendet werden. % (\zB Abb.~\ref{fig:latex-pdf-workflow}).
Gerasterte Grafiken, wie sie üblicherweise als GIF- oder PNG-Dateien
auf Webseiten vorliegen, haben in einem Druckdokument nichts zu suchen, notfalls
müssen sie mit einem entsprechenden Werkzeug \emph{neu} gezeichnet werden (natürlich
unter Angabe der ursprünglichen Quelle).

In diesem Fall kommt als Datenformat nur PDF %(oder EPS im DVI-PS-Workflow) 
in Frage,
dieses bietet sich aber auch in anderen Umgebungen als universelles
Vektor-Format an.
Zur Erstellung von PDF-Vektorgrafiken wird ein geeignetes
Grafikprogramm, \zB\ %\emph{Freehand} von \emph{Macromedia} oder
\emph{Illustrator} von \emph{Adobe} benötigt.
Manche gängigen Grafikprogramme 
unterstützen allerdings keinen direkten Export von PDF-Dateien
oder erzeugen unsaubere Dateien. Vor der Entscheidung
für eine bestimmte Zeichensoftware sollte das im Zweifelsfall
ausprobiert werden.
PDF kann im Notfall über einen entsprechenden Druckertreiber erzeugt werden.


\subsubsection{Vektorgrafiken mit \emph{Inkscape}}
\label{sec:InkscapeGraphics}

Mit \emph{Inkscape}\footnote{\url{https://inkscape.org/}} können Vektorgrafiken auf
sehr einfache Weise erstellt werden.
Das Basisformat von Inkscape ist SVG,
nach dem Export als PDF können solche Grafiken aber wie üblich mit
\verb!\includegraphics[..]{..}! in \latex\ eingefügt werden.

Eine interessante Möglichkeit dabei ist, Texte innerhalb der Grafik
durch \latex\ automatisch ersetzen zu lassen.
Dadurch werden in der fertigen Grafik dieselben Schriften wie im Fließtext
verwendet und \va\ mathematische Elemente entsprechend ersetzt.
Abbildung \ref{fig:InkscapeExample} zeigt ein Beispiel dazu:
%
\begin{itemize}
\item
Die ursprüngliche Inkscape-Grafik \nolinkurl{images/inkscape-template.svg}  enthält
Texte, die nachträglich von \latex\ ersetzt werden sollen 
(siehe Abb.~\ref{fig:InkscapeExample}\,(a)).
\item
Durch \textsf{Save a Copy...} (als PDF) in Inkscape, mit den Einstellungen wie in 
Abb.~\ref{fig:InkscapeExample}\,(c), werden folgende zwei Files erzeugt:
\begin{itemize}
\item[] \nolinkurl{inkscape-template.pdf}: eine PDF-Datei der Grafik ohne Texte, 
\item[] \nolinkurl{inkscape-template.pdf_tex}: eine \latex-Datei mit allen relevanten Informationen.
\end{itemize}
\end{itemize}
%
Die Einbindung der Grafik in das Dokument erfolgt schließlich durch
\begin{itemize}
\item[] \verb!\input{images/inkscape-template.pdf_tex}!,
\end{itemize}
mit dem in Abb.~\ref{fig:InkscapeExample}\,(b) gezeigten Ergebnis.



\begin{figure}
\centering\small
\begin{tabular}{cc}
\includegraphics[scale=1.0]{inkscape-template-orig} &
\input{images/inkscape-template.pdf_tex}
\\
(a) & (b)
\\[6pt]
\multicolumn{2}{c}{\includegraphics[width=0.4\textwidth]{inkscape-pdf-save-screenhot}%
~~\raisebox{25mm}{(c)}}
\end{tabular}
\caption{Beispiel für eine mit \emph{Inkscape} erzeugte Vektorgrafik
(\nolinkurl{inkscape-template.svg}).
Originalgrafik im \textit{Inkscape}-Editor (a);
beim Einfügen werden die Texte automatisch durch LaTeX ersetzt (b).
Beim Speichern in Inkscape (als PDF) ist auf die Einstellung "`PDF+LaTeX"' zu achten (c).}
\label{fig:InkscapeExample}
\end{figure}


\subsubsection{Einbettung von Schriften}

Die Wiedergabe von Textelementen ist abhängig von der auf dem
Computer (oder Drucker) installierten Schriften und der Form der
Schrifteinbettung im Quelldokument. Die korrekte Darstellung am
Bildschirm eines Computers bedeutet nicht, dass dasselbe Dokument
auf einem anderen Computer oder Drucker genau so dargestellt wird.
Dieser Umstand ist besonders wichtig, wenn Druckdokumente online
zur Verfügung gestellt werden. Kontrollieren Sie daher genau, ob
die innerhalb Ihrer Grafiken verwendeten Schriften auch exakt wie
beabsichtigt im Ausdruck aufscheinen.


\subsubsection{Strichstärken -- \emph{Hairlines} vermeiden!}

In Grafik-Programmen wie \emph{Freehand} und \emph{Illustrator},
die sich im Wesentlichen an der \emph{PostScript}-Funktionalität
orientieren, ist es möglich, Linien bzgl.\ ihrer Stärke als
"`Hairline"' zu definieren. Im zugehörigen \emph{PostScript}-Code
wird dies als \texttt{linewidth} mit dem Wert \texttt{0} ausgedrückt und
sollte am Ausgabegerät "`möglichst dünne"' Linien ergeben. 
Das Ergebnis ist ausschließlich vom jeweiligen Drucker
abhängig und somit kaum vorhersagbar.
\textbf{Fazit:} Hairlines vermeiden und stattdessen immer konkrete
Strichstärken ($\geq 0.25\,\mathrm{pt}$) einstellen!





\subsection{\tex-Schriften auch in Grafiken?}
\label{sec:tex-schriften-in-grafiken}

Während bei Abbildungen, die mit externen
Grafik-Programmen erzeugt werden, meist mit ähnlich aussehende
Schriften (wie \emph{Times-Roman} oder \emph{Garamond}) Abhilfe schaffen,
besteht bei Puristen oft der verständliche Wunsch, die 
\emph{Computer-Modern} (CM) Schriftfamilie von {\tex}/{\latex} auch
innerhalb von eingebetteten Grafiken einzusetzen.

\subsubsection{\emph{BaKoMa}-Schriften (TrueType)}

Glücklicherweise stehen einige Portierungen von CM als {\em
TrueType}-Schriften zur Verfügung, die auch in herkömmlichen
DTP-Anwendungen unter \emph{Windows} und \emph{Mac~OS} verwendet werden
können. Empfehlenswert ist \zB\ die \emph{BaKoMa Fonts Collection},%
\footnote{\url{http://ctan.org/pkg/bakoma-fonts}}
die neben den CM-Standardschriften auch die mathematischen Schriften
der AMS-Familie ent\-hält und zudem kostenfrei ist. Natürlich
müssen die TrueType Schriften vor der Verwendung zunächst auf dem
eigenen PC installiert werden. 


\subsubsection{\emph{Latin Modern Roman} Fonts (OpenType)}

Eine Alternative dazu sind die "`LM-Roman"'%
\footnote{\url{http://www.gust.org.pl/projects/e-foundry/latin-modern}}
 Open-Type Schriften, die speziell für die Verwendung im Umfeld von \latex\ entwickelt wurden.
Sie sind auch Teil der MikTeX-Installation.%
\footnote{\zB unter \url{C:/Program Files (x86)/MikTeX 2.9/fonts/opentype/public/lm/}}
Diese Schriften enthalten \ua\ Zeichen mit Umlauten und sind daher auch für 
deutsche Texte recht bequem zu verwenden.




\subsection{Für Gourmets: Grafiken mit \latex-Overlays}
\label{sec:GraphicOverlays}

Bisweilen ist es erforderlich, ein bestehendes Bilder oder eine Grafik mit 
\latex-eigenen (Vektor-)Elementen zu überlagern, \zB\ für Markierungen
oder Beschriftungen. Ein typisches Beispiel ist in Abb.~\ref{fig:overpic-example}
gezeigt, wo eine mit \emph{Mathematica} generierte PDF-Grafik
mit mathematischen Elementen annotiert wird.


\begin{figure}
\centering\small
%\includegraphics[width=0.85\textwidth]{mathematica-example}
\vspace*{3mm}
\begin{overpic}[width=0.85\textwidth]{mathematica-example}
	\put(101,14){$x$}%
	\put(4,31){$f(x)$}%
	\put(29.5,28){\line(1,1){2}}%
	{\color{green!70!black}\put(29.5,28){\circle*{2.0}}}%
	\put(32,30){$\cos(\frac{7}{3} x)$}%
	\put(59,28){\line(1,1){2}}%
	{\color{blue!70!black}\put(59,28){\circle*{2.0}}}%
	\put(61.5,30){$\cos(x)$}%
\end{overpic}
\caption{Beispiel für die Verwendung des \texttt{overpic}-Pakets zum Einfügen
von \latex-Elementen über eine importierte Grafik.
In diesem Fall wurden die mathematischen Elemente $x$, $f(x)$, $\cos(x)$ und $\smash{\cos(\frac{7}{3} x)}$
sowie zwei diagonale Geraden und gefüllte (färbige) Kreise eingefügt.
Darunter liegt die Vektor\-grafik \texttt{mathematica-example.pdf}.}
\label{fig:overpic-example}
\end{figure}



Dazu wird das \texttt{overpic}-Paket%
\footnote{\url{https://www.ctan.org/pkg/overpic}}
verwendet und zum Importieren der Grafik anstelle von \verb!\includegraphics!
die Umgebung \verb!\begin{overpic}! \ldots \verb!\end{overpic}! verwendet 
(mit ähnlicher Syntax):

\begin{LaTeXCode}[numbers=none]
\begin{overpic}[width=0.85\textwidth]{mathematica-example}
	\put(101,14){$x$}%
	\put(4,31){$f(x)$}%
	\put(29.5,28){\line(1,1){2}}%
	...
\end{overpic}
\end{LaTeXCode}

Die \texttt{overpic}-Umgebung bildet gleichzeitig eine \texttt{picture}-Umgebung, 
in der \latex-Zeichenanweisungen (wie \verb!\put! u.ä.) platziert werden
können, wie in obigem Beispiel gezeigt.\footnote{Die Standard-Zeichenanweisungen
in \latex sind ziemlich restriktiv, weshalb hier zusätzlich das \texttt{pict2e}-Paket
(\url{https://www.ctan.org/pkg/pict2e}) verwendet wird.}
Die $x/y$-Positionen sind in Prozent der Bildbreite angegeben.
Weitere Details finden sich im Quelltext.





\subsection{Abbildungen mit mehreren Elementen}

Werden mehrere Bilder oder Grafiken zu einer Abbildung zusammengefasst, 
wird üblicherweise eine gemeinsame Caption verwendet, wie in Abb.~\ref{fig:Bearings}
dargestellt. Im Text könnte ein Verweis auf einen einzelnen Teil der Abbildung, etwa das 
einreihige Rollenlager in Abb.~\ref{fig:Bearings}\,(c), so aussehen:
%
\begin{LaTeXCode}[numbers=none]
    ... Abb.~\ref{fig:Bearings} (c) ... 
\end{LaTeXCode}


\subsection{Quellenangaben in Captions}
\label{sec:QuellenangabenInCaptions}

Wenn Bilder, Grafiken oder Tabellen aus anderen Quellen verwendet werden, dann 
muss ihre Herkunft in jedem Fall klar ersichtlich gemacht werden, und zwar am 
besten direkt in der Caption.
Wird beispielsweise eine Grafik aus einem Buch oder einer sonstigen 
zitierfähigen Publikation verwendet, dann sollte diese in das Literaturverzeichnis 
aufgenommen und wie üblich mit
\verb!\cite{..}! zitiert werden, wie in Abb.\ \ref{fig:Bearings} demonstriert. 
Weitere Details zu dieser Art von Quellenangaben finden sich in 
Kap.\ \ref{cha:Literatur} (insbes.\ Abschnitt \ref{sec:KategorieOnline}).

\begin{figure}
\centering\small
\begin{tabular}{@{}c@{\hspace{12mm}}c@{}} % mittlerer Abstand = 12mm
  \includegraphics[width=.45\textwidth]{overhang-mounting} &
  \includegraphics[width=.45\textwidth]{straddle-mounting} 
\\
  (a) & (b)
\\[4pt]	%vertical extra spacing (4 points)
  \includegraphics[width=.45\textwidth]{ball-bearing-1} &
  \includegraphics[width=.45\textwidth]{ball-bearing-2} 
\\
  (c) & (d)
\end{tabular}
%
\caption{Diverse Maschinenelemente als Beispiel für eine
Abbildung mit mehreren Elementen.
\emph{Overhang Mounting}~(a), \emph{Straddle Mounting}~(b),
einreihiges Rollenlager~(c), Schmierung von Rollenlagern~(d).
Diese Abbildung verwendet eine gewöhnliche Tabelle (\texttt{tabular}) mit
2 Spalten und 4 Zeilen (Details finden sich im Quelltext).
Bildquelle~\cite{Faires1934}.}
\label{fig:Bearings}
\end{figure}




\section{Tabellen}

Tabellen werden häufig eingesetzt um numerische Zusammenhänge, Testergebnisse
etc.\ in übersichtlicher Form darzustellen.
Ein einfaches Beispiel ist Tab.~\ref{tab:processors}, der \latex-Quelltext dazu
findet sich in Prog.~\ref{prog:processors-source}.


\begin{table}
\caption{Prozessor-Familien im Überblick.}
\label{tab:processors}
\centering
\setlength{\tabcolsep}{5mm}	% separator between columns
\def\arraystretch{1.25}			% vertical stretch factor (Standard = 1.0)
\begin{tabular}{|r||c|c|c|} \hline
& \emph{PowerPC} & \emph{Pentium} & \emph{Athlon} \\
\hline\hline
Manufacturer & Motorola & Intel & AMD \\
\hline
Speed & high & medium & high   \\
\hline
Price & high & high   & medium \\
\hline
\end{tabular}
\end{table}

\begin{program}
% place caption consistently either at the top or bottom:
\caption{\latex\ Quelltext zu Tab.~\ref{tab:processors}.
Die Erzeugung des dargestellten Listings selbst ist in Abschn.\ \ref{sec:programmtexte} beschrieben.}
\label{prog:processors-source}
%
\begin{LaTeXCode}[numbers=none]
\begin{table}
	\caption{Prozessor-Familien im Überblick.}
	\label{tab:processors}
	\centering
	\setlength{\tabcolsep}{5mm}	% separator between columns
	\def\arraystretch{1.25}		% vertical stretch factor
	\begin{tabular}{|r||c|c|c|} 
		\hline
		& \emph{PowerPC} & \emph{Pentium} & \emph{Athlon} \\
		\hline
		\hline
		Manufacturer & Motorola & Intel & AMD \\
		\hline
		Speed & high & medium & high   \\
		\hline
		Price & high & high   & medium \\
		\hline
	\end{tabular}
\end{table}
\end{LaTeXCode}
%
\end{program}

Manchmal ist es notwendig, in Tabellen relativ viel Text in engen Spalten
unter zu bringen, wie in Tab.~\ref{tab:synthesis-techniques}. In diesem Fall
ist es sinnvoll, auf den Blocksatz zu verzichten und gleichzeitig die
strengen Abteilungsregeln zu lockern. Details dazu finden sich im zugehörigen
\latex-Quelltext.


%--------------------------------------------------------------------------------
% Table with narrow columns
%--------------------------------------------------------------------------------
\begin{table}
\caption{Beispiel für eine Tabelle mit mehrzeiligem Text in engen Spalten.
Hier werden die Zeilen für den Blocksatz zu kurz, daher wird linksbündig
gesetzt (im "`Flattersatz"').}
\label{tab:synthesis-techniques}
\centering
\def\rr{\rightskip=0pt plus1em \spaceskip=.3333em \xspaceskip=.5em\relax}
\setlength{\tabcolsep}{1ex}
\def\arraystretch{1.20}
\setlength{\tabcolsep}{1ex}
\small
\begin{english}
\begin{tabular}{|p{0.2\textwidth}|c|p{0.3\textwidth}|p{0.2\textwidth}|}
\hline
   \multicolumn{1}{|c}{\emph{Method}} &
   \multicolumn{1}{|c}{\emph{Implem.}} &
   \multicolumn{1}{|c}{\emph{Features}} &
   \multicolumn{1}{|c|}{\emph{Status}} \\
\hline\hline
   {\rr polygon shading} &
   SW/HW &
   {\rr flat-shaded polygons} &
   \\
\hline
  {\rr flat shading with z-buffer} &
  SW/HW &
  {\rr depth values} &
  \\
\hline
  {\rr goraud shading with z-buffer} &
  SW/HW &
  {\rr smooth shading, simple fog, point light sources} &
  {\rr SGI entry models} \\
\hline
  {\rr phong shading with z-buffer} &
  SW/HW &
  {\rr highlights} &
  \\
\hline
  {\rr texture mapping with z-buffer} &
  SW/HW &
  {\rr surface textures, simple shadows} &
  {\rr SGI high end, flight simulators} \\
\hline
%  {\rr reflection mapping with z-buffer} &
%  SW/HW &
%  {\rr reflections} &
%  {\rr SGI next generation} \\
%\hline
%  {\rr raytracing} &
%  SW &
%  {\rr refraction, real camera model, area light sources with penumbra, realistic material models} &
%  {\rr common ray\-tracers} \\
%\hline
%  {\rr raytracing + global illumination simulation} &
%  SW &
%  {\rr indirect illumination} &
%  \textit{Radiance} \\
%\hline
%  {\rr raytracing + global illumination simulation + dissipating media} &
%  none &
%  {\rr realistic clouds, scattering, ...} &
%  {\rr research} \\
%\hline
\end{tabular}
\end{english}
\end{table}

%--------------------------------------------------------------------------------



\section{Programmtexte}
\label{sec:programmtexte}

Die Einbindung von Programmtexten (source code) ist eine häufige Notwendigkeit,
\va natürlich bei Arbeiten im Bereich der Informatik.


\subsection{Formatierung von Programmcode}
\label{sec:FormatierungVonProgrammcode}

Es gibt für \latex\ spezielle Pakete zur Darstellung von Programmen, die \ua\ auch die automatische
Nummerierung der Zeilen vornehmen, insbesondere die Pakete \texttt{listings}%
\footnote{\url{https://ctan.org/pkg/listings}}
und \texttt{listingsutf8}.%
\footnote{\url{https://ctan.org/pkg/listingsutf8}}
Damit sind auch die in Tabelle~\ref{tab:CodeUmgebungen} aufgelisteten Code-Umgebungen realisiert.
%
\begin{table}
\caption{In \nolinkurl{hgb.sty} vordefinierte Code-Umgebungen.}
\label{tab:CodeUmgebungen}
\centering
\begin{tabular}{llll}
	\hline
	C (ANSI): & \verb!\begin{CCode}! & \verb!...! \verb!\end{CCode}! \\
	C++ (ISO): & \verb!\begin{CppCode}! & \verb!...! \verb!\end{CppCode}! \\
	C\#: & \verb!\begin{CsCode}! & \verb!...! \verb!\end{CsCode}! \\
	CSS: & \verb!\begin{CssCode}! & \verb!...! \verb!\end{CssCode}! \\
	HTML: & \verb!\begin{HtmlCode}! & \verb!...! \verb!\end{HtmlCode}! \\
	Java: & \verb!\begin{JavaCode}! & \verb!...! \verb!\end{JavaCode}! \\
	JavaScript: & \verb!\begin{JsCode}! & \verb!...! \verb!\end{JsCode}! \\
	\latex: & \verb!\begin{LaTeXCode}! & \verb!...! \verb!\end{LaTeXCode}! \\
	Objective-C: & \verb!\begin{ObjCCode}! & \verb!...! \verb!\end{ObjCCode}! \\
	PHP: & \verb!\begin{PhpCode}! & \verb!...! \verb!\end{PhpCode}! \\
	Swift: & \verb!\begin{SwiftCode}! & \verb!...! \verb!\end{SwiftCode}! \\
	XML: & \verb!\begin{XmlCode}! & \verb!...! \verb!\end{XmlCode}! \\
	Generisch: & \verb!\begin{GenericCode}! & \verb!...! \verb!\end{GenericCode}! \\
	\hline
\end{tabular}
\end{table}
%
Die Verwendung ist äußerst einfach, \zB\ für Quellcode in der Programmiersprache C schreibt man
%
\begin{quote}
\begin{verbatim}
\begin{CCode}
    ... 
\end{CCode}
\end{verbatim}
\end{quote}
%
Der Quellcode innerhalb dieser Umgebungen wird in der jeweiligen Programmiersprache interpretiert, wobei Kommentare erhalten bleiben. Diese Umgebungen können sowohl alleinstehend (im Fließtext) oder innerhalb von Float-Umgebungen (insbes.\ \texttt{program}) verwendet werden. Im ersten Fall wird der Quelltext auch über Seitengrenzen umgebrochen. Mit \verb!/+! ... \verb!+/! ist eine Escape-Möglichkeit nach \latex\ vorgesehen, die \va\ zum Setzen von Labels für Verweise auf einzelne Programmzeilen nützlich ist, \zB\ mit
%
\begin{quote}
\verb!/+\label{ExampleCodeLabel}+/!
\end{quote}
%
Ein Beispiel mit Java ist in Prog.~\ref{prog:CodeExample} gezeigt, wobei der oben angeführte Label in Zeile \ref{ExampleCodeLabel} steht.
Man beachte, dass innerhalb der Kommentare auch mathematischer Text 
(wie etwa in Zeile \ref{MathInCode} von Prog.~\ref{prog:CodeExample}) stehen kann.


\subsubsection{Nummerierung der Code-Zeilen}

Alle in Tabelle~\ref{tab:CodeUmgebungen} angeführten Code-Umgebungen können
mit optionalen Argumenten verwendet werden, die insbesondere zur Steuerung der
Zeilennummerierung hilfreich. 
Im Normalfall (also ohne zusätzliche Angabe) mit
%
\begin{quote}
\verb!\begin{!\texttt{\emph{some}Code}\verb!} ... !
\end{quote}
%
werden alle Code-Zeilen (einschließlich der Leerzeilen) bei 1 beginnend und 
fortlaufend nummeriert.
%
Bei aufeinanderfolgenden Codesegmenten ist es oft hilfreich, die Nummerierung 
aus dem vorherigen Abschnitt kontinuierlich weiter laufen zu lassen,
ermöglicht durch die Angabe des optionalen Arguments 
\texttt{firstnumber={\obnh}last}:
%
\begin{quote}
\verb!\begin{!\texttt{\emph{some}Code}\verb!}[firstnumber=last] ... !
\end{quote}
%
Um die Nummerierung der Codezeilen gänzlich zu unterbinden genügt die Angabe
des optionalen Arguments
\texttt{numbers={\obnh}none}:
%
\begin{quote}
\verb!\begin{!\texttt{\emph{some}Code}\verb!}[numbers=none] ... !
\end{quote}
%
In diesem Fall ist natürlich die Verwendung von Zeilenlabels im Code nicht
sinnvoll.


\subsection{Platzierung von Programmcode}

Da Quelltexte sehr umfangreich werden können, ist diese Aufgabe nicht
immer leicht zu lösen. Abhängig vom Umfang und vom Bezug zum Haupttext
gibt es grundsätzlich drei Möglichkeiten zur Einbindung von Programmtext:
%
\begin{itemize}
\item[a)] im laufenden Text für kurze Programmstücke,
\item[b)] als Float-Element (\texttt{program}) für mittlere Programmtexte bis max.\ eine Seite oder
\item[c)] im Anhang (für lange Programmtexte).
\end{itemize}

\subsubsection{Programmtext im laufenden Text}

Kurze Codesequenzen können ohne weiteres im laufenden Text
eingebettet werden, sofern sie an den gegebenen Stellen von unmittelbarer
Bedeutung sind. Die folgende (rudimentäre) Java-Methode \texttt{extractEmail} sucht
nach einer E-Mail-Adresse in der Zeichenkette
\texttt{line}:
%
\begin{JavaCode}[numbers=none]
static String extractEmail(String line) {
    line = line.trim(); // find the first blank
    int i = line.indexOf(' '); 
    if (i > 0)
        return line.substring(i).trim();
    else
        return null;
}
\end{JavaCode}
\medskip

\noindent
Dieses Codestück wurde mit 
%
\begin{quote}
\begin{verbatim}
\begin{JavaCode}[numbers=none]
static String extractEmail(String line) {
    line = line.trim(); // find the first blank
    ...
}
\end{JavaCode}
\end{verbatim}
\end{quote}
%
erstellt (siehe Abschn.\ \ref{sec:FormatierungVonProgrammcode}). 
In-line Programmstücke sollten maximal einige Zeilen lang sein und 
nach Möglichkeit nicht durch Seitenumbrüche geteilt werden.
%Um auch längere Programmzeilen unterzubringen, empfiehlt es sich, dafür
%eine entsprechend kleine Schriftgröße zu wählen (als Standardgröße ist
%\texttt{footnotesize} eingestellt). 


\subsubsection{Programmtexte als Float-Elemente}
Sind längere Codesequenzen notwendig, die in unmittelbarer Nähe des laufenden Texts
stehen müssen, sollten diese genauso wie andere Abbildungen als Float-Elemente
behandelt werden. Diese Programmtexte sollten den Umfang von einer Seite nicht übersteigen.
Im Notfall können auch bis zu zwei Seiten in aufeinanderfolgende Abbildungen gepackt werden,
jeweils mit eigener Caption. In \texttt{hgb.sty} ist eine neue Float-Umgebung \texttt{program} definiert, die analog zu \texttt{table} verwendet wird:
%
\begin{quote}
\begin{verbatim}
\begin{program}
\caption{Der Titel zu diesem Programmstück.}
\label{prog:xyz}
\begin{JavaCode}
  class IrgendWas {
    ...
  }
\end{JavaCode}
\end{program}
\end{verbatim}
\end{quote}
%
Wenn gewünscht, kann die Caption auch unten angebracht werden 
(jedenfalls aber konsistent und nicht gemischt).
Natürlich darf auch hier nicht mit einer linearen Abfolge im fertigen
Druckbild gerechnet werden, daher sind Wendungen wie
"`... im  folgenden Programmstück ..."' zu vermeiden und entsprechende Verweise
einzusetzen. Beispiele sind Programme \ref{prog:processors-source} und \ref{prog:CodeExample}.

\begin{program}
% place caption consistently either at the top or bottom:
\caption{Beispiel für die Auflistung von Programmcode als Float-Element.}
\label{prog:CodeExample}
\begin{JavaCode}
import ij.ImagePlus;
import ij.plugin.filter.PlugInFilter;
import ij.process.ImageProcessor;

public class My_Inverter implements PlugInFilter {
	int agent_velocity;
  String title = ""; // just to test printing of double quotes

	public int setup (String arg, ImagePlus im) {
		return DOES_8G;	// this plugin accepts 8-bit grayscale images /+\label{pr:IjSamplePlugin10}+/
	}

	public void run (ImageProcessor ip) {
		int w = ip.getWidth();	/+\label{ExampleCodeLabel}+/
		int h = ip.getHeight(); 
		
		/* iterate over all image coordinates */
		for (int u = 0; u < w; u++) { 
			for (int v = 0; v < h; v++) {
				int p = ip.getPixel(u, v); 
				ip.putPixel(u, v, 255-p); // invert: /+$I'(u,v) \leftarrow 255 - I(u,v)$\label{MathInCode}+/
			}
		}
	}		
} // end of class My_Inverter
\end{JavaCode}
%
\end{program}


\subsubsection{Programmtext im Anhang}

Für längere Programmtexte, speziell wenn sie vollständige
Implementierungen umfassen und im aktuellen Kontext nicht
unmittelbar relevant sind, muss zur Ablage in einem getrennten
Anhang am Ende des Dokuments gegriffen werden. Für Hinweise auf bestimmte
Details können entweder kurze Ausschnitte in den laufenden Text
gestellt oder mit entsprechenden Seitenverweisen gearbeitet werden. Ein
solches Beispiel ist der \latex-Quellcode in Anhang
\ref{app:latex} (Seite \pageref{app:latex}).%
\footnote{%
Grundsätzlich ist zu überlegen, ob die gedruckte Einbindung der gesamten
Programmtexte einer Implementierung für den Leser überhaupt sinnvoll ist, oder
ob diese nicht besser elektronisch (auf Datenträger) beigefügt und nur exemplarisch
beschrieben werden.}

\chapter[Mathem.\ Formeln etc.]{Mathematische Formeln, Gleichungen und Algorithmen}
\label{cha:Mathematik}


\chapter[Umgang mit Literatur]{Umgang mit Literatur und anderen Quellen}
\label{cha:Literatur}

\paragraph{Anmerkung:}
Der Titel dieses Kapitels ist absichtlich so
lang geraten, dass er nicht mehr in die Kopfzeile der Seiten passt. 
In diesem Fall kann in der \verb!\chapter!-Anweisung 
als optionales Argument \verb![..]! ein verkürzter Text für die
Kopfzeile (und das Inhaltsverzeichnis) angegeben werden:
%
\begin{LaTeXCode}[numbers=none]
\chapter[Umgang mit Literatur]{Umgang mit Literatur und anderen Quellen}
\end{LaTeXCode}

\section{Allgemeines}

Der richtige Umgang mit Quellen ist ein wesentliches Element bei der Erstellung
wissenschaftlicher Arbeiten im Allgemeinen (\sa\ Abschnitt \ref{sec:Plagiarismus}).
Für die Gestaltung von Quellenangaben sind unterschiedlichste Richtlinien in
Gebrauch, bestimmt \ua\ vom jeweiligen Fachgebiet oder Richtlinien von Verlagen und Hochschulen.
Diese Vorlage sieht ein Schema vor, das in den natur\-wissen\-schaftlich-technischen 
Disziplinen üblich ist.%
\footnote{Anpassungen an andere Formen sind relativ leicht möglich.}
Technisch basiert dieser Teil auf \texttt{BibTeX} \cite{Patashnik1988}
\bzw\ \texttt{Biber}%
\footnote{Wird seit Version 2013/02/19 anstelle von \texttt{bibtex} verwendet 
	(s.\ \url{http://biblatex-biber.sourceforge.net/}),}
in Kombination mit dem Paket \texttt{biblatex} \cite{Lehman2015}.


Die Verwaltung von Quellen besteht grundsätzlich aus zwei Elementen: 
\emph{Quellenverweise} im Text beziehen sich auf Einträge im \emph{Quellenverzeichnis}
(oder in mehreren Quellenverzeichnissen).
Das Quellenverzeichnis ist eine
Zusammenstellung aller verwendeten Quellen, typischerweise ganz am Ende des Dokuments.
Wichtig ist, dass jeder Quellenverweis einen zugehörigen, eindeutigen
Eintrag im Quellenverzeichnis aufweist und jedes Element im Quellenverzeichnis auch
im Text referenziert wird.



\section{Quellenverweise}

Um einen Eintrag im Quellenverzeichnis zu erstellen und im Text darauf zu verweisen, stellt \latex ein zentrales Kommando zur Verfügung.

\subsection{Das {\tt {\bs}cite} Makro}

Für Quellenverweise im laufenden Text verwendet man die Anweisung
\begin{itemize}
\item[] \verb!\cite{!\textit{Verweise}\verb!}! 
				\quad oder \quad
        \verb!\cite[!\textit{Zusatztext}\verb!]{!\textit{Verweise}\verb!}!.
\end{itemize}

\noindent%
\textit{Verweise} ist eine durch Kommas getrennte Auflistung von $1$--$n$ Quellen-\emph{Schlüsseln}
zur Identifikation der entsprechenden Einträge im Quellenverzeichnis.
Mit \textit{Zusatztext} können Ergänzungstexte zum aktuellen Quellenverweis angegeben
werden, wie \zB Kapitel- oder Seitenangaben bei Büchern.
Einige Beispiele dazu:

\begin{LaTeXCode}[numbers=none]
Mehr dazu findet sich in \cite{Kopka2003}.
\end{LaTeXCode}
$\rightarrow$ Mehr dazu findet sich in \cite{Kopka2003}.

\begin{LaTeXCode}[numbers=none]
Mehr über \emph{Styles} in \cite[Kap.\ 3]{Kopka2003}.
\end{LaTeXCode}
$\rightarrow$ Mehr über \emph{Styles} in \cite[Kap.\ 3]{Kopka2003}.

\begin{LaTeXCode}[numbers=none]
Die Angaben in \cite[S.\ 274--277]{BurgeBurger1999} sind falsch.
\end{LaTeXCode}
$\rightarrow$ Die Angaben in \cite[S.\ 274--277]{BurgeBurger1999} sind falsch.

\begin{LaTeXCode}[numbers=none]
Überholt sind auch \cite{BurgeBurger1999, Patashnik1988, Duden1997}.
\end{LaTeXCode}
$\rightarrow$ Überholt sind auch \cite{BurgeBurger1999, Patashnik1988, Duden1997}.

Die Sortierung der Angaben im letzten Beispiel erfolgt automatisch.



\subsection{Häufige Fehler}

\subsubsection{Verweise außerhalb des Satzes}
Quellenverweise sollten innerhalb oder am Ende eines Satzes (\dah vor
dem Punkt) stehen, nicht \emph{außerhalb}:
%
\begin{center}
\begin{tabular}{rl}
 \textbf{Falsch:}  & \ldots hier ist der Satz aus. \cite{Oetiker2015} Und jetzt geht es weiter \ldots \\
 \textbf{Richtig:} & \ldots hier ist der Satz aus \cite{Oetiker2015}. Und jetzt geht es weiter \ldots
\end{tabular}
\end{center}

\subsubsection{Verweise ohne vorangehendes Leerzeichen}

Ein Quellenverweis ist \emph{immer} durch ein Leerzeichen vom vorangehenden Wort getrennt, niemals wird er (wie etwa eine Fußnote) direkt an das Wort geschrieben:

\begin{center}
\begin{tabular}{rl}
\textbf{Falsch:}  & \ldots hier folgt die Quellenangabe\cite{Oetiker2015} und es geht weiter \ldots \\
\textbf{Richtig:} & \ldots hier folgt die Quellenangabe \cite{Oetiker2015} und es geht weiter \ldots
\end{tabular}
\end{center}

\subsubsection{Zitate}
Falls ein ganzer Absatz (oder mehr) aus einer Quelle zitiert wird,
sollte der Verweis im vorlaufenden Text und nicht
\emph{innerhalb} des Zitats selbst platziert werden. Als Beispiel die folgende Passage
aus \cite{Oetiker2015}:
\begin{quote}
Typographical design is a craft. Unskilled authors often commit
serious formatting errors by assuming that book design is mostly a
question of aesthetics---``If a document looks good artistically,
it is well designed.'' But as a document has to be read and not
hung up in a picture gallery, the readability and
understandability is of much greater importance than the beautiful
look of it.%
\footnote{Man beachte die Verwendung von englischen Hochkommas innerhalb dieses
Zitats.}
\end{quote}
Für das Zitat selbst sollte übrigens die dafür vorgesehene Umgebung
%
\begin{itemize}
 \item[] \verb!\begin{quote}! \emph{Zitierter Text ...} \verb!\end{quote}!
\end{itemize}
%
verwendet werden, die durch beidseitige Einrückungen das
Zitat vom eigenen Text klar abgrenzt und damit die Gefahr von
Unklarheiten (wo ist das Ende des Zitats?) mindert.
Wenn gewünscht, kann das Innere des Zitats auch in Hochkommas verpackt 
\emph{oder} kursiv gesetzt werden -- aber nicht beides!



\subsection{Umgang mit Sekundärquellen}

In seltenen Fällen kommt es vor, dass man eine Quelle \textbf{A} angeben
möchte (oder muss), die man zwar nicht zur Hand -- und damit auch nicht selbst gelesen --
hat, die aber in einer \emph{anderen}, vorliegenden Quelle \textbf{B} zitiert wird.
In diesem Fall wird \textbf{A} als \emph{Original-} oder \emph{Primärquelle} und \textbf{B} 
als \emph{Sekundärquelle} bezeichnet. Dabei sollten folgende Grundregeln beachtet werden:
%
\begin{itemize}
\item
\textbf{Sekundärquellen} nach Möglichkeit überhaupt \textbf{vermeiden}!
\item
Um eine Quelle in der üblichen Form zitieren zu können, muss man sie \textbf{immer selbst
eingesehen} (gelesen) haben!
\item
Nur wenn man die Quelle wirklich \textbf{nicht} beschaffen kann, ist ein Verweis über eine Sekundärquelle
zulässig. In diesem Fall sollten korrekterweise Pri\-mär- und Sekundärquelle \emph{gemeinsam} 
angegeben werden, wie im nachfolgenden Beispiel gezeigt.
\item
\textbf{Wichtig:} In das Quellenverzeichnis wird \textbf{nur die tatsächlich vorliegende Quelle} 
(\textbf{B}) und nicht die Originalarbeit aufgenommen!
\end{itemize}
%
\textbf{Beispiel:} Angenommen man möchte aus dem berühmten Buch \emph{Dialogo} von Galileo Galilei 
(an das man nur schwer herankommt) eine Stelle zitieren, die man in einem neueren Werk aus dem Jahr 1969 
gefunden hat. Das könnte man \zB\ mit folgender Fußnote bewerkstelligen.%
\footnote{Galileo Galilei, \emph{Dialogo sopra i due massimi sistemi del mondo tolemaico e copernicano}, 
S.~314 (1632). Zitiert nach \cite[S.~59]{Hemleben1969}.} % Alle Seitennummern sind frei erfunden!


\section{Quellenverzeichnis}

Für die Erstellung des Quellenverzeichnisses gibt es in \latex zwei
Möglichkeiten:
\begin{enumerate}
\item Das Quellenverzeichnis manuell zu formatieren (nicht zu empfehlen, 
s.\ \cite[Abschn.\ 11.3]{Kopka2003}).
\item Die Verwendung von \texttt{BibTeX}/\texttt{Biber} und einer zugehörigen 
"`Literaturdatenbank"' (s.\ \cite[Kap.\ 12]{Kopka2003}).
\end{enumerate}
Tatsächlich ist die erste Variante nur bei sehr wenigen Literaturangaben interessant.
Die Arbeit mit \texttt{BibTeX}/\texttt{biber} macht sich hingegen schnell bezahlt und ist zudem wesentlich
flexibler.

\subsection{Literaturdaten in BibTeX}
\label{sec:bibtex}

BibTeX ist ein eigenständiges Programm, das aus einer "`Literaturdatenbank"' (eine oder mehrere
Textdateien mit vorgegebener Struktur) ein für \latex geeignetes Quellenverzeichnis
erzeugt. Literatur zur Verwendung von BibTeX findet sich online, \zB \cite{Feder2006, Patashnik1988}.
Die BibTeX-Datei zu dieser Vorlage ist \nolinkurl{literatur.bib} (im Hauptverzeichnis).

BibTeX-Dateien können natürlich mit einem Texteditor manuell erstellt werden und für
viele Literaturquellen sind bereits fertige BibTeX-Einträge online verfügbar.
Dabei sollte man allerdings vorsichtig sein, denn diese Einträge sind (auch bei großen
Institutionen und Verlagen) \textbf{häufig falsch oder syntaktisch fehlerhaft}!
Man sollte sie daher nicht ungeprüft übernehmen und insbesondere die Endergebnisse genau kontrollieren.
Darüber hinaus gibt es eigene Anwendungen zur Wartung von
BibTeX-Verzeichnissen, wie beispielsweise
\emph{JabRef}.\footnote{\url{http://jabref.sourceforge.net/}}


\subsubsection{Verwendung von \texttt{biblatex} und \texttt{biber}}

Dieses Dokument verwendet \texttt{biblatex} (Version 1.4 oder höher) in Verbindung
mit dem Programm \texttt{biber}, 
das viele Unzulänglichkeiten des traditionellen BibTeX-Work\-flows behebt und dessen Möglichkeiten deutlich erweitert.%
\footnote{Tatsächlich ist \texttt{biblatex} die erste radikale (und längst notwendige) Überarbeitung des mittlerweile stark in die Jahre gekommenen BibTeX-Workflows. Zwar wird dabei BibTex weiterhin für 
die Sortierung der Quellen verwendet, die Formatierung der Einträge und viele andere Elemente werden jedoch ausschließlich über \latex-Makros gesteuert.}
Allerdings sind die in \texttt{biblatex} verwendeten Literaturdaten nicht mehr vollständig 
rückwärts-kompatibel zu BibTeX. Es ist daher in der Regel notwendig, bestehende oder aus
Online-Quellen übernommene BibTeX-Daten manuell zu überarbeiten (\sa\ Abschnitt~\ref{sec:TippsZuBibtex}).

In dieser Vorlage sind die Schnittstellen zu \texttt{biblatex} weitgehend in der Datei \nolinkurl{hgbthesis.cls} verpackt. Die typische Verwendung in der \latex-Haupt\-datei sieht folgendermaßen aus:
%
\begin{LaTeXCode}[numbers=none]
\documentclass[master,german]{hgbthesis}
   ...
\bibliography{literatur} /+\label{tex:literatur1}+/
   ...
\begin{document}
   ...
\MakeBibliography{Quellenverzeichnis} /+\label{tex:literatur2}+/
\end{document}
\end{LaTeXCode}
%
In der "`Präambel"' (Zeile \ref{tex:literatur1}) wird mit \verb!\bibliography{literatur}! 
auf eine (modifizierte) BibTex-Datei \nolinkurl{literatur.bib} verwiesen.%
\footnote{Das Makro 
\texttt{{\bs}bibliography} ist eigentlich ein Relikt aus BibTeX
und wird in \texttt{biblatex} durch die Anweisung \texttt{{\bs}addbibresource} 
ersetzt. Beide Anweisungen sind gleichwertig, allerdings wird nur mit 
\texttt{{\bs}bibliography} die zugehörige \texttt{.bib}-Datei im Fileverzeichnis von TeXnicCenter angezeigt.}
Falls mehrere BibTeX-Dateien verwendet werden, können sie in der gleichen Form angegeben werden.

Die Anweisung \verb!\MakeBibliography{..}! am Ende des Dokuments (Zeile~\ref{tex:literatur2})
besorgt die Ausgabe des Quellenverzeichnisses, hier mit dem Titel "`Quellenverzeichnis"'.
Dabei sind zwei Varianten möglich:
%
\begin{description}
\item[\texttt{{\bs}MakeBibliography\{title\}}] ~ \newline
   Erzeugt ein in mehrere \emph{Kategorien} (s.\ Abschnitt \ref{sec:BibKategorien}) geteiltes Quellenverzeichnis 
   mit der Hauptüberschrift \texttt{title}. Diese Variante wird im vorliegenden Dokument verwendet.
\item[\texttt{{\bs}MakeBibliography[nosplit]\{title\}}] ~ \newline
   Erzeugt ein traditionelles \emph{einteiliges} Quellenverzeichnis mit der
   Überschrift \texttt{title}. 
\end{description}
%
Die Überschrift \texttt{title} scheint in beiden Fällen ohne Nummerierung 
im Inhaltsverzeichnis auf (auf \texttt{chapter}-Ebene).
Bei geteiltem Quellenverzeichnis werden die Überschriften der einzelnen Abschnitte auf 
\texttt{section}-Ebene eingetragen.


\subsection{Kategorien von Quellenangaben}
\label{sec:BibKategorien}

Für geteilte Quellenverzeichnisse sind in dieser Vorlage folgende drei Kategorien vorgesehen
(s.\ Tabelle \ref{tab:QuellenUndEintragstypen}):%
\footnote{Diese sind in der Datei \nolinkurl{hgbthesis.cls} definiert.
Allfällige Änderungen sowie die Definition zusätzlicher Kategorien sind 
bei Bedarf relativ leicht möglich.}
%
\begin{description}
	\item[\texttt{literature}] -- für alle Quellen, die in gedruckter Form vorliegen;
	\item[\texttt{avmedia}] -- für Filme, audio-visuelle Medien und Computer Games (auf DVD, CD, \usw);
	\item[\texttt{online}] -- für Werke, die \emph{ausschließlich} online verfügbar sind.
\end{description}
%
Jedes Quellenobjekt wird aufgrund des angegebenen BibTeX-Eintragtyps 
\texttt{@\emph{type}} automatisch einer dieser Kategorien 
zugeordnet (s.\ Tabelle~\ref{tab:BibKategorien}).
Angeführt sind hier nur die wichtigsten Eintragstypen, die allerdings die meisten
Fälle in der Praxis abdecken sollten und nachfolgend durch Beispiele erläutert sind.
Alle nicht explizit angegebenen Einträge werden grundsätzlich in der Kategorie \emph{literature} 
ausgegeben.

\begin{table}
\caption{Quellenarten und empfohlene BibTeX-Eintragstypen.}
\label{tab:QuellenUndEintragstypen}
\centering
\begin{tabular}{llc}
	\hline
	Kategorie \emph{literature} & \texttt{@\emph{type}} & Seite\\
	\hline
	Buch (Textbuch, Monographie) & \texttt{@book} & \pageref{sec:@book}\\
	Sammelband (Hrsg.\ + mehrere Autoren) & \texttt{@incollection} & \pageref{sec:@incollection} \\
	Konferenz-, Tagungsband & \texttt{@inproceedings} & \pageref{sec:@inproceedings}\\
	Beitrag in Zeitschrift, Journal & \texttt{@article} & \pageref{sec:@article}\\
	Bachelor-, Master-, Diplomarbeit, Dissertation & \texttt{@thesis} & \pageref{sec:@thesis}\\
	Technischer Bericht, Laborbericht & \texttt{@techreport} & \pageref{sec:@techreport}\\
	Handbuch, Produktbeschreibung, Norm & \texttt{@manual} & \pageref{sec:@manual}\\
	Gesetzestext, Verordnung etc. & \texttt{@misc} & \pageref{sec:@misc}\\
%
	\hline
	Kategorie \emph{avmedia} & \\
	\hline
	Audio (CD) & \texttt{@audio} & \pageref{sec:@audio}\\
	Video (auf VHS, DVD, Blu-ray Disk) & \texttt{@video} & \pageref{sec:@video}\\
	Film (Kino) & \texttt{@movie} & \pageref{sec:@movie}\\
	Computer Game, Softwareprodukt & \texttt{@software} & \pageref{sec:@software}\\
%
	\hline
	Kategorie \emph{online} & \\
	\hline
	Webseite, Wiki-Eintrag, Blog etc. & \texttt{@online} & \pageref{sec:@online-www}\\
	Online-Video, Bild, Grafik & \texttt{@online} & \pageref{sec:@online-video}\\
\end{tabular}
\end{table}


\begin{table}
\caption{Kategorien von Quellenangaben und zugehörige BibTeX-Eintragstypen.
Bei geteiltem Quellenverzeichnis werden die Einträge jeder Kategorie in einem
eigenen Abschnitt gesammelt.
Grau gekennzeichnete Elemente sind Synonyme für die jeweils darüber stehenden Typen.}
\label{tab:BibKategorien}
\centering
\definecolor{midgray}{gray}{0.5}
\setlength{\tabcolsep}{6mm}
\begin{tabular}{lll}
	\emph{literature} & \emph{avmedia} & \emph{online} \\
	\hline
	\texttt{@book}          & \texttt{@audio}                & \texttt{@online} \\
	\texttt{@incollection}  & \texttt{\color{midgray}@music} & \texttt{\color{midgray}@electronic} \\
	\texttt{@inproceedings} & \texttt{@video}                & \texttt{\color{midgray}@www} \\
	\texttt{@article}       & \texttt{@movie}                &  \\
	\texttt{@thesis}        & \texttt{@software}             &  \\
	\texttt{@techreport}    &  &  \\
	\texttt{@manual}        &  &  \\
	\texttt{@misc}          &  &  \\
	\ldots                  &  &  \\
	\hline
\end{tabular}
\end{table}

%%------------------------------------------------------

\subsection{Gedruckte Quellen (\emph{literature})}
\label{sec:KategorieLiterature}

Diese Kategorie umfasst alle Werke, die in gedruckter Form publiziert wurden,
also beispielsweise in Büchern, Konferenzbänden, Zeitschriftenartikeln, Diplomarbeiten \usw
In den folgenden Beispielen ist jeweils der BibTeX-Eintrag in der Datei \nolinkurl{literatur.bib}
angegeben, gefolgt vom zugehörigen Ergebnis im Quellenverzeichnis.


\subsubsection{\texttt{@book}}
\label{sec:@book}
Ein einbändiges Buch (Monographie), das von einem Autor oder mehreren Autoren zur Gänze gemeinsam verfasst und (typischerweise) von einem Verlag herausgegeben wurde.
% 
\begin{itemize}
\item[] 
\begin{GenericCode}[numbers=none]
@book{BurgerBurge2006,
    author={Burger, Wilhelm and Burge, Mark},
    title={Digitale Bildverarbeitung},
	  subtitle={Eine Einführung mit Java und ImageJ},
    publisher={Springer-Verlag},
    location={Heidelberg},
    edition={2},
    year={2005},
    hyphenation={german}
}
\end{GenericCode}
\item[\cite{BurgerBurge2006}]
Wilhelm Burger und Mark Burge. \textit{Digitale Bildverarbeitung. Eine
Einführung mit Java und ImageJ}. 2.\ Aufl.\ Heidelberg: Springer-Verlag,
2006.
\end{itemize}
%\printbibliography[keyword=bookexample1,heading=noheader]\nocite{BurgerBurge2006}
%
\emph{Hinweis:} Die Auflagennummer (\texttt{edition}) wird üblicherweise nur angegeben, 
wenn es \emph{mehr} als eine Ausgabe gibt -- also insbesondere \textbf{nicht für die 1.\ Auflage}, 
wenn diese die einzige ist!
\textbf{ISBN-Nummern} sollte man auch getrost \textbf{weglassen}.

%%------------------------------------------------------

\subsubsection{\texttt{@incollection}}
\label{sec:@incollection}
Ein in sich abgeschlossener und mit einem eigenen Titel versehener
Beitrag eines oder mehrerer Autoren in einem Buch oder Sammelband.
Dabei ist \texttt{title} der Titel des Beitrags, \texttt{booktitle} der Titel des Sammelbands und
\texttt{editor} der Name des Herausgebers.
%
\begin{itemize}
\item[] 
\begin{GenericCode}[numbers=none]
@incollection{BurgeBurger1999,
  author={Burge, Mark and Burger, Wilhelm},
  title={Ear Biometrics},
  booktitle={Biometrics: Personal Identification in Networked Society},
  publisher={Kluwer Academic Publishers},
  year={1999},
  location={Boston},
  editor={Jain, Anil K. and Bolle, Ruud and Pankanti, Sharath},
  chapter={13},
  pages={273-285},
  hyphenation={english}
}
\end{GenericCode}
\item[\cite{BurgeBurger1999}]
Mark Burge und Wilhelm Burger. "`Ear Biometrics"'. In:\ \textit{Biometrics:
Personal Identification in Networked Society}. Hrsg.\ von Anil K.\ Jain,
Bolle Ruud und Pankanti Sharath. Boston:\ Kluwer Academic Publis-
hers, 1999. Kap.\ 13, S.\ 273--285.
\end{itemize}
%\printbibliography[keyword=incollectionexample1,heading=noheader]\nocite{BurgeBurger1999}

%%------------------------------------------------------

\subsubsection{\texttt{@inproceedings}}
\label{sec:@inproceedings}
Konferenzbeitrag, individueller Beitrag in einem Tagungsband.
Man beachte die Verwendung des neuen Felds \texttt{venue}
zur Angabe des Tagungsorts und 
\texttt{location} für den Ort der Publikation (des Verlags).

\begin{itemize}
\item[]
\begin{GenericCode}[numbers=none]
@inproceedings{Burger1987,
	author={Burger, Wilhelm and Bhanu, Bir},
	title={Qualitative Motion Understanding},
	booktitle={Proceedings of the Intl.\ Joint Conference on Artificial Intelligence},
	year={1987},
	month={5},
	editor={McDermott, John P.},
	venue={Mailand},
	publisher={Morgan Kaufmann Publishers},
	location={San Francisco},
	pages={819-821},
	hyphenation={english}
}
\end{GenericCode}
\item[\cite{Burger1987}]
Wilhelm Burger und Bir Bhanu. "`Qualitative Motion Understanding"'.
In:\ \textit{Proceedings of the Intl.\ Joint Conference on Artificial Intelligence}.
(Mailand). Hrsg.\ von John P.\ McDermott. San Francisco:\ Morgan
Kaufmann Publishers, Mai 1987, S.\ 819--821.
\end{itemize}
%\printbibliography[keyword=inproceedingsexample1,heading=noheader]\nocite{Burger1987}

%%------------------------------------------------------

\subsubsection{\texttt{@article}}
\label{sec:@article}
Beitrag in einer Zeitschrift, einem wissenschaftlichen Journal oder einer Tageszeitung.
Dabei steht \texttt{volume} üblicherweise für den Jahrgang und \texttt{number} für die 
Nummer innerhalb des Jahrgangs. Der Zeitschriftennamen (\texttt{journal} oder
\texttt{journaltitle}) sollte nur in begründeten Fällen abgekürzt werden, um Missverständnisse
zu vermeiden.
%
\begin{itemize}
\item[]
\begin{GenericCode}[numbers=none]
@article{Mermin1989,
	author={Mermin, Nathaniel David},
	title={What's wrong with these equations?},
	journal={Physics Today},
	volume={42},
	number={10},
	year={1989},
	pages={9-11},
	hyphenation={english}
}
\end{GenericCode}
\item[\cite{Mermin1989}]
Nathaniel David Mermin. "`What's wrong with these equations?"' In:\ \textit{Physics
Today} 42.10 (1989), S. 9--11.
\end{itemize}
%\printbibliography[keyword=articleexample1,heading=noheader]\nocite{Mermin1989}
%
\emph{Hinweis:} Die Angabe einer Ausgabe für \emph{mehrere} Monate ist in \texttt{biblatex} nicht mehr
über das Feld \texttt{month} möglich, denn dieses darf nur mehr \emph{einen} Wert enthalten.
In diesem Fall kann jedoch einfach das \texttt{issue}-Feld verwendet (\zB\ \verb!issue={Mai/Juni}!
im BibTeX-Eintrag zu \cite{Guttman2001}) und das \texttt{number}-Feld weggelassen werden.


%%------------------------------------------------------

\subsubsection{\texttt{@thesis}}
\label{sec:@thesis}
Dieser (neue) Eintragstyp kann allgemein für akademische Abschlussarbeiten verwendet werden. Er ersetzt
insbesondere die bekannten BibTeX-Einträge \texttt{@phdthesis} (für Dissertationen) sowie
\texttt{@mastersthesis} (für Di\-plom- und Masterarbeiten), die allerdings weiterhin verwendet werden können. Zusätzlich ist damit etwa auch die Angabe von Bachelorarbeiten möglich.

\bigskip % kludge to force heading to next page
\paragraph{Dissertation (Doktorarbeit):}
\begin{itemize}
\item[]
\begin{GenericCode}[numbers=none]
@thesis{Eberl1987,
	author={Eberl, Gerhard},
	title={Automatischer Landeanflug durch Rechnersehen},
	type={phdthesis},
	year={1987},
	month={8},
	institution={Universität der Bundeswehr, Fakultät für Raum- und Luftfahrttechnik},
	location={München},
	hyphenation={german}
}
\end{GenericCode}
\item[\cite{Eberl1987}]
Gerhard Eberl. "`Automatischer Landeanflug durch Rechnersehen"'.
Diss.\ München:\ Universität der Bundeswehr, Fakultät für Raum- und
Luftfahrttechnik, Aug.\ 1987.
\end{itemize}
%\printbibliography[keyword=phdthesisexample1,heading=noheader]\nocite{Eberl1987}

\paragraph{Magister- oder Masterarbeit:} ~ \newline
Analog zur Dissertation (s.\ oben), allerdings mit \texttt{type=\{mathesis\}}.%
%\footnote{Man beachte dabei, dass offiziell (zumindest in Österreich) auch die 
%Abschlussarbeiten von \emph{Master}studiengängen weiterhin als "`Diplomarbeiten"' 
%bezeichnet werden.} % ab März 2012 gilt das nicht mehr

%%------------------------------------------------------

\paragraph{Diplomarbeit:} ~ \newline
Analog zur Dissertation (s.\ oben), allerdings mit \texttt{type=\obnh\{Diplomarbeit\}}:%
%
\begin{itemize}
\item[]
\begin{GenericCode}[numbers=none]
@thesis{Artner2007,
	author={Artner, Nicole Maria},
	title={Analyse und Reimplementierung des Mean-Shift Tracking-Verfahrens},
	type={Diplomarbeit},
	year={2007},
	month={7},
	institution={University of Applied Sciences Upper Austria, Digitale Medien},
	location={Hagenberg, Austria},
	url={http://theses.fh-hagenberg.at/thesis/Artner07},
  hyphenation={german}
}
\end{GenericCode}
\item[\cite{Artner2007}]
Nicole Maria Artner. "`Analyse und Reimplementierung des Mean-Shift
Tracking-Verfahrens"'. Diplomarbeit. Hagenberg, Austria:\ Upper 
Austria University of Applied Sciences, Digitale Medien, Juli 2007. 
\textsc{url}:\ \url{http://theses.fh-hagenberg.at/thesis/Artner07}.
\end{itemize}
%\printbibliography[keyword=diplomarbeitexample1,heading=noheader]\nocite{Artner2007}
%
Der Inhalt des Felds \verb!url={..}! wird dabei automatisch und ohne zusätzliche
Kennzeichnung als URL gesetzt (mit dem \verb!\url{..}! Makro).

%%------------------------------------------------------

\paragraph{Bachelorarbeit:} ~ \newline
Bachelorarbeiten gelten in der Regel zwar nicht als "`richtige"' Publikationen, bei Bedarf müssen sie aber dennoch referenziert werden können. 
%
\begin{itemize}
\item[]
\begin{GenericCode}[numbers=none]
@thesis{Bacher2004,
	author={Bacher, Florian},
	title={Interaktionsmöglichkeiten mit Bildschirmen und großflächigen Projektionen},
	type={Bachelorarbeit},
	year={2004},
	month={6},
	institution={Upper Austria University of Applied Sciences, Medientechnik und {-design}},
	location={Hagenberg, Austria},
	hyphenation={german},
	keywords={bachelorarbeitexample1} 
}
\end{GenericCode}
\item[\cite{Bacher2004}]
Florian Bacher. "`Interaktionsmöglichkeiten mit Bildschirmen und groß-
flächigen Projektionen"'. Bachelorarbeit. Hagenberg, Austria:\ Upper
Austria University of Applied Sciences, Medientechnik und {-design},
Juni 2004.
\end{itemize}
%\printbibliography[keyword=bachelorarbeitexample1,heading=noheader]\nocite{Bacher2004}
%


%%------------------------------------------------------

\subsubsection{\texttt{@techreport}}
\label{sec:@techreport}
Das sind typischerweise nummerierte Berichte (\emph{technical reports}) aus Unternehmen, 
Hochschulinstituten oder Forschungsprojekten.
Wichtig ist, dass die herausgebende Organisationseinheit (Firma, Institut, Fakultät \etc) und 
Adresse angegeben wird. Sinnvollerweise wird auch der zugehörige URL angegeben, sofern vorhanden. 
%
\begin{itemize}
\item[]
\begin{GenericCode}[numbers=none]
@techreport{Drake1948,
  author={Drake, Huber M. and McLaughlin, Milton D. and Goodman, Harold R.},
  title={Results obtained during accelerated transonic tests of the {Bell} {XS-1} airplane in flights to a {MACH} number of 0.92},
  institution={NASA Dryden Flight Research Center},
  year={1948},
  month={1},
  location={Edwards, CA},
  number={NACA-RM-L8A05A},
  url={http://www.nasa.gov/centers/dryden/pdf/...05A.pdf},
  hyphenation={english}
}
\end{GenericCode}
\item[\cite{Drake1948}]
Huber M.\ Drake, Milton D.\ McLaughlin und Harold R.\ Goodman.
\textit{Results obtained during accelerated transonic tests of the Bell XS-1
airplane in flights to a MACH number of 0.92}. Techn.\ Ber.\ NACA-
RM-L8A05A. Edwards, CA:\ NASA Dryden Flight Research Center,
Jan. 1948. \textsc{url}:\ \url{http://www.nasa.gov/centers/dryden/pdf/87528main_RM-L8A05A.pdf}.
\end{itemize}
%\printbibliography[keyword=techreportexample1,heading=noheader]\nocite{Drake1948}

%%------------------------------------------------------

\subsubsection{\texttt{@manual}}
\label{sec:@manual}
Dieser Publikationstyp bietet sich für Dokumente an, bei denen üblicherweise kein Autor genannt ist, wie etwa bei Produktbeschreibungen von Herstellern, Anleitungen, Präsentationen, Normen, White Papers \usw
%
\begin{itemize}
\item[]
\begin{GenericCode}[numbers=none]
@manual{AMS2002,
	author={{American Mathematical Society}},
	title={User's Guide for the {\tt amsmath} Package},
	year={2002},
	month={2},
	version={2.0},
	url={ftp://ftp.ams.org/pub/tex/doc/amsmath/amsldoc.pdf},
	hyphenation={english}
}
\end{GenericCode}
\item[\cite{AMS2002}]
American Mathematical Society. 
\textit{User's Guide for the {\tt amsmath} Package}. 
Version 2.0. Feb. 2002. \textsc{url}: \url{ftp://ftp.ams.org/pub/tex/doc/amsmath/amsldoc.pdf}.
\end{itemize}
%\printbibliography[keyword=manualexample1,heading=noheader]\nocite{AMS2002}
%
In obigem Beispiel gilt zu beachten: 
Da kein Autor bekannt ist, wird hier der Name des \emph{Unternehmens} oder der \emph{Institution} im \texttt{author}-Feld angegeben, allerdings innerhalb einer \textbf{zusätzlichen Klammer} \texttt{\{..\}}, damit das Argument nicht fälschlicherweise als \emph{Vorname}+\emph{Nachname} interpretiert wird.%
\footnote{Im Unterschied zu BibTeX wird in \texttt{biblatex} bei \texttt{@manual}-Einträgen das Feld \texttt{organization} nicht als Ersatz für \texttt{author} akzeptiert.}


%%------------------------------------------------------
\subsubsection{\texttt{@misc}}
\label{sec:@misc}
Sollte mit den bisher angeführten Eintragungstypen für gedruckte Publikationen
nicht das Auslangen gefunden werden, sollte man sich zunächst die weiteren (hier nicht näher beschriebenen) 
Typen im \texttt{biblatex}-Handbuch \cite{Lehman2015} ansehen, beispielsweise
\texttt{@collection} für einen Sammelband als Ganzes (also nicht nur ein Beitrag darin)
oder \texttt{@patent} für ein Patent oder eine Patentanmeldung.

Wenn nichts davon passt, dann kann auf den Typ \texttt{@misc} zurückgegriffen werden, der ein
Textfeld \texttt{howpublished} vorsieht, in dem die Art der Publikation individuell 
angegeben werden kann. Das folgende Beispiel zeigt die Anwendung für einen Gesetzestext 
(\sa\ \cite{FhStG1993} und \cite{EuRichtlinie2000}).
%
\begin{itemize}
\item[]
\begin{GenericCode}[numbers=none]
@misc{OoeRaumordnungsgesetz1994,
	title={Oberösterreichisches Raumordnungsgesetz 1994},
	howpublished={LGBl 1994/114 idF 1995/93},
	url={http://www.ris.bka.gv.at/Dokumente/LrOO/...538.pdf},
	hyphenation={german}
}
\end{GenericCode}
\item[\cite{OoeRaumordnungsgesetz1994}]
\textit{Oberösterreichisches Raumordnungsgesetz 1994}. LGBl 1994/114 idF
1995/93. \textsc{url}: \url{http://www.ris.bka.gv.at/Dokumente/LrOO/LOO40007538/LOO40007538.pdf}.
\end{itemize}
%\printbibliography[keyword=miscexample1,heading=noheader]\nocite{OoeRaumordnungsgesetz1994}
%



\subsection{Filme und audio-visuelle Medien (\emph{avmedia})}
\label{sec:KategorieAvmedia}

Diese Kategorie ist dazu vorgesehen, audio-visuelle Produktionen wie Filme, 
Tonaufzeichnungen, Audio-CDs, DVDs, VHS-Kassetten \usw\ zu erfassen.
Damit gemeint sind Werke, die in physischer (jedoch nicht in gedruckter) Form
veröffentlicht wurden.
Nicht gemeint sind damit audio-visuelle Werke (Tonaufnahmen, Bilder, Videos) 
die ausschließlich online verfügbar sind -- diese sollten mit einem Elementtyp 
\texttt{@online} (s.\ Tabelle~\ref{tab:BibKategorien} und Abschnitt~\ref{sec:KategorieOnline}) ausgezeichnet werden.

Die nachfolgend beschriebenen Typen \texttt{@audio}, \texttt{@video} und \texttt{@movie} 
sind \emph{keine} BibTeX-Standardtypen. Sie sind aber in \texttt{biblatex} vorgesehen
(und implizit durch \texttt{@misc} ersetzt) und werden hier empfohlen, um die automatische 
Gliederung des Quellenverzeichnisses zu ermöglichen.

\subsubsection{\texttt{@audio}}
\label{sec:@audio}
Hier ein Beispiel für die Spezifikation einer Audio-CD:
%
\begin{itemize}
\item[] 
\begin{GenericCode}[numbers=none]
@audio{Zappa1995,
  author={Zappa, Frank},
  title={Freak Out},
  howpublished={Audio-CD},
  year={1995},
  month={5},
  note={Rykodisc, New York},
  hyphenation={english},
  keywords={audioexample1}
}
\end{GenericCode}
\item[\cite{Zappa1995}]
Frank Zappa. \textit{Freak Out}. Audio-CD. Rykodisc, New York. Mai 1995.
\end{itemize}
%\printbibliography[keyword=audioexample1,heading=noheader]\nocite{Zappa1995}
%
Anstelle von \verb!howpublished={Audio-CD}! könnte auch 
\verb!type={audiocd}! verwendet werden.


\subsubsection{\texttt{@video}}
\label{sec:@video}
Hier ein Beispiel für die Spezifikation einer DVD-Edition:
%
\begin{itemize}
\item[] 
\begin{GenericCode}[numbers=none]
@video{Futurama1999,
  author={Groening, Matt},
  title={Futurama},
	titleaddon={Season 1 Collection},
  howpublished={DVD},
  year={2002},
  month={2},
  note={Twentieth Century Fox Home Entertainment},
  hyphenation={english}
 }
\end{GenericCode}
\item[\cite{Futurama1999}]
Matt Groening. \textit{Futurama}. Season 1 Collection. DVD. 
Twentieth Century Fox Home Entertainment. Feb.\ 2002.
\end{itemize}
%\printbibliography[keyword=videoexample1,heading=noheader]\nocite{Futurama1999}
%
In diesem Fall ist das angegebene Datum der \emph{Erscheinungstermin}. 
Falls kein eindeutiger Autor namhaft gemacht werden kann, lässt man das
\texttt{author}-Feld weg und verpackt die entsprechenden Angaben im \texttt{note}-Feld, wie im nachfolgenden Beispiel gezeigt.


\subsubsection{\texttt{@movie}}
\label{sec:@movie}
Dieser Eintragstyp ist für Filme reserviert. 
Hier wird von vornherein \emph{kein} Autor angegeben, weil dieser bei 
einer Filmproduktion \ia\ nicht eindeutig zu benennen ist. 
Im folgenden Beispiel (\sa\ \cite{Psycho1960}) sind die betreffenden Daten 
im \texttt{note}-Feld angegeben:%
\footnote{Übrigens achtet \texttt{biblatex} netterweise darauf, dass der  
Punkt am Ende des \texttt{note}-Texts in der Ausgabe nicht verdoppelt wird.}
%
\begin{itemize}
\item[] 
\begin{GenericCode}[numbers=none]
@movie{Nosferatu1922,
    title={Nosferatu -- A Symphony of Horrors},
    howpublished={Film},
    year={1922},
    note={Drehbuch/Regie: F. W. Murnau. Mit Max Schreck, Gustav von Wangenheim, Greta Schröder.},
    hyphenation={english}
}
\end{GenericCode}
\item[\cite{Nosferatu1922}]
\textit{Nosferatu -- A Symphony of Horrors}. Film. 
Drehbuch/Regie:\ F.\ W.\ Murnau. 
Mit Max Schreck, Gustav von Wangenheim, Greta Schröder.
1922.
\end{itemize}
%\printbibliography[keyword=movieexample1,heading=noheader]\nocite{Nosferatu22}
%
Die Angabe \verb!howpublished={Film}! ist hier sinnvoll, um die Verwechslung
mit einem (möglicherweise gleichnamigen) Buch auszuschließen.



\subsubsection{\texttt{@software}}
\label{sec:@software}
Dieser Eintragstyp ist insbesondere für Computerspiele geeignet (in Ermangelung
eines eigenen Eintragstyps). Hier ein konkretes Beispiel \cite{LegendOfZelda1998}:
%
\begin{itemize}
\item[] 
\begin{GenericCode}[numbers=none]
@software{LegendOfZelda1998,
   author={Miyamoto, Shigeru and Aonuma, Eiji and Koizumi, Yoshiaki},
   title={The Legend of Zelda: Ocarina of Time},
   howpublished={N64-Spielmodul},
   publisher={Nintendo},
   year={1998},
   hyphenation={english}
}
\end{GenericCode}
\end{itemize}



\subsubsection{Zeitangaben zu Musikaufnahmen und Filmen} 

Einen Verweis auf eine bestimmten Stelle in einem Musikstück oder Film kann man 
ähnlich ausführen wie die Seitenangabe in einem Druckwerk.
Besonders legendär (und häufig parodiert) ist beispielsweise die Duschszene
in \emph{Psycho} \cite[T=00:32:10]{Psycho1960}.
Alternativ zur simplen Zeitangabe "`T=\emph{hh}:\emph{mm}:\emph{ss}"' 
könnte man eine bestimmte Stelle auch auf den Frame genau durch 
den zugehörigen \emph{Timecode} "`TC=\emph{hh:mm:ss:ff}"' angeben, 
\zB\ \cite[TC=00:32:10:12]{Psycho1960} für Frame \emph{ff}=12.



\subsection{Online-Quellen (\emph{online})}
\label{sec:KategorieOnline}

Bei Verweisen auf Online-Resourcen sind grundsätzlich drei Fälle zu unterscheiden:
%
\begin{itemize}
\item[A.] Man möchte allgemein auf eine Webseite verweisen, etwa auf die 
	"`Panasonic products for business"' Seite.%
	\footnote{\url{http://business.panasonic.co.uk/}}
	In diesem Fall wird nicht auf ein konkretes "`Werk"' verwiesen und daher
	erfolgt \emph{keine} Aufnahme ins Quellenverzeichnis. Stattdessen
	genügt eine einfache Fußnote mit \verb!\footnote{\url{..}}!, wie im vorigen
	Satz gezeigt.
\item[B.] Ein gedrucktes oder audio-visuelles Werk 
	(s.\ Abschnitte \ref{sec:KategorieLiterature}--\ref{sec:KategorieAvmedia})
	ist \emph{zusätzlich} auch online verfügbar. In diesem Fall ist die Primär\-publikation 
	aber \emph{nicht} "`online"' und es genügt, ggfs.\ den zugehörigen Link im 
	\texttt{url}-Feld anzugeben, das bei jedem Eintragstyp zulässig ist.
\item[C.] Es handelt sich im weitesten Sinn um ein Werk, das aber 
	\emph{ausschließlich} online verfügbar ist, wie \zB\ ein Wiki-Eintrag, 
	ein digitales Bild,	eine Software, ein YouTube-Video oder ein Blog-Eintrag.
	Die Kategorie \emph{online} ist genau (und \emph{nur}) für diese 
	Art von Quellen vorgesehen.
\end{itemize}




\subsubsection{Beispiel: Wiki-Eintrag}
\label{sec:@online-www}
Durch den Umfang und die steigende Qualität dieser Einträge erscheint
die Aufnahme in das Quellenverzeichnis durchaus berechtigt.
Beispielsweise bezeichnet man als "`Reliquienschrein"'
einen Schrein, in dem die Reliquien eines oder 
mehrerer Heiliger aufbewahrt werden \cite{WikiReliquienschrein2015}.
%
\begin{itemize}
\item[]
\begin{GenericCode}[numbers=none]
@online{WikiReliquienschrein2015,
	url={http://de.wikipedia.org/wiki/Reliquienschrein},
	urldate={2015-12-09}
}
\end{GenericCode}
\item[\cite{WikiReliquienschrein2015}]
\textsc{url}: \url{http://de.wikipedia.org/wiki/Reliquienschrein}
(besucht am 20.11.{\hskip0pt}2014).
\end{itemize}
%\printbibliography[keyword=onlineexample2,heading=noheader]\nocite{WikiReliquienschrein2015}
%
In diesem Fall besteht die Quellenangabe praktisch nur mehr aus dem URL.
Durch die (optionale) Angabe von \texttt{urldate} (im \texttt{YYYY-MM-DD} Format) wird automatisch die Information eingefügt, wann das Online-Dokument eingesehen wurde.


\subsubsection{Beispiel: online Video} 
\label{sec:@online-video}
%
\begin{itemize}
\item[]
\begin{GenericCode}[numbers=none]
@online{HistoryOfComputers2008,
    title={History of Computers},
    year={2008},
		month={9},
    url={http://www.youtube.com/watch?v=LvKxJ3bQRKE},
    hyphenation={english}
}
\end{GenericCode}
\item[\cite{HistoryOfComputers2008}]
\textit{History of Computers}. Sep. 2008. 
\textsc{url}: \url{http://www.youtube.com/watch?v=LvKxJ3bQRKE}.
\end{itemize}
%\printbibliography[keyword=onlineexample1,heading=noheader]\nocite{HistoryOfComputers2008}
%
Bis auf das Feld \texttt{url} können alle Angaben weggelassen werden. Die Angabe von 
weiteren Details (\zB\ \texttt{author}) ist natürlich ebenso möglich.

\subsubsection{Beispiel: Bildquelle}

In Abbildungen wird sehr häufig fremdes Bildmaterial verwendet, dessen Herkunft natürlich 
in jedem Fall angegeben werden sollte. Die Angabe von URLs unmittelbar in den Captions von Abbildungen
ist problematisch, weil sie meistens für das Schriftbild ziemlich störend sind.
Einfacher ist es, auch einzelne Bilder und Grafiken ins Quellenverzeichnis aufzunehmen und
wie üblich mit dem \verb!\cite{..}! Kommando darauf zu verweisen.
Beispiele dazu sind die Angaben zu Abb.\ \ref{fig:CocaCola}--\ref{fig:ibm360}.
%
\begin{itemize}
\item[]
\begin{GenericCode}[numbers=none]
@online{IBM360,
	url={http://www.plyojump.com/classes/mainframe_era.php}
}
\end{GenericCode}
\item[\cite{IBM360}]
\textsc{url}: \url{http://www.plyojump.com/classes/mainframe_era.php}.
\end{itemize}
%\printbibliography[keyword=onlineexample3,heading=noheader]\nocite{IBM360}
%

\subsection{Elektronische Datenträger als Ergänzung zur Arbeit}

Wird der Abschlussarbeit ein elektronischer Datenträger (CD-ROM, DVD
etc.) beigelegt, empfiehlt sich die angeführten Webseiten in
elektronischer Form (vorzugsweise als PDF-Da\-tei\-en) abzulegen
und die zugehörige Quelle im Quellenverzeichnis mit einem 
entsprechenden Verweis im \texttt{note}-Feld -- \zB\
"`Kopie auf CD-ROM (Datei \nolinkurl{xyz.pdf})"' --
zu versehen.
Für die Angabe von solchen Dateinamen, die nicht als Online-Link
zu öffnen sind, ist übrigens die Verwendung von 
\verb!\nolinkurl{...}! anstelle von \verb!\url{...}! zu empfehlen.


\subsection{Tipps zur Erstellung von BibTeX-Dateien}
\label{sec:TippsZuBibtex}

Die folgenden Dinge sollten bei der Erstellung korrekter BibTeX-Dateien beachtet werden.

\subsubsection{\texttt{month}-Attribut}

Das \texttt{month}-Attribut ist in \texttt{biblatex} (im Unterschied zu BibTeX) numerisch
und wird beispielsweise einfach in der Form \verb!month={8}! (für den Monat August)
angegeben.

\subsubsection{\texttt{language}/\texttt{hyphenation}-Attribut}

Das \texttt{language} Attribut ermöglichte im Zusammenhang mit dem \texttt{babelbib}-Paket 
den korrekten Satz mehrsprachiger Quellenverzeichnisse. \texttt{babelbib} wird in 
dieser Vorlage \emph{nicht} mehr verwendet, sondern durch \texttt{biblatex} \cite{Lehman2015} ersetzt.
Hier wird allerdings das \texttt{language}-Feld anders interpretiert und
stattdessen das \texttt{hyphenation}-Feld verwendet, das nach Möglichkeit 
bei jedem Quelleneintrag angegeben werden sollte, also beispielsweise
\begin{quote}
\verb!hyphenation={german}! \quad oder \quad \verb!hyphenation={english}!
\end{quote}
für ein deutsch- \bzw\ englischsprachiges Dokument.

\subsubsection{\texttt{edition}-Attribut}

Im Unterschied zu BibTeX ist mit \texttt{biblatex} auch das \texttt{edition}-Feld numerisch.
Es ist lediglich die Nummer selbst anzugeben, also etwa
\verb!edition={3}!
bei einer dritten Auflage. Die richtige Interpunktion in der Quellenangabe wird in Abhängigkeit von der Spracheinstellung automatisch hinzugefügt 
("`3.\ Auflage"' \bzw\ "`3rd edition"').

Wie bereits auf Seite \pageref{sec:@book} angemerkt, sollte im Fall einer
\textbf{1.~Auflage} das \texttt{edition}-Feld \textbf{nicht} verwendet werden,
wenn es ohnehin keine andere Auflage gibt!


\subsubsection{Übernahme von fertigen BibTeX-Einträgen}

Viele Verlage und Literatur-Broker bieten fertige BibTeX-Einträge zum Herunterladen an.
Dabei ist jedoch größte Vorsicht geboten, denn diese Einträge sind häufig
unvollständig oder syntaktisch fehlerhaft!
Sie sollten bei der Übernahme \emph{immer} auf Korrektheit überprüft werden!
Besonders sollte dabei auf die korrekte Angabe der Namen, also 
\texttt{author=\{\textit{Nachname1}, \textit{Vorname1a} \emph{Vorname1b} 
and \textit{Nachname2}, \textit{Vorname2a} \ldots \}}%
\footnote{Das ist \va\ bei mehrteiligen Nachnamen wichtig, weil in diesem Fall
Vor- und Nachnamen nicht korrekt zugeordnet werden können, \zB\ 
\texttt{author=\{van Beethoven, Ludwig and ter Linden, Jaap\}}
für ein hypothetisches Werk von Ludwig van Beethoven und Jaap ter Linden.}
geachtet werden, sowie die Angabe von \texttt{volume}, \texttt{number} und \texttt{pages}.
Die Namen von Konferenzen sind oft völlig falsch (auch bei ACM und IEEE).
ISBN- und ISSN-Nummern sind entbehrlich und können getrost weggelassen werden.
Gleiches gilt für DOI-Nummern -- aber das möge der Betreuer entscheiden.
 

\subsubsection{Listing aller Quellen}

Durch die Anweisung \verb!\nocite{*}! -- an beliebiger Stelle im Dokument platziert -- werden \emph{alle} bestehenden Einträge der BibTeX-Datei im Quellenverzeichnis aufgelistet, also auch jene, für die es keine explizite \verb!\cite{}! Anweisung gibt. Das ist ganz nützlich, um während des Schreibens der Arbeit eine aktuelle Übersicht auszugeben. Normalerweise müssen aber alle angeführten Quellen auch im Text referenziert sein!


\subsubsection{Ergänzungen zu \texttt{biblatex}}

\texttt{biblatex} gibt bei der Verarbeitung dieses Dokuments einige Warnungen aus ("`Package biblatex Warning"'), 
die in der Regel ignoriert werden können:
%\begin{small}
%\begin{verbatim}
  %Package biblatex Warning: Data encoding is 'latin1'.
  %(biblatex)                Use backend=bibtex8 or backend=biber.
%\end{verbatim}
%\end{small}
%Das ist ein Hinweis auf die neueren Bib-Prozessoren \texttt{bibtex8} \bzw\ \texttt{biber}, die
%mit Nicht-ASCII-Dateien besser umgehen können als das klassische \texttt{bibtex}-Programm.
\begin{small}
\begin{verbatim}
  Package biblatex Warning: No driver for entry type 'video'.
  (biblatex)                Using fallback driver ...
\end{verbatim}
\end{small}
Für die neuen Eintragstypen \texttt{audio}, \texttt{video} und \texttt{movie} sind derzeit 
noch keine eigenen "`driver"' hinterlegt. Stattdessen wird im Hintergrund der \texttt{misc}-Typ 
als "`fallback"' verwendet.

Wenn eine der drei Kategorien des Quellenverzeichnisses (s.~Abschnitt \ref{sec:BibKategorien})
ohne Eintrag ist, wird eine Warnung "`\texttt{Empty bibliography on input line \ldots}"' ausgegeben, die auch getrost ignoriert werden können.




\section{Plagiarismus}
\label{sec:Plagiarismus}

Als "`Plagiat"' bezeichnet man die Darstellung eines fremden Werks als eigene Schöpfung, 
in Teilen oder als Ganzes, egal ob bewusst oder unbewusst.
Plagiarismus ist kein neues Problem im Hochschulwesen, hat sich aber durch die 
breite Verfügbarkeit elektronischer Quellen in den letzten Jahren dramatisch 
verstärkt und wird keineswegs als Kavaliersdelikt betrachtet.
Viele Hochschulen bedienen sich als Gegenmaßnahme heute ebenfalls elektronischer Hilfsmittel 
(die den Studierenden zum Teil nicht zugänglich sind), und man sollte daher bei jeder 
abgegebenen Arbeit damit rechnen, dass sie routinemäßig auf Plagiatsstellen untersucht wird!
Werden solche erst zu einem späteren Zeitpunkt entdeckt, kann das im schlimmsten Fall sogar 
zur nachträglichen (und endgültigen) Aberkennung des akademischen Grades führen.

Um derartige Probleme zu vermeiden, sollte eher übervorsichtig agiert und zumindest folgende Regeln beachtet werden:
%
\begin{itemize}
\item
Die Übernahme kurzer Textpassagen ist nur unter korrekter Quellenangabe zulässig, wobei der Umfang (Beginn und Ende) des Textzitats in jedem einzelnen Fall klar erkenntlich gemacht werden muss. 
\item
Insbesondere ist es nicht zulässig, eine Quelle nur eingangs zu erwähnen und nachfolgend wiederholt nicht-ausgezeichnete Textpassagen als eigene Wortschöpfung zu übernehmen. 
\item
Auf gar keinen Fall tolerierbar ist die direkte oder paraphrasierte Übernahme längerer Textpassagen, ob mit oder ohne Quellenangabe. Auch indirekt übernommene oder aus einer anderen Sprache übersetzte Passagen müssen mit entsprechenden Quellenangaben gekennzeichnet sein! 
\end{itemize}
%
Im Zweifelsfall finden sich detailliertere Regeln in jedem guten Buch über wissenschaftliches Arbeiten oder man fragt sicherheitshalber den Betreuer der Arbeit.



\chapter{Drucken der Abschlussarbeit}
\label{cha:Drucken}


\section{PDF-Workflow}
\label{sec:pdf}

Heutzutage wird \latex\ praktisch immer so benutzt, dass damit direkt PDF-Dokumente
(ohne den früher üblichen Umweg über DVI und PostScript) erzeugt werden.
In modernen Editoren (\zB \emph{TeXstudio} oder \emph{Overleaf}) funktioniert dies
ohne weiteren Konfigurationsaufwand.


\subsection{PDF Archivformat (PDF/A)}
\label{sec:PDFA}

Viele Institutionen verlangen die Abgabe von Abschlussarbeiten im PDF/A-Format, einer
standardisierten Variante von PDF für die Langzeitarchivierung.%
\footnote{\url{https://de.wikipedia.org/wiki/PDF/A}}
Dieses Dokument wird standardmäßig im PDF/A-Format (PDF/A-2b, um genau zu sein),
aufgrund der Anweisung
%
\begin{LaTeXCode}[numbers=none]
\RequirePackage{hgbpdfa}
\end{LaTeXCode}
%
am Beginn der Datei \verb!main.tex! (wodurch \verb!hgbpdfa.sty! geladen wird).
Man beachte, dass diese Anweisung \emph{vor} der \verb!\documentclass!-Deklaration
platziert werden muss. Erforderliche Meta\-daten (\zB Autor und Titel) werden
automatisch aus den Dokumenteinstellungen übernommen und in das Ausgabe-PDF eingefügt.%
\footnote{Dieses Setup basiert auf neuer Funktionalität, die aktuell in den
\texttt{pdflatex}-Kern eingebaut wird und erfordert das Paket 
\texttt{pdfmanagement-testphase} in Version 0.95s (2022-09-26) oder höher.
Bei älteren Versionen (\zB zurzeit noch auf \emph{Overleaf}) wird eine Warnung
ausgegeben und keine PDF/A-konforme Ausgabe erzeugt.}


\subsection{PDF/A Problemstellen}
\label{sec:PDFA-issues}

Die Aktivierung der PDF/A-Option erzeugt eine Ausgabedatei, die \emph{vorgibt},
PDF/A-konform zu sein, was aber nicht bedeutet, dass sie es tatsächlich \emph{ist}.
Obwohl dieses Dokument ein PDF/A-konformes Dokument erzeugt, ist dies bei abgeleiteten
Dokumenten möglicherweise nicht der Fall. Es ist daher wichtig, die resultierende
PDF-Datei vor der Abgabe mit einer der unten beschriebenen Methoden zu 
\emph{validieren}. Die meisten Verletzungen des PDF/A-Standards entstehen durch 
die Einbindung anderer PDF-Dateien, insbesondere von Grafiken. Typische Probleme 
sind die Verwendung von nicht eingebetteten Schriftarten und falschen oder 
unerwünschten Farbräumen. Im aktuellen Setup wird von sRGB-Farben ausgegangen, die man
grundsätzlich auch bei der Erstellung eigener Illustrationen verwenden sollte.

Probleme mit importierten PDF-Dateien können im finalen (zusammengesetzten) Dokument
schwer zu lokalisieren sein. Wenn die problematische Datei bekannt ist und nicht 
neu generiert werden kann, lässt sie sich eventuell mit anderen Tools wie Adobe \emph{Acrobat} 
(\emph{Distiller}) oder \emph{Ghostscript}%
\footnote{\url{https://ghostscript.com/}}
reparieren.


\subsection{PDF/A Validierung}
\label{sec:PDFA-validation}

Eine einfache (und kostenlose) Methode zur Überprüfung der PDF/A-Konformität bietet
\textsf{veraPDF} in zwei Varianten:
%
\begin{itemize}
\item eine Open-Source-Validierungssoftware%
  \footnote{\url{https://verapdf.org/software} (Windows, macOS, Linux)} sowie
\item ein Online-Validierungsservice.%
  \footnote{\url{https://demo.verapdf.org}}
\end{itemize}
%
Abbildung \ref{fig:verapdf-report} zeigt ein Beispiel. Ein ähnliches Service bietet
auch \textsf{pdf-online.com},%
\footnote{\url{https://www.pdf-online.com/osa/validate.aspx}}
dessen Einstellung leider für 2023 angekündigt ist.
Natürlich ist die PDF/A-Validierung auch im Werkzeugsatz von Adobe \emph{Acrobat} enthalten.

\begin{figure}[htbp]
    \centering
    \fbox{\includegraphics[width=.60\textwidth]{verapdf-report}}
    \caption{Bericht, der vom \textsf{veraPDF}-Client nach erfolgreicher Validierung 
    \emph{dieses} Dokuments erstellt wurde. Man beachte, dass der als PDF importierte Screenshot 
    selbst \emph{nicht} PDF/A-konform ist.}
    \label{fig:verapdf-report}
\end{figure}


\section{Drucken}

Vor dem Drucken der Arbeit empfiehlt es sich, einige Dinge zu beachten, um
unnötigen Aufwand (und auch Kosten) zu vermeiden.

\subsection{Drucker und Papier}

Die Abschlussarbeit sollte in der Endfassung unbedingt auf einem qualitativ
hochwertigen \emph{Laserdrucker} ausgedruckt werden; Ausdrucke mit
Tintenstrahldruckern sind \emph{nicht} ausreichend. Auch das verwendete
Papier sollte von guter Qualität (holzfrei) und üblicher Stärke (typ.\  
$80\,\mathrm{g} / \mathrm{m}^2$) sein. Falls nor einzelne \emph{farbige} Seiten 
notwendig sind, kann man diese auch einzeln auf einem Farb-Laserdrucker ausdrucken
und dem übrigen (schwarz/weiß gedruckten) Dokument beifügen.

Übrigens sollten \emph{alle} abzugebenden Exemplare \emph{gedruckt} (und
nicht kopiert) werden! Die Kosten für den Druck sind nicht höher als die für
Kopien, der Qualitätsunterschied ist jedoch -- \va\ bei Bildern und Grafiken
-- meist deutlich.

\subsection{Druckgröße}

Zunächst sollte sichergestellt werden, dass die in der fertigen PDF-Datei
eingestellte Papiergröße tatsächlich \textrm{A4} ist! Das geht \zB\ mit
\emph{Adobe Acrobat} oder \emph{SumatraPDF} über \texttt{File} $\rightarrow$
\texttt{Properties}, wo die Papiergröße des Dokuments angezeigt wird:
\begin{center}
	\textrm{Richtig:} A4 = $8{,}27 \times 11{,}69$ in \bzw\ $210 \times 297$ mm.
\end{center}
Falls das nicht übereinstimmt, ist vermutlich irgendwo im Workflow versehentlich
"Letter" als Papierformat eingestellt.


Ein häufiger und leicht zu übersehender Fehler beim Ausdrucken von
PDF-Doku\-menten wird durch die versehentliche Einstellung der Option "Fit to
page" im Druckmenü verursacht, wobei die Seiten meist zu klein ausgedruckt
werden. Überprüfen Sie daher die Größe des Ausdrucks anhand der eingestellten
Textbreite%
\footnote{\Convert[unit=mm]{\the\textwidth}	im aktuellen Dokument} % using 'lengthconvert' package
oder mithilfe der Messgrafik am Ende dieses Dokuments gezeigt.
Sicherheitshalber sollte diese Messgrafik bis zur Fertigstellung der
Arbeit beibehalten und die entsprechende Seite erst ganz am Schluss zu
entfernt werden. Wenn, wie häufig der Fall, einzelne Seiten getrennt in Farbe
gedruckt werden, so sollten natürlich auch diese genau auf die Einhaltung der
Druckgröße kontrolliert werden!


\section{Binden der Arbeit}

Die Endfassung der Abschlussarbeit ist üblicherweise in fest gebundener Form
einzureichen.%
\footnote{Für \emph{Bachelorarbeiten} genügt, je nach Vorgaben des
Studiengangs, meist eine einfache Bindung (\zB\ durch einen guten Copyshop
oder die Bibliothek der Hochschule).}
Dabei ist eine Bindung zu verwenden, die das Ausfallen von einzelnen Seiten
nachhaltig verhindert, \zB durch eine traditionelle Rückenbindung
(Buchbinder*in) oder durch handelsübliche Klammerungen aus Kunststoff oder
Metall. Eine einfache Leimbindung ohne Verstärkung ist jedenfalls
\emph{nicht} ausreichend.%
\footnote{An der Fakultät Hagenberg ist bei \emph{Masterarbeiten} zumindest
eines der Exemplare \emph{ungebunden} abzugeben -- dieses wird später von
einem*einer Buchbinder*in in einheitlicher Form gebunden und verbleibt danach
in der Bibliothek.}

Falls man -- was sehr zu empfehlen ist -- die Arbeit bei einem*einer
professionellen Buchbinder*in durchführen lässt, sollte man auch auf die
Prägung am Buchrücken achten, die kaum zusätzliche Kosten verursacht. Üblich
ist dabei die Angabe des Familiennamens des*der Autors*Autorin und des Titels
der Arbeit. Ist der Name und/oder der Titel der Arbeit zu lang, muss man 
notfalls eine gekürzte Version angeben, wie \zB:
%
\begin{center}
	\setlength{\fboxsep}{3mm}
	\fbox{\textsc{Schlaumeier}
		\textperiodcentered\ \textsc{Part.\ Lösungen zur allg.\ Problematik}}
\end{center}
%
Nach dem Binden sollte man die fertige Arbeit unbedingt nochmals auf 
Vollständigkeit, korrekte Anordnung der Seites \etc\ überprüfen.



\begin{comment}	% this is outdated
\section{Elektronische Datenträger (CD-R, DVD)}

Speziell bei Arbeiten im Bereich der Informationstechnik (aber nicht nur
dort) fallen fast immer Informationen an, wie Programme, Daten, Grafiken,
Kopien von Internetseiten \usw, die für eine spätere Verwendung elektronisch
verfügbar sein sollten. Vernünftigerweise wird man diese Daten während der
Arbeit bereits gezielt sammeln und der fertigen Arbeit auf einer CD-ROM oder
DVD beilegen.%
\footnote{Als Alternative sehen Institute zunehmend den Upload dieser Daten
in ein entsprechendes Online-Archiv vor, zumal CD/DVD-Laufwerke in neuen
Geräten kaum mehr eingebaut werden. Das konkrete Vorgehen sollte man
jedenfalls mit den zuständigen Stellen abstimmen.}
Es ist außerdem sinnvoll -- schon allein aus Gründen der elektronischen
Archivierbarkeit -- auch die eigene Arbeit selbst als PDF-Datei beizulegen.%
\footnote{Auch Bilder und Grafiken könnten in elektronischer Form nützlich
sein, die \latex-Dateien sind hingegen überflüssig.}

Falls ein elektronischer Datenträger (CD-ROM, DVD) beigelegt wird, sollte auf
folgende Dinge geachtet werden:
%
\begin{enumerate}
	\item Jedem abzugebenden Exemplar muss eine identische Kopie des
	Datenträgers beiliegen.
	\item Verwenden Sie qualitativ hochwertige Rohlinge und überprüfen
	Sie nach der Fertigstellung die tatsächlich gespeicherten Inhalte
	des Datenträgers!
	\item Der Datenträger sollte in eine im hinteren Umschlag eingeklebte
	Hülle eingefügt sein und sollte so zu entnehmen sein, dass die Hülle
	dabei \emph{nicht} zerstört wird (die meisten Buchbinder haben geeignete
	Hüllen parat).
	\item Der Datenträger muss so beschriftet sein, dass er der
	Abschlussarbeit eindeutig zuzuordnen ist, am Besten durch ein
	gedrucktes Label%
	\footnote{Nicht beim lose abgegebenen Bibliotheksexemplar --
	dieses erhält ein standardisiertes Label durch die Bibliothek.}
	oder sonst durch \emph{saubere}	Beschriftung mit der Hand und einem
	feinen, wasserfesten Stift.
	\item Nützlich ist auch ein (grobes) Verzeichnis der Inhalte des
	Datenträgers (wie exemplarisch in Anhang \ref{app:materials}).
\end{enumerate}
\end{comment}
\chapter{Schlussbemerkungen}
\label{cha:Schluss}



%%%----------------------------------------------------------
\appendix                                            % Anhang 
%%%----------------------------------------------------------

\chapter{Technische Informationen}
\label{app:TechnischeInfos}

\newcommand*{\checkbox}{{\fboxsep 1pt%
\framebox[1.30\height]{\vphantom{M}\checkmark}}}

\section{Aktuelle Dateiversionen}

\begin{center}
\begin{tabular}{|l|l|}
\hline
Datum & Datei \\
\hline\hline
\hgbDate       & \texttt{hgb.sty} \\
\hline
\end{tabular}
\end{center}


\section{Details zur aktuellen Version}


Das ist eine völlig überarbeitete Version der DA/BA-Vorlage, die
\mbox{UTF-8} kodierte Dateien vorsieht und ausschließlich im PDF-Modus arbeitet.
Der "klassische" DVI-PS-PDF-Modus wird somit nicht mehr unterstützt!

\subsection{Allgemeine technische Voraussetzungen}

Eine aktuelle \latex-Installation mit
\begin{itemize}
		\item \texttt{biber}-Programm (BibTeX-Ersatz, Version $\geq 1.5$),
		\item \texttt{biblatex}-Paket (Version $\geq 2.5$, 2013/01/10),
		\item Latin Modern Schriften (Paket \texttt{lmodern}).%
			\footnote{\url{https://www.ctan.org/pkg/lm}, \url{https://tug.org/FontCatalogue/latinmodernroman/}}
\end{itemize}

Darüber hinaus ein Texteditor für \mbox{UTF-8} kodierte (Unicode) Dateien, sowie Software zum Öffnen und Betrachten von PDF-Dateien.


\subsection{Verwendung unter Windows}
\label{sec:VerwendungUnterWindows}

Eine typische Installation unter Windows sieht folgendermaßen aus:
%
\begin{enumerate}
\item \textbf{MikTeX}%
	\footnote{\url{https://miktex.org/} -- \textbf{Achtung:} 
	Generell wird die \textbf{Komplettinstallation} von MikTeX ("Complete MiKTeX") empfohlen, 
	da diese bereits alle notwendigen Zusatzpakete und Schriftdateien enthält. Hierzu wird der
	MikTeX Net Installer (im Gegensatz zum Basic Installer) benötigt.
	Bei der Installation ist darauf zu achten, 
	dass die automatische Installation erforderlicher Packages 
	durch "\emph{Install missing packages on-the-fly: = Yes}" ermöglicht wird (NICHT "\emph{Ask me first}")!
	Außerdem ist zu empfehlen, unmittelbar nach der Installation von MikTeX sowie in weiterer Folge regelmäßig
	mit dem Programm \texttt{MikTeX Console} ein Update der installierten Pakete durchzuführen.}
	(\latex-Basisumgebung),
\item \textbf{TeXstudio}%
	\footnote{\url{https://www.texstudio.org/}}
	(Editor, unterstützt UTF-8 und beinhaltet einen integrierten PDF-Viewer).
\end{enumerate}

Alternative Editoren und PDF-Viewer:
%
\begin{enumerate}
	\item TeXnicCenter,%
	\footnote{\url{https://www.texniccenter.org/}}
	\item Texmaker,%
	\footnote{\url{https://www.xm1math.net/texmaker/}}
	\item Lyx,%
	\footnote{\url{https://www.lyx.org/}}
	\item TeXworks,%
	\footnote{\url{https://www.tug.org/texworks/}}
	\item WinEdt,%
	\footnote{\url{https://www.winedt.com/}}
	\item Sumatra PDF (\latex-freundlicher PDF-Viewer).%
	\footnote{\url{https://www.sumatrapdfreader.org/}}
\end{enumerate}

\subsection{Verwendung unter Mac~OS}

Für Mac~OS empfiehlt sich die folgende Konfiguration:
%
\begin{enumerate}
\item 
	\textbf{MacTex}%
	\footnote{\url{https://tug.org/mactex/} -- \textbf{Achtung:} Aktuelle MacTeX-Distributionen verlangen in
		der Regel eine weitgehend aktuelle Version von Mac~OS. Auf älteren Betriebssystemen kann alternativ
		TeXLive mit einem speziellen Installationsscript installiert werden.
		Um die Pakete der \LaTeX-Distribution aktuell zu halten, sollte regelmäßig das \texttt{TeX Live Utility}
		ausgeführt werden.}
	(\latex-Basisumgebung),
\item \textbf{TeXstudio}%
	(Editor, unterstützt UTF-8 und beinhaltet einen integrierten PDF-Viewer).
\end{enumerate}

Alternative Editoren und PDF-Viewer:
%
\begin{enumerate}
	\item Texmaker,%
	\item Lyx,%
	\item TeXworks,%
	\item Skim (\latex-freundlicher PDF-Viewer).%
	\footnote{\url{https://skim-app.sourceforge.io/}}
\end{enumerate}

\subsection{Verwendung unter Linux}

Unter Linux kann folgendes Setup zum Einsatz kommen:
%
\begin{enumerate}
	\item 
	\textbf{TeX Live}%
	\footnote{\url{https://tug.org/texlive/} -- Eine Installation unter Linux erfolgt -- abhängig von der verwendeten
	Distribution -- am einfachsten mit Hilfe des jeweiligen Paketverwaltungssystems (\zB \texttt{apt-get}).}
	(\latex-Basisumgebung),
	\item \textbf{TeXstudio}%
	(Editor, unterstützt UTF-8 und beinhaltet einen integrierten PDF-Viewer).
\end{enumerate}

Alternative Editoren und PDF-Viewer:
%
\begin{enumerate}
	\item Texmaker,%
	\item Lyx,%
	\item TeXworks,%
	\item qpdfview (\latex-freundlicher PDF-Viewer).%
	\footnote{\url{https://launchpad.net/qpdfview}}
\end{enumerate}

\subsection{Verwendung von Online-Editoren}

Neben einer lokalen \latex-Installation mit Editor gibt es mittlerweile auch Online-Editoren, die das Arbeiten mit \latex-Dokumenten
im Browser ermöglichen. Die \latex-Basisumgebung ist dabei auf den Servern des Dienstes installiert, Dokumente können im Editor neu
erstellt oder auch bestehende Vorlagen (wie etwa dieses Dokument) hochgeladen und weiter bearbeitet werden. Die meisten Plattformen
ermöglichen darüber hinaus ein kollaboratives Arbeiten an einem Dokument.

Der bekannteste und mit dieser Vorlage getestete Editor ist \emph{Overleaf}\footnote{\url{https://www.overleaf.com/}}. Um schnell
Vorlagendokumente aus dem \texttt{hagenberg-thesis} Paket zu importieren, können die Import-Links in der Readme zum Repository
dieser Vorlage\footnote{\url{https://github.com/Digital-Media/HagenbergThesis}} verwendet werden.

Alternativ existieren noch weitere Online-Editoren:
%
\begin{enumerate}
	\item Papeeria,%
	\footnote{\url{https://papeeria.com/}}
	\item CoCalc.%
	\footnote{\url{https://cocalc.com/}}
\end{enumerate}
	% Technische Ergänzungen
\chapter{Inhalt der CD-ROM/DVD}
\label{app:cdrom}

	% Inhalt der CD-ROM/DVD
\chapter{Chronologische Liste der Änderungen}


Diese Auflistung wird nicht mehr aktualisiert.
Siehe \emph{Commits} auf \url{https://github.com/Digital-Media/HagenbergThesis}.



\begin{comment}
\begin{sloppypar}
\begin{description}
%
\item[2002/01/07]
\verb!\newfloat{program}! repariert (auch ohne Chapter). Dank an Werner Bailer!
%
\item[2002/03/06]
Copyright-Notice an internat.\ Standard angepasst. Dank an Karin Kosina!
%
\item[2002/07/28]
"`Studiengang"' $\rightarrow$ "`Diplomstudiengang"'
%
\item[2003/08/24]
Neues Macro: \verb!\Messbox{breite}{hoehe}! -- zur Kontrolle der 
Druckgröße ohne PS-Datei. Erweiterungen für Bakkalaureatsarbeiten
%
\item[2005/04/09]
Diverse Korrekturen: Captions von Tabellen nach oben gesetzt. 
\texttt{caption} auf neue Versionen adaptiert.
\texttt{subfig} wird nicht mehr verwendet
%
\item[2006/01/20]
Adaptiert zur Verwendung als Praktikumsbericht 
(2.\ Bakk.-Arbeit)
%
\item[2006/03/24]
Fehler in \verb!\erklaerung! beseitigt (Dank an David Schwingenschlögl)
%
\item[2006/04/06]
Verwendung von T1-Fontencoding zur besseren Silbentrennung bei 
Umlauten etc.
%
\item[2006/06/21]
Neu: Bachelorstudiengang / Masterstudiengang.
Literaturverweise auf Bakk-Arbeiten.
\texttt{upquote.sty} eliminiert (Problem mit TS1-Kodierung).
Verwende Komma (statt Punkt) als Trennzeichen in Dezimalzahlen.
%
\item[2006/09/14]
Anmerkungen zum Thema Plagiarismus.
%
\item[2007/07/16]
Ergänzungen für Code-Listings (listings) und Algorithmen 
(\texttt{algorithmicx}).
BiBTeX-Datei aufgeräumt, Verwendung der Literaturformate 
verbessert.
Komma (statt Punkt) als Trennzeichen in Dezimalzahlen wieder 
entfernt.
Verwendung der \texttt{ae}-Fonts eliminiert (\texttt{cm-super} Fonts müssen 
installiert sein, ab MikTeX 2.5). 
Beispiel für Ersetzung in EPS-Dateien mit \texttt{psfrag}.
%
\item[2007/10/04]
Version 5.90: Das Laden der Pakete \verb!inputenc! (Option \texttt{latin}) und 
\verb!graphicx! (Option \texttt{dvips})
aus der Hauptdatei in die \texttt{sty}-Datei übertragen; \texttt{upquote} funktioniert nun.
Paket \texttt{eurosym} ergänzt für Euro-Symbol (Anregung von Andreas 
Doubrava).
Problem mit \texttt{color}-package repariert (gerasterter PDF-Ausdruck).
Hinweise bzgl.\ Literatur ergänzt (\texttt{month}, \texttt{edition}),
BibTeX-Datei gesäubert.
Hinweis zum Einfügen von vertikalem Abstand zwischen Absätzen.
Mathematik aufgeräumt, Verwendung von \texttt{amsmath}, 
Fallunterscheidungen.
Diverse Änderungen bei Tabellen und Programmkode.
Beispiele für BibTeX-Angaben von Spezialquellen: Audio-CDs, 
Videos, Filme. Einbinden von Dateien mit \verb!\include{..}!
Neue Datei: \verb!_SimpleReport.tex! für kurze Reports (Projekte etc.).
%
\item[2007/11/11]
Version 5.91: Hinweise zur Einstellung der Output-Profile in
TexNicCenter, Inverse Search Einstellung in YAP im Anhang.
%
\item[2008/04/01]
Version 6.00beta -- kompletter Umbau!
Auslagerung der Doku\-menten-relevanten Teile in eine eigene 
\emph{class}-Datei (\texttt{hgbthesis.cls}) mit Optionen.
Die neue Style-Datei \texttt{hgb.sty} ist nun unabhängig vom 
Dokumententyp und nicht mehr kompatibel mit älteren Versionen!
Die Liste der Änderungen ist jetzt in der Datei \verb!_ChangeLog.tex!
(DIESE Datei) und diese wird im Anhang eingebunden.
Heading-Style auf Sans Serif geändert (ohne grausliche "`Caps"').
%
\item[2008/05/22]
Neue Vorlage für Technical Reports (Klasse \texttt{hgbreport.cls}).
Spracheinstellung nunmehr mit \texttt{babel}-Paket, Hauptsprache
des Dokuments kann als Option der Klasse angegeben werden.
Sprachumschaltung innerhalb des Dokuments funktioniert nun
richtig. Mit der Sprachoption \texttt{german} wird automatisch die neue deutsche 
Orthographie (\texttt{ngerman}) verwendet.
\texttt{babelbib} wird zur Formatierung des Literaturverzeichnisses
verwendet (neue BibTeX-Style-Optionen!).
Header werden nunmehr mit \texttt{fancyhdr}-Paket erzeugt.
Versionsnummerierung von \texttt{.cls} und \texttt{.sty} Files wird beendet 
(ab jetzt gilt: \emph{Datum} = \emph{Version}). 
%
\item[2008/06/10]
Neues Listing-Environment \texttt{PhpCode}; bei allen Listing-Eviron\-ments ist nun 
\texttt{mathescape=false} (kein Math-Mode nach \verb!$!). 
Bug bei Sprachumschaltung auf \texttt{ngerman} beseitigt.
%
\item[2008/08/15]
Diverse Kleinigkeiten in Literaturangaben überarbeitet (Dank an Norbert Wenzel), Spracheinstellung vereinheitlicht, Umlaute in \texttt{.bib}-Datei ersetzt.
%
\item[2008/10/15] 
Zusätzliche Hinweise zur MikTeX-Installation (Windows) sowie LaTeX unter Mac OS~X und Linux.
Liste der Abkürzungen ergänzt.%
\item[2008/11/15] 
Diverse Schreibfehler korrigiert (Dank an Silvia Fuchshuber). Hinweis auf 
\texttt{sloppypar}-Umgebung.
%
\item[2008/12/09] 
BibTeX-Tools: neuer Hinweis auf JabRef ergänzt, BibEdit entfernt (ist nicht mehr auffindbar).
%
\item[2009/02/09]
\texttt{hgb.sty}: Option "`\texttt{spaces}"' zu \texttt{url}-Package ergänzt (ermöglicht gezielten Zeilenumbruch in URLs). 
Im allgemeinen Setup für \texttt{listings}: \texttt{keepspaces=true};
Obsoletes Environment \texttt{sourcecode} deaktiviert.
Escape-Mode für \texttt{LaTeXCode}-Umgebung geändert.
\verb!_DaBa.tex!: Hinweis auf die Verwendung von \verb!\urldef! für die Angabe von URLs in Captions. \texttt{diplom} (statt \texttt{master}) als Standard-Dokumententyp in \verb!_DaBa.tex! ("`Diplomarbeit"'). Neuer Abschnitt zum Umgang mit ``Quellenangaben in Captions''.
\texttt{literatur.bib}: alle URLs (bisher in \texttt{note}-Einträgen) auf \verb!url={..}! geändert.
%
\item[2009/04/14]
Hinweis zum Einfügen einfacher Anführungszeichen ergänzt.
%
\item[2009/07/18]
Literaturangaben korrigiert und ergänzt.
%
\item[2009/11/27]
Experimentelle Version: Massive Änderungen, Umstieg auf \texttt{pdflatex}.
%
\item[2010/06/15]
Erstes Release der neuen Version mit \texttt{pdflatex}.
\item[2010/06/23]
Konflikt zwischen \texttt{pdfsync}-Package und \texttt{array}-Package (wird relativ häufig benutzt) durch \verb!\RequirePackage[novbox]{pdfsync}! behoben.
Seitenunterkante durch \verb!\flushbottom! fixiert,
variablen Absatzzwischenraum reduziert.
\item[2010/07/27]
Sprache der Erklärungsseite auf "`\texttt{german}"' fixiert (auch wenn die Hauptsprache des Dokuments  Englisch ist). %Datumsproblem - Hinweis von Philipp Winter
\item[2010/12/03]
Anmerkungen und Beispiele zum Zitieren von Gesetzestexten und Videos (Zeitangabe) ergänzt.
Hinweis auf \verb!\nolinkurl{..}! zur Angabe von Dateinamen.
\item[2011/01/29]
Einbau der Creative Commons Lizenz und entsprechender Hinweis in 
Abschnitt \ref{sec:HagenbergEinstellungen}. Neue Makros
\verb!\strictlicense!,
\verb!\cclicense! und
\verb!\license{...}!.
BibTeX-Einträge für Audio-CDs und Filme korrigiert, Beispiel für Online-Video ergänzt.
\item[2011/02/01]
Neues Makro \verb!\betreuerin{..}! zur Angabe einer (weiblichen) Betreuerin. 
%
\item[2011/06/26]
Umstellung der gesamten Literaturverwaltung auf \texttt{biblatex} mit dem Ziel, 
getrennte Abschnitte für verschiedene Kategorien von Einträgen im Quellenverzeichnis
zu ermöglichen. Die Wahl fiel auf \texttt{biblatex} (es gäbe andere Optionen), weil
damit BibTeX weiterhin nur einmal aufgerufen werden muss (und nicht für
mehrere Dateien). Damit verbunden sind allerdings massive Änderungen bei der
Syntax der BibTeX-Felder und es gibt auch mehrere neue Felder.
Aktuell sind 3 Kategorien von Quellen vorgesehen, entsprechende Änderungen in 
\nolinkurl{hgbthesis.cls}. Der klassische BibTeX-Workflow wird aktuell nicht
mehr unterstützt, die Möglichkeit einer künftigen Dok-Option ist aber 
vorgesehen. Das Literatur-Kapitel ist komplett überarbeitet, die .bib-Datei
wurde ausgemistet. Neu ist die Empfehlung zur Aufnahme von Bildquellen
in das Quellenverzeichnis, womit lange URLs in Captions (dort sind keine
Fußnoten möglich) nicht mehr notwendig sind. 
"`Persönliche Kommunikation"' als Literaturquelle entfernt (den Inhalt
von Interviews sollte man direkt im Anhang wiedergeben).
Das verwendete Bildmaterial wurde
erneuert, aktuell werden nur mehr Public Domain Bilder verwendet. 
Das Kapitel "`Hinweise für Word-Benutzer"' wurde endgültig entfernt.
\verb!\flushbottom! wieder auf \verb!\raggedbottom! geändert, um übermäßige 
Abstände zwischen Absätzen zu vermeiden.
%
\item[2012/05/10]
Hinweis auf die in Österreich bislang nicht zulässige Verwendung von "`Masterarbeit"'
entfernt, \texttt{master} ist nunmehr die Default-Dokumentenoption.
Anmerkungen zu lästigen \texttt{biblatex}-Warnungen ergänzt.
Angaben für Windows-Programmpfade auf Win7 angepasst, 
MikTeX 2.9 als Minimalerfordernis.\newline
Überflüssige Makros \verb!\damonat! und \verb!\dajahr! endgültig entfernt, statt
\verb!\abgabemonat! und \verb!\abgabejahr! ist nun das neue Makro
\verb!\abgabedatum{yyyy}{mm}{dd}! vorgesehen (unter Verwendung von internen Zählern).
Zur Formatierung von Datumsangaben wir das \texttt{datetime}-Paket verwendet.
\newline
Neue Fassung der eidesstattlichen Erklärung (inkl.\ englischer Version).\newline
PDF-Suche auf \texttt{synctex} umgestellt (\texttt{pdfsync}-Paket ist veraltet und
wird nun nicht mehr verwendet).
\newline
Die älteren Dateiversionen von \texttt{algorithmicx.sty} und \texttt{alg\-pseudo\-code.sty}
(bisher explizit beigefügt) wurden weggelassen.
\newline
Hinweis auf die \emph{Latin Modern Roman} OTF-Schriften ergänzt.
%
\item[2012/07/21]
Quellenverzeichnis: sprachabhängige Einstellung der Überschriften eingerichtet.
Titel des Quellenverzeichnisses auf "`Quellenverzeichnis"' (DE) \bzw\ "`References"' (EN) 
fixiert. Makro \verb!\MakeBibliography! hat damit keinen erforderlichen Parameter mehr.
%
\item[2012/09/17]
Wegen Änderungen im \texttt{biblatex}-package (Version 1.7, 2011/11/13) die Verwendung von
BibTeX als backend eingestellt (\texttt{backend=bibtex8}).
%
\item[2012/10/13]
Option \texttt{lowtilde} beim URL-package eingestellt (erzeugt \url{~} statt \verb!~!).
%
\item[2012/12/01]
In Abschnitt \ref{sec:FormatierungVonProgrammcode} zusätzliche Code-Umgebungen ergänzt:
\texttt{JsCode},
\texttt{PhpCode},
\texttt{HtmlCode},
\texttt{CssCode},
\texttt{XmlCode}.
%
\item[2012/12/08]
Die Code-Umgebungen in Abschn.\ \ref{sec:FormatierungVonProgrammcode} ergänzt und 
zur Verwendung von optionalen Argumenten erweitert (Hinweise in Abschnitt 
\ref{sec:FormatierungVonProgrammcode} auf die Argumente
\texttt{firstnumber=last} und \texttt{numbers=none}).
Quellenverzeichnis: den Eintragstyp \texttt{@software} für Games empfohlen und im Verzeichnis
der Kategorie \emph{avmedia} zugeordnet (Tab.~\ref{tab:BibKategorien} ergänzt). 
Game-Beispiel (von Manuel Wieser) und zusätzliche Tabelle \ref{tab:QuellenUndEintragstypen}
zur besseren Übersicht eingefügt.
%
\item[2013/05/17]
Wichtigste Änderung ist die vollständige Umstellung auf \textbf{UTF-8} unter Beibehaltung des 
\texttt{pdflatex}-Workflows. 
Damit sind zahlreiche weitere Modifikationen verbunden:
\newline
Alle Dateien (auch \texttt{.cls}, \texttt{.sty} und \texttt{.bib}) wurden auf UTF-8 konvertiert.
Damit sollte es auch keine Probleme mehr mit Umlauten und Sonderzeichen unter MacOS geben.
\newline
Die verwendete Standard-Schriftfamilie ist nun "`Latin Modern"' (\texttt{lmodern}). 
Sie ersetzt die "`CM-Super"' Schriften, mit denen es immer wieder Installationsprobleme gab.
Weiters wird jetzt das \texttt{cmap}-Paket zur besseren Such- und Kopierbarkeit von PDFs verwendet.
\newline
Das \texttt{listings}-Paket wurde durch \texttt{listingsutf8} ersetzt und für Umlaute im Quellcode adaptiert.
Eventuell sind weitere Adaptierungen notwendig.
\newline
\texttt{biber} (min.\ Version 1.5!) wird nun anstatt \texttt{bibtex} (unterstützt keine UTF-8 Dateien) verwendet,
zusammen mit \texttt{biblatex} (Version 2.5).
Die Anweisung \verb!\bibliography! wird (wieder) verwendet, allerdings nun in der Präambel,
um die \texttt{.bib}-Datei im Fileverzeichnis anzuzeigen.
\newline
Das Makro \verb!\C! (für die Menge der komplexen Zahlen \Cpx) musste wegen Problemen in der T1-Kodierung
ersetzt werden und heißt nun \verb!\Cpx!. Die Makros 
\verb!\R!, \verb!\Z!, \verb!\N!, \verb!\Q! und \verb!\Cpx! können nun auch außerhalb des Mathematik-Modus verwendet werden.
\newline
Der DVI-PS-PDF Workflow wird ab dieser Version überhaupt nicht mehr unterstützt. 
Damit ist auch das \texttt{psfrag}-Paket nicht mehr verwendbar. Entspechende Hinweise 
wurden aus dem Text entfernt.
\newline
\texttt{hyperref} wurde auf UTF-8 umgestellt.
Die grässlichen Standard-Rahmen und Farben der automatischen \texttt{hyperref}-Links wurden entfernt \bzw\ durch 
dezentere Farben ersetzt. Dadurch wird auch die Screen-Version der PDFs wieder lesbar.
\newline
Im Quellenverzeichnis wurde versuchsweise die \texttt{backref}-Option aktiviert. 
Damit werden bei allen Einträgen auch die zugehörigen Zitierstellen angegeben
(erscheint durchaus sinnvoll).
\newline
Die bisherigen Korrekturen zur \texttt{biblatex}-Formatierung wurden entfernt, 
alles arbeitet nun mit Standard-Einstellungen. Die ursächlichen Probleme in \texttt{biblatex}
scheinen in der aktuellen Version behoben zu sein.
\newline
Das Output-Profil für TeXnicCenter wurde für den neuen Workflow mit \texttt{biber} adaptiert und liegt nun in
\nolinkurl{_tc_output_profile_sumatra_utf8.tco}.
\newline
Das Windows-Script \verb!_clean.bat! wurde entfernt, da TeXnicCenter nun ein eigenes "`Clean Project"'-Kommando aufweist (in "`Build"').
\newline
Allgemeine Einstellungen zu \emph{headings} und \emph{biblatex} wurden aus der Datei \texttt{hgbthesis.cls} entfernt und in 
\texttt{hgbheadings.sty} \bzw\ \texttt{hgbbib.sty} verlagert. Diese können nun unabhängig verwendet werden (s.\ Beispiel in 
\texttt{\_TermReport.tex}).
\newline
Die Klassen-Datei \texttt{hgbtermreport.cls} wurde eliminiert, das Dokument \texttt{\_TermReport.tex} basiert nunmehr
auf der generischen LaTeX-Klasse \texttt{report}  und verwendet keine eigene \texttt{.cls} Datei mehr.
%
\item[2014/11/05]
Neu: Logo auf der Frontseite bei allen Dokumententypen. Dazu gibt es ein neues Kommando
\verb!\logofile{pic}!, wobei \verb!pic! der Name eine PDF-Datei im
Verzeichnis \verb!images/! ist. Falls \emph{kein} Logo erwünscht ist, 
kann man die Zeile einfach weglassen oder durch \verb!\logofile{}! ersetzen.
\newline
\texttt{hyperref}-Einstellungen: Einfärbung der Links wieder entfernt (\texttt{colorlinks = false}), weil beim Druck
nicht abschaltbar. Stattdessen einheitliche (dezente) Rahmen für alle Linkarten.
Zahlreiche Tippfehler eliminiert (Dank an Daniel Karzel).
\newline
Wegen eines Bugs in \texttt{biblatex 1.9} wurden die expliziten Abteilungen (\verb!\-!) in \texttt{literatur.bib}
vorübergehend entfernt (mit entsprechenden Folgen im Ergebnis). Der Bug soll in \texttt{biblatex 2.0} (derzeit noch
nicht verfügbar) behoben sein.
\newline
Package \texttt{color} auf \texttt{xcolor} geändert. In \texttt{hgb.sty} neues "`Convenience-Makro"' \verb!\etc! ergänzt.
Output-Profil für TeXnicCenter/SumatraPDF (Windows) repariert, forward/inverse Search funktioniert nun
(Datei \verb!_tc_output_profile_sumatra_utf8.tco!).
%
\item[2015/04/28]
Paket \texttt{subdepth} (zur verbesserten Platzierung von Sub- und Superscripts) 
in hgb.sty ergänzt.
%
\item[2015/07/14]
Hinweis und Abhilfe für die (nicht automatische) Silbentrennung in zusammengesetzten Wörtern.
Neu in \texttt{hgbheadings.sty}: \verb!\RequirePackage[raggedright]{titlesec}! verhindert Blocksatz
in Section-Überschriften (sehr unschön bei längeren Überschriften). 
Neu (in Abschn.~\ref{sec:GraphicOverlays}): Beispiel für die Verwendung des \texttt{overpic}-Pakets
zur Annotierung von importierten Grafiken (verwendet zudem das \texttt{pict2e}-Paket).
%
\item[2015/08/03]
Logo-Datei auf \texttt{logo.pdf} umbenannt.
\item[2015/09/17]
Anweisung \verb!\RequirePackage[utf8]{inputenc}! in die Doku\-menten\-dateien (\texttt{\_xxx.tex})
verschoben (auf Anregung von Markus Kohm: "`\ldots für die Verwendung von lualatex oder xelatex 
ist die Anweisung in hgb.sty störend, da bei diesen beiden aufgrund der nativen utf8-Unterstützung 
\texttt{inputenc} keinesfalls verwendet werden darf"').
\item[2015/09/19]
\texttt{hgb.sty} aufgeräumt.
Makros \verb!\@savesymbol! und \verb!\@restoresymbol! aus \texttt{hgb.sty} entfernt
(wurden nicht mehr verwendet; ggfs.\ Paket \texttt{savesym} als Ersatz).
Makro \verb!\optbreaknh! (optional break with no hyphen) auf \verb!\obnh! umbenannt.
Teile von \texttt{hgb.sty} in neue Dateien \texttt{hgbabbrev.sty} (div.\ Abkürzungen)
und \texttt{hgblistings.sty} (Code-Listings) verschoben.
Hintergrundtönung der Code-Listings heller (auf 5\% Grau) eingestellt.
Layout: \verb!\textfraction! auf 0.1 (statt fehlhafterweise 0.01) eingestellt.
\texttt{hgbbib.sty}: \verb!\clearpage! am Beginn des Quellenverzeichnisses entfernt
(für \texttt{article}-Template).
\item[2015/09/19]
Alle \texttt{.cls} und \texttt{.sty} Dateien sind jetzt ANSI-codiert (Header eingefügt), wie
laut CTAN-Richtlinien vorgesehen. Umlautzeichen wurden durch Makros ersetzt.
Nur \texttt{hgblistings.sty} ist weiterhin UTF-8 (wegen notwendiger literaler Umlaute).
\verb!\RequirePackage[utf8]{inputenc}! steht sonst nur mehr am Beginn
der jeweiligen (\texttt{.tex}) Haupttextdatei.
\item[2015/10/29]
Verwendung von "`In:"' im Quellenverzeichnis vor \texttt{article}-Einträgen
(Eigenart von biblatex) durch passendes Makro in \texttt{hgbbib.sty} unterbunden 
(Dank an S.\ Dreiseitl).
\item[2015/11/04]
Hinweise in Abschnitt \ref{sec:Software} auf TeXstudio unter Windows, Mac OS und Linux.
Release-Ausgabe.
\item[2015/12/08]
Source Directories neu strukturiert in \texttt{frontmatter}, \texttt{chapters}, 
\texttt{appendix}.
\item[2016/06/09]
Bibliography-Aliases für die Quellentypen
\texttt{video}, \texttt{movie}, \texttt{audio} und \texttt{software}
eingefügt (in \texttt{hgbbib.sty}) -- unterbindet Warnungen wegen
fehlender biblatex-Driver.
\item[2016/06/11]
Repository portiert auf GitHub (SourceForge eingefroren).  
Overleaf als experimentelle online LaTeX-Umgebung.
Hauptdateien umbenannt (auf \texttt{\_thesis}, \texttt{\_praktikum}, etc.).
\item[2016/09/28]
"`Numerierung"' auf "`Nummerierung"' geändert.
Code-Einbindung im Anhang repariert.
\item[2016/10/06]
In \texttt{hgb.sty}: Makros \verb!\Frametext! und \verb!\FramePic! eliminiert (ersetzt durch \verb!\fbox{...}!),
dazu \verb!\fboxsep! global auf Null gesetzt.
Hinweise auf \texttt{subfig}-Paket entfernt.
\item[2016/10/07]
In \texttt{hgbbib.sty}: Zeilenumbrüche bei URLs im Quellenverzeichnis werden an beliebigen Zeichen ermöglicht.
\end{description}
\end{sloppypar}
\end{comment}



%\section*{To Do} 
%\begin{itemize}
%\item Anhang B (CD-ROM Inhalt) überarbeiten -- ist nicht aktuell!
%\item Inkscape
%\item biblatex Bib-Driver für audio, video etc. ergänzen.
%\item Mathematik umbauen, typische Fehler stärker berücksichtigen (ua. Leerzeilen vor/nach Gleichungen).
%\item Literaturempfehlungen zum Schreiben von Diplomarbeiten
%\item Hinweise für Literatursuche (Bibliotheksverbund, CiteSeer,...)
%\end{itemize}





	% Chronologische Liste der Änderungen
\chapter{\latex-Quellkode}
\label{app:latex}

\section*{Hauptdatei {\tt\_thesis.tex}}

\paragraph{Anmerkung:}
Das sollte nur ein \emph{Beispiel} für die Einbindung von Quellcode
in einem Anhang sein. Der \latex-Quellkode der eigenen
Abschlussarbeit ist meist \emph{nicht} interessant genug, um ihn hier
wiederzugeben!

\begin{footnotesize}
\verbatiminput{_thesis.tex}
\end{footnotesize}





	% Quelltext dieses Dokuments

%%%----------------------------------------------------------


\makeatletter
\g@addto@macro{\backmatter}{
	\renewcommand{\chaptermark}[1]{%
		\markboth{#1}{}%
	}    
}
\makeatother
\backmatter

\MakeBibliography                        % Quellenverzeichnis
%\addcontentsline{toc}{chapter}{Abkürzungen}
\printacronyms
%\printacronyms[heading=chapter]
%\printacronyms[heading=chapter*]
%%%----------------------------------------------------------

%%% Messbox zur Druckkontrolle ------------------------------
\chapter*{Messbox zur Druckkontrolle}



\begin{center}
{\Large --- Druckgröße kontrollieren! ---}

\bigskip

\calibrationbox{100}{50} % Angabe der Breite/Hoehe in mm

\bigskip

{\Large --- Diese Seite nach dem Druck entfernen! ---}

\end{center}



%%%----------------------------------------------------------
\end{document}
%%%----------------------------------------------------------