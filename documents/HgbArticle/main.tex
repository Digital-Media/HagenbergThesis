%% A simple template for a scientific article using the Hagenberg setup
%% based on the standard LaTeX 'article' class
%%% äöüÄÖÜß  <-- no German umlauts here? Use a UTF-8 compatible editor!

%%% Magic comments for setting the correct parameters in compatible IDEs
% !TeX encoding = utf8
% !TeX program = pdflatex 
% !TeX spellcheck = en_US
% !BIB program = biber

\RequirePackage[utf8]{inputenc} % Remove when using lualatex or xelatex!
\RequirePackage{hgbpdfa}        % Creates a PDF/A-2b compliant document

\documentclass[english,twocolumn,smartquotes]{hgbarticle}
% Valid options in [..]: 
%    Main language: 'german' (default), 'english'
%    Turn on smart quote handling: 'smartquotes'
%    APA bibliography style: 'apa'
%    Typeset the text in two columns: 'twocolumn'
%%%-----------------------------------------------------------------------------

\usepackage{lipsum}       % Create lorem ipsum dummy text (can be removed)
\flushbottom              % Add vertical space where necessary to fill the page
\graphicspath{{images/}}  % Location of images and graphics
\bibliography{references} % Biblatex bibliography file (references.bib)

%%%-----------------------------------------------------------------------------
\begin{document}
%%%-----------------------------------------------------------------------------

\author{
	Alex A.\ Wiseguy\\ 
	\email{wiseguy@gmail.com}
	\and
	Adrian Spongehead\\
	\email{spongehead@gmx.net}
}
\title{Reasoning About the Unreasonable}
\date{}

%%%-----------------------------------------------------------------------------
\maketitle
%%%-----------------------------------------------------------------------------

\begin{abstract}\noindent
This document shows how to author an \emph{article} using the full functionality
of the \texttt{hagenberg-thesis} framework. It uses class \texttt{hgb\-article} which is based
on the standard \latex \textsf{article} class. This may be useful, for example, if you wish to
publish part of your thesis as a journal article or conference paper without major changes.
Note that this is not intended as a substitute for \latex's standard \texttt{article}
class and many publishers provide their own \latex\ class files.
\end{abstract}

%%%-----------------------------------------------------------------------------

\section{Introduction}

If you wish to write this report in German, you should replace the specification
%
\begin{verbatim}
\documentclass[english,twocolumn,
            smartquotes]{hgbarticle}
\end{verbatim}
%
at the top of this document by
%
\begin{verbatim}
\documentclass[german,twocolumn,
    smartquotes]{hgbarticle}
\end{verbatim}
%
Remove the \texttt{twocolumn} option for a single-column layout. If you do not
wish for simplified quotations, remove the \texttt{smartquotes} document option.

Also note that the topmost sectioning command available in the \texttt{article}
class is \verb!\section{}!, i.e., there are \textbf{no chapters}!
When converting a thesis, you thus need to \emph{demote} all chapter commands to
sections, sections to subsections etc.

\subsection{Mathematical Elements}

This document is typeset in the typical two-column format required by many
scientific journals. Unfortunately, the narrow text width often creates
problems with mathematical structures \cite{Voss2014}. While smaller
equations like
%
\begin{equation}
\bar{\mathbf{M}} =  
\begin{pmatrix}
	A \ast H^{G}_{\sigma}   & C \ast H^{G}_{\sigma} \\
	C \ast H^{G}_{\sigma}   & B \ast H^{G}_{\sigma} 
\end{pmatrix}
=
\begin{pmatrix}
	\bar{A}   & \bar{C} \\
	\bar{C}   & \bar{B} 
\end{pmatrix}
\end{equation}
%
may fit without any modification, larger structures like the one shown in
Equation \ref{wideEquation} need special treatment.
%
\begin{figure*}[t]
	\begin{equation}
		\begin{split}
			\lambda_{1,2}
			&= \frac{\mathrm{tr}(\bar{\mathbf{M}})}{2} \pm \sqrt{\Bigl
			(\frac{\mathrm{tr}(\bar{\mathbf{M}})}{2}\Bigr)^2
			- \mathrm{det}(\bar{\mathbf{M}})}  \\
			&= \frac{1}{2} \cdot \left( \bar{A} + \bar{B} \pm \sqrt{\bar{A}^2 -
			2 \cdot \bar{A} \cdot \bar{B} + \bar{B}^2 + 4 \cdot \bar{C}^2}
			\right)
			,
		\end{split}
		\label{wideEquation}
	\end{equation}
\end{figure*}
%
In this case, the equation was wrapped into a \verb!\begin{figure*}! \ldots
\verb!\end{figure*}! environment, which produces an unnumbered float object
that extends across the full page width. Additional details can be found in
the source text.

\subsection{Graphics and Images}

Similarly, the horizontal space available for graphical elements is the width
of a single column. Thus elements inserted with \verb!\includegraphics{..}! 
should use the length \verb!\columnwidth! for scaling. An example is shown in
Fig.~\ref{fig:CocaCola} (see the source text for details).

\begin{figure}
	\centering
	\includegraphics[width=1.0\columnwidth]{cola-public-domain-photo-p}
	\caption{Coca-Cola advertisement photographed in 1940 \cite{CocaCola1940}.}
	\label{fig:CocaCola}
\end{figure}

And by the way, the \texttt{lipsum} package was used to create the following
dummy texts. 

\subsection{Bibliography}

The use of citations and the compilation of the bibliography work much the
same as in the thesis template. The only difference is that
\verb!\printbibliography! is used directly at the end of the document.

%%%-----------------------------------------------------------------------------

\section{Existing Techniques}

\lipsum[2-4]

%%%-----------------------------------------------------------------------------

\section{Our Radically New Approach}

\lipsum[5-7]

\subsection{Initial Steps}

\lipsum[5-7]

\subsection{Alternatives}

\lipsum[5-7]

\subsection{Future Activities}

\lipsum[5-7]

%%%-----------------------------------------------------------------------------

\section{Summary and Conclusion}

\lipsum[8-9]

%%%-----------------------------------------------------------------------------
\printbibliography % alternatively: \MakeBibliography[nosplit]
%%%-----------------------------------------------------------------------------

%%%-----------------------------------------------------------------------------
\end{document}
%%%-----------------------------------------------------------------------------
