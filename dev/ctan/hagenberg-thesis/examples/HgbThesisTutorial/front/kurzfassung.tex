\chapter{Kurzfassung}

An dieser Stelle steht eine Zusammenfassung der Arbeit, Umfang
max.\ 1 Seite. Im Unterschied zu anderen Kapiteln ist die
Kurzfassung (und das Abstract) üblicherweise nicht in Abschnitte
und Unterabschnitte gegliedert. 
Auch Fußnoten sind hier falsch am Platz.

Kurzfassungen werden übrigens häufig -- zusammen mit Autor und Titel
der Arbeit -- %
in Literaturdatenbanken aufgenommen. Es ist daher darauf zu
achten, dass die Information in der Kurzfassung für sich 
\emph{allein} (\dah ohne weitere Teile der Arbeit) zusammenhängend und
abgeschlossen ist. Insbesondere werden an dieser Stelle (wie \ua
auch im \emph{Titel} der Arbeit und im \emph{Abstract})
normalerweise \emph{keine Literaturverweise} verwendet! Falls
unbedingt solche benötigt werden -- etwa weil die Arbeit eine
Weiterentwicklung einer bestimmten, früheren Arbeit darstellt --,
dann sind \emph{vollständige} Quellenangaben in der Kurzfassung
selbst notwendig, \zB %
[\textsc{Zobel} J.: \textit{Writing for Computer Science -- The Art of
Effective Commu\-nica\-tion}. Springer-Verlag, Singa\-pur, 1997].

Auch sollte daran gedacht werden, dass bei der Aufnahme in Datenbanken
Sonderzeichen oder etwa Aufzählungen mit "Knödellisten" in der
Regel verloren gehen. Dasselbe gilt natürlich auch für das 
\emph{Abstract}.


Inhaltlich sollte die Kurzfassung \emph{keine} Auflistung der
einzelnen Kapitel sein (dafür ist das Einleitungskapitel
vorgesehen), sondern dem Leser einen kompakten, inhaltlichen
Überblick über die gesamte Arbeit verschaffen. Der hier verwendete
Aufbau ist daher zwangsläufig anders als der in der Einleitung.
