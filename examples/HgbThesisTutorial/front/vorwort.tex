\chapter{Vorwort} 	% engl. Preface


Dies ist \textbf{Version \hgbDate} der \latex-Dokumentenvorlage für 
verschiedene Abschlussarbeiten an der Fakultät für Informatik, Kommunikation
und Medien der FH Oberösterreich in Hagenberg, die mittlerweile auch 
an anderen Hochschulen im In- und Ausland gerne verwendet wird.

Das Dokument entstand ursprünglich auf Anfragen von Studierenden,
nachdem im Studienjahr 2000/01 erstmals ein offizieller
\latex-Grundkurs im Studiengang Medientechnik und -design an der
FH Hagenberg angeboten wurde. Eigentlich war die Idee, die bereits
bestehende \emph{Word}-Vorlage für Diplomarbeiten "einfach" in
\latex\ zu übersetzen und dazu eventuell einige spezielle
Ergänzungen einzubauen. Das erwies sich rasch als wenig
zielführend, da \latex, \va was den Umgang mit Literatur und
Grafiken anbelangt, doch eine wesentlich andere Arbeitsweise
verlangt. Das Ergebnis ist -- von Grund auf neu geschrieben und
wesentlich umfangreicher als das vorherige Dokument --
letztendlich eine Anleitung für das Schreiben mit \latex, ergänzt
mit einigen speziellen (mittlerweile entfernten) Hinweisen für \emph{Word}-Benutzer.
Technische Details zur aktuellen Version finden sich in Anhang \ref{app:TechnischeInfos}.

Während dieses Dokument anfangs ausschließlich für die Erstellung
von Diplomarbeiten gedacht war, sind nunmehr auch  
\emph{Masterarbeiten}, \emph{Bachelor\-arbeiten} und \emph{Praktikumsberichte} 
abgedeckt, wobei die Unterschiede bewusst gering gehalten wurden.

Bei der Zusammenstellung dieser Vorlage wurde versucht, mit der
Basisfunktionalität von \latex das Auslangen zu finden und -- soweit möglich --
auf zusätzliche Pakete zu verzichten. Das ist nur zum Teil gelungen;
tat\-säch\-lich ist eine Reihe von ergänzenden "Paketen" notwendig, wobei jedoch
nur auf gängige Erweiterungen zurückgegriffen wurde.
Selbstverständlich gibt es darüber hinaus eine Vielzahl weiterer Pakete,
die für weitere Verbesserungen und Finessen nützlich sein können. Damit kann
sich aber jeder selbst beschäftigen, sobald das notwendige Selbstvertrauen und
genügend Zeit zum Experimentieren vorhanden sind.
Eine Vielzahl von Details und Tricks sind zwar in diesem Dokument nicht explizit
angeführt, können aber im zugehörigen Quelltext jederzeit ausgeforscht
werden.

Zahlreiche KollegInnen haben durch sorgfältiges Korrekturlesen und
konstruktive Verbesserungsvorschläge wertvolle Unterstützung
geliefert. Speziell bedanken möchte ich mich bei Heinz Dobler für
die konsequente Verbesserung meines "Computer Slangs", bei
Elisabeth Mitterbauer für das bewährte orthographische Auge und
bei Wolfgang Hochleitner für die Tests unter Mac~OS.

Die Verwendung dieser Vorlage ist jedermann freigestellt und an
keinerlei Erwähnung gebunden. Allerdings -- wer sie als Grundlage
seiner eigenen Arbeit verwenden möchte, sollte nicht einfach
("ung'schaut") darauf los werken, sondern zumindest die
wichtigsten Teile des Dokuments \emph{lesen} und nach Möglichkeit
auch beherzigen. Die Erfahrung zeigt, dass dies die Qualität der
Ergebnisse deutlich zu steigern vermag.

Dieses Dokument und die zugehörigen \latex-Klassen sind seit Nov.\ 2017 auf CTAN%
\footnote{Comprehensive TeX Archive Network} 
als Paket \texttt{hagenberg-thesis} verfügbar unter
%
\begin{itemize}
\item[]\url{https://ctan.org/pkg/hagenberg-thesis}.
\end{itemize}
%
Den jeweils aktuellen Quelltexte sowie zusätzliche Materialien findet man unter
%
\begin{itemize}
\item[]\url{https://github.com/Digital-Media/HagenbergThesis}.%
\footnote{Unter \url{https://github.com/Digital-Media/HagenbergThesis/blob/master/CHANGELOG.md}
sowie genauer unter \url{https://github.com/Digital-Media/HagenbergThesis/commits/master} 
findet man auch eine (früher im Anhang dieses Dokuments enthaltene) chronologische Auflistung der 
Änderungen.}
\end{itemize}



\noindent
Trotz großer Mühe enthält ein Dokument wie dieses immer Fehler und Unzulänglichkeiten
-- Kommentare, Verbesserungsvorschläge und sinnvolle Ergänzungen
sind daher willkommen, am einfachsten als Kommentar oder Fehlermeldung ("Issue") 
auf GitHub oder jederzeit auch per E-Mail an
%
\begin{itemize}
\item[]
Dr.\ Wilhelm Burger, Department für Digitale Medien,\newline
Fachhochschule Oberösterreich, Campus Hagenberg (Österreich)\newline
\nolinkurl{wilhelm.burger@fh-hagenberg.at}
\end{itemize}

\noindent
Übrigens, hier im Vorwort (das bei Diplom- und Masterarbeiten üblich, bei Bachelorarbeiten 
aber entbehrlich ist) kann kurz auf die Entstehung des Dokuments eingegangen werden.
Hier ist auch der Platz für allfällige Danksagungen (\zB an den Betreuer, 
den Begutachter, die Familie, den Hund, \ldots), Widmungen und philosophische 
Anmerkungen. Das sollte man allerdings auch nicht übertreiben und auf 
einen Umfang von maximal zwei Seiten beschränken.




